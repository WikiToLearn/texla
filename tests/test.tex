 
\begin{itemize}
	\item \verb|\newtheorem{teo}{Teorema}|
	\begin{teo}[Teoremone bello ciccione]
		testo del teoremone che ti spaccamalissimo proprio
	\end{teo}
	\begin{proof}
		dopo i teoremi bisogna sempre mettere la dimostrazione, altrimenti gli studenti pignoli non credono alle tue parole
	\end{proof}
	\begin{ciao}
		ciaociao\end{ciao
			\begin{itemize}
				ciao
			\end{itemize}
			\item \verb|\newtheorem{cor}{Corollario}|
			\begin{cor}[Nome del Corollario]
				Di solito dopo un teorema viene sempre un corollario, e spesso è più chiaro il corollario del teorema stesso
			\end{cor}
			\begin{proof}
				anche i corollari hanno la dimostrazione, ma è molto breve.
			\end{proof}
		\end{itemize}
		
		\begin{itemize}
			ciaocia
		\end{itemize}