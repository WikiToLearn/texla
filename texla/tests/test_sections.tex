

\begin{document}

\textbf{BIOFISICA}

Laura D'Alfonso, uff. 4044 U1, laura.dalfonso@mib.infn.it

Physical Chemistry -- Eisemberg and Crothers

Biophysical Chemistry -- Cantor and Schimmel

\textbf{Proteine e acidi nucleici}

Sono macromolecole (molecole grosse), di \(10^{3} - 10^{12}\) amu.
Interagiscono e spostano le molecole del solvente in cui sono immerse
(acqua..), sono flessibili e hanno scale di tempi e distanze particolari
da considerare nel momento dell'osservazione.

L'acqua è il solvente principale, è polare, fa legami H. Tutte le
molecole sono da considerare in interazione con l'acqua (ad eccezione
delle molecole cristallizzate).

Le forze che determinano la struttura delle macromolecole:

\begin{itemize}
\item
  Interazioni covalenti

  \begin{itemize}
  \item
    Legame a H: \(6\frac{\text{kcal}}{\text{mol}}\)
  \end{itemize}
\end{itemize}

Per descrivere l'interazione interatomica e intermolecolare si usa il
potenziale di Lennard-Jones: a distanze piccole si generano forze
repulsive molto intense, a corto raggio d'azione

\[V\left( r \right) = 4\varepsilon\left\lbrack \left( \frac{\sigma}{r} \right)^{12} - \left( \frac{\sigma}{r} \right)^{6} \right\rbrack\]

\includegraphics[width=3.73819in,height=2.62083in]{media/image1.emf}

\(\varepsilon\) è la profondità della buca di potenziale e \(\sigma\) è
il diametro di atomi/molecole.

La parte che va come la sesta potenza è il contributo attrattivo delle
forze di Van der Waals (interazioni deboli) e prevale a grandi distanze;
la parte in potenza di 12 descrive le forze repulsive a corto raggio fra
i nuclei (a corto raggio non sono più schermati dagli elettroni) e fra
gli elettroni stessi (si respingono quando a coppie tendono ad occupare
gli stessi numeri quantici).

\[RT = 0,6\frac{\text{kcal}}{\text{mol}} = 2,5\frac{\text{kJ}}{\text{mol}}\]

\begin{itemize}
\item
  Interazioni non-leganti

  \begin{itemize}
  \item
    Elettrostatiche
  \item
    Van der Waals
  \end{itemize}
\item
  Effetti di solvente e interazioni idrofobiche
\end{itemize}

Struttura macromolecole

\begin{itemize}
\item
  Struttura primaria. Sequenza di amminoacidi, o acidi nucleici. Sono
  direzionali perché gli estremi sono diversi. Nel DNA hanno due gruppi
  fosfati diversi alle estremità, le proteine hanno gruppo amminico e
  gruppo carbossilico.
\item
  Struttura secondaria
\item
  Struttura terziaria
\item
  Struttura quaternaria
\end{itemize}

\textbf{Virus }

Collana di proteine che, con la sua struttura quaternaria, racchiude la
parte centrale che contiene il DNA (ha una struttura quaternaria: la
doppia elica) del virus.

\textbf{Proteine }

Ci sono 20 amminoacidi essenziali. Esistono anche delle forme modificate
che li rendono 21/22.

Non tutti hanno la stessa frequenza in natura, non è detto trovarli
tutti nella stessa proteina. Il primo amminoacido conosciuto è
l'aspargina.

Classificazione amminoacidi: in base a carica e caratteristiche chimiche
(pK)

\begin{itemize}
\item
  Acidi
\item
  Basici. Fondamentali perché la carica della proteina dipenderà dalla
  carica degli aa che la costituiscono.
\item
  Importanti dal punto di vista conformazionale. Glicina (aa base) e
  prolina (sostituisce senza conseguenze altri aa). Hanno la
  caratteristica di avere su un estremo un gruppo azotato e sull'altro
  un ossigeno e poi una serie di carboni in mezzo. Tutti gli altri aa
  aggiungono a questa struttura base un gruppo R (residuo).
\item
  Aromatici. Interessanti dal punto di vista spettroscopico grazie
  all'anello.
\item
  Alifatici.
\end{itemize}

\textbf{Legame peptidico}. Crea la struttura primaria, tiene insieme gli
aa. Il carbonio legato al gruppo R è detto C-\(\alpha\).

\includegraphics[width=2.90614in,height=2.40734in]{media/image2.png}

\textbf{Configurazioni trans e cis}. La configurazione cis ha problemi
di ingombro volumico (sterico).

La catena laterale sarà disposta in modo tale da formare un certo angolo
(angolo diedro) non casuale, lo si vede nel Ramachandran Plot:

\includegraphics[width=4.41111in,height=2.47284in]{media/image3.jpeg}

Alcuni aa sono cruciali per la proteina (forma e funzione), altri
possono essere sostituiti senza creare problemi. La funzione è
conseguenza della struttura, che è determinata dal tipo di aa e dal tipo
di legame. Una sostituzione fra aa molto diversi in forma e dimensione
ovviamente comporta un cambiamento complessivo nella proteina, se invece
si fanno sostituzioni con aa analoghi, complessivamente la modifica non
avrà risultati pesanti.

Studi funzionali sono stati fatti e ci indicano quali sostituzioni
possono essere fatte.

Ci sono tante proteine che sono analoghe in noi e negli animali. La
beta-lactoglobulina esiste in vari animali ma non esiste nell'uomo, la
sua funzione è svolta da altre proteine.

La struttura secondaria più comune delle proteine è quella elicoidale,
ma non è l'unica, poi ci sono parametri che posti 0 o valori semplici
danno ad esempio la forma lineare o la β-sheet.

Nella struttura β-foglietto le interazioni avvengono sui tratti
paralleli, quindi fra aa anche lontani nella proteina.

\textbf{Mioglobina}. Presenza del gruppo EME (prostetico), protetto
dalla struttura della proteina ma comunque accessibile dal solvente,
perché l'O lo deve raggiungere (si occupa del trasporto dell'O).

Questione relaxed-tight e legame struttura-funzione.

Emoglobina. 4 subunità in posizione specifica. Legame allosterico. Il
legame ha un effetto sulla conformazione. Quando l'O si lega alla
proteina essa cambia la sua conformazione in modo da avvolgerlo e
conservarlo. L'O si potrà liberare solo sotto determinate condizioni di
pressione, una volta arrivato dove deve arrivare (polmoni).

La mioglobina non mostra legame cooperativo.

Curve di ossigenazione.

\textbf{Costante di associazione semplice. }

\[O_{2} + M \rightarrow MO_{2}\]

\[K_{a} = \frac{\left\lbrack MO_{2} \right\rbrack}{\left\lbrack M \right\rbrack\left\lbrack O_{2} \right\rbrack},\ M_{\text{tot}} = \left\lbrack MO_{2} \right\rbrack + \left\lbrack M \right\rbrack \rightarrow K_{a} = \frac{\left\lbrack MO_{2} \right\rbrack}{\left\lbrack O_{2} \right\rbrack\left( M_{\text{tot}} - \left\lbrack MO_{2} \right\rbrack \right)} \rightarrow\]

\[\left\lbrack MO_{2} \right\rbrack = \frac{K_{a}M_{\text{tot}}\left\lbrack O_{2} \right\rbrack}{1 + K_{a}\lbrack O_{2}\rbrack} \rightarrow \left\lbrack MO_{2} \right\rbrack - \frac{K_{a}\left\lbrack MO_{2} \right\rbrack\left\lbrack O_{2} \right\rbrack}{1 + K_{a}\left\lbrack O_{2} \right\rbrack} = \frac{K_{a}\left\lbrack O_{2} \right\rbrack\left\lbrack M \right\rbrack}{1 + K_{a}\lbrack O_{2}\rbrack}\]

\textbf{Acidi} \textbf{nucleici}

Le basi azotate sono 4: adenina, timina, citosina, guanina. Il legame
avviene su coppie selezionate di basi (AT, GC). La sequenza di basi è
casuale, se ne trova una complementare, si legano a formare una sorta di
scala. Una volta appaiate, assumono una configurazione spaziale non
lineare: a doppia elica (e questa è proprio sempre un'elica).

La struttura terziaria del DNA è abbastanza varia e si forma per
interazione fra istoni; ci sono strutture terziarie non canoniche di
tutti i tipi: hairpin, cruciform, folded, DNA superavvolti etc.

Cellule. Nucleo più spesso e parte circostante responsabile della
chemiotassi (? spostamento).

\textbf{Tecniche spettroscopiche di immagine}

\begin{itemize}
\item
  Diffrazione a raggi X. Comodo sui cristalli, perché per il resto i
  raggi X distruggono la proteina.
\end{itemize}

\begin{quote}
Protein data bank: banca dati gratuita che contiene tutti i file sulle
proteine apribili con quel programma che ci mette sull'e-learning.
\end{quote}

\begin{itemize}
\item
  Assorbimento, fluorescenza, dicroismo circolare. 360-780 nm fascia del
  visibile.
\end{itemize}

\includegraphics[width=3.30417in,height=1.52778in]{media/image4.gif}

\begin{itemize}
\item
  FTIR. Spettroscopia infrarossa. Si vedono le vibrazioni dei legami.
  Nell'IR si può fare anche spettroscopia Raman.
\item
  Spettroscopia nel radio. NMR.
\end{itemize}

\begin{longtable}[c]{@{}llll@{}}
\toprule
Banda & Transizioni visibili & Lunghezza d'onda & Energia\tabularnewline
\midrule
\endhead
UV & Elettroniche & 190-360nm & 6eV\tabularnewline
Vis & Elettroniche & 360-780nm & 2eV\tabularnewline
IR & Vibrazionali & 780-1500 nm & 0.1eV\tabularnewline
Radio & Spin nucleare & Metri & 12000eV\tabularnewline
\bottomrule
\end{longtable}

\textbf{Sistemi termici}.

\begin{itemize}
\item
  Isolato. Non scambia calore né massa con l'esterno.
\item
  Chiuso. Scambia solo energia.
\item
  Aperto. Scambia energia e massa.
\item
  Adiabatico. Non scambia calore né massa.
\end{itemize}

Dalton: \(1D = 1,66 \times 10^{- 27}\text{kg}\)

Proteine: \(10^{3}D\), DNA: \(10^{6}D\)

\textbf{Molarità. }

\[1M = peso\ molecolare\ \left\lbrack \frac{g}{L} \right\rbrack\]

\textbf{Principio di conservazione della massa}:

\(C_{\text{stock}}\) è la concentrazione nota della soluzione proteica.

\[m = C_{\text{stock}}V_{\text{stock}} = C_{\exp}V_{\exp}\]

\[V_{\exp} = V_{\text{stock}} + V_{\text{solvente}}\]

\textbf{pH}

\[pH = - \operatorname{}\left\lbrack H_{3}O^{+} \right\rbrack\]

Caso dell'acqua:

\[\left\lbrack H_{3}^{+}O \right\rbrack + \left\lbrack HO^{-} \right\rbrack = \left\lbrack H_{2}O \right\rbrack\]

\[K_{w} = \left\lbrack H_{3}O^{+} \right\rbrack\left\lbrack HO^{-} \right\rbrack = 1,0 \times 10^{- 14}M^{2}\]

\[\delta = 1000\frac{g}{L} = 1\frac{g}{\text{mL}},peso\ molecolare = 18D \rightarrow Molarita = 55.5M\]

\[pH = 0 \rightarrow \left\lbrack H_{3}^{+}O \right\rbrack = 1M\]

\[pH = 9 \rightarrow \left\lbrack H_{3}O^{+} \right\rbrack = 1nM\]

Il nostro pH fisiologico è 7.4 ma non è detto che tutto ciò che
interagisce con noi abbia tale pH.

\includegraphics[width=6.69306in,height=1.16646in]{media/image5.png}

pH basso, pH neutro, pH alto

A pH bassi c'è un'alta concentrazione di \(H^{+}\) quindi le cariche
saranno positive, a pH alto c'è alta concentrazione di \(OH^{-}\) quindi
cariche negative.

\textbf{Titolazioni acido-base}

Equazione di Anderson-Hasselbach

La costante di dissociazione è un indice della probabilità che una
reazione avvenga.

\[\left\lbrack \text{AH} \right\rbrack \rightarrow^{K_{d}}\left\lbrack A^{-} \right\rbrack + \left\lbrack H_{3}O^{+} \right\rbrack\]

\[K_{d} = \frac{\left\lbrack A^{-} \right\rbrack\left\lbrack H_{3}^{+}O \right\rbrack}{\lbrack AH\rbrack}\]

\[pH = - Log_{10}\left\lbrack H_{3}O^{+} \right\rbrack\]

\[pK = - \operatorname{Log}K_{d}\]

\[pK - pH = Log_{10}\frac{\left\lbrack \text{AH} \right\rbrack}{\left\lbrack A^{-} \right\rbrack}\]

\[\left\lbrack \text{AH} \right\rbrack + \left\lbrack A^{-} \right\rbrack = A_{0}\]

\[pK - pH = Log_{10}\frac{\left\lbrack \text{AH} \right\rbrack}{A_{0} - \left\lbrack \text{AH} \right\rbrack}\]

L'andamento della concentrazione protonata AH è in funzione del pH una
volta che conosco la pK.

\[\left\lbrack \text{AH} \right\rbrack = A_{0}\frac{10^{pK - pH}}{1 + 10^{pK - pH}}\]

Questa cosa ha andamento sigmoidale.

Uso: proteina fluorescente come indicatore di pH intracellulare non
invasivo.

\textbf{TERMODINAMICA CHIMICA}

Riferimento di energia
\(KT = \frac{1}{40}eV = 0,025\ eV,\ RT = 0,6\frac{\text{kcal}}{\text{mol}}\)

Lavoriamo in sistemi aperti e non isolati a temperatura e pressione
costante.

Cap 4-5 Heisenberg

\textbf{I componenti chimici della cellula}

\begin{itemize}
\item
  \textbf{Composti inorganici}

  \begin{itemize}
  \item
    Acqua
  \item
    Ioni minerali

    \begin{itemize}
    \item
      Cationi: \(Na^{+},\ K^{+}\)
    \item
      Anioni: \(Cl^{-},\ SO_{4}^{- -}\)
    \end{itemize}
  \end{itemize}
\end{itemize}

Elementi più comuni: H,C,N,O

\begin{itemize}
\item
  \textbf{Composti organici}

  \begin{itemize}
  \item
    Carbonio. Due tipi di orbitali \(sp^{2},\ sp^{3}\), la differenza
    sono gli angoli di legame, cambia la geometria degli orbitali quindi
    la geometria della molecola.
  \item
    Ossigeno. 8 elettroni, una sola configurazione \(sp^{3}\). Ci sono
    due orbitali molecolari liberi quindi ci sono due possibilità di
    legame.
  \end{itemize}
\end{itemize}

\textbf{I legami}

\begin{itemize}
\item
  Legame \textbf{covalente}: scambio paritario di elettroni (messi in
  condivisione). In base a come viene effettuata la condivisione ci sono
  due tipi di legame covalente:

  \begin{itemize}
  \item
    Polare
  \item
    Apolare
  \end{itemize}
\end{itemize}

\begin{quote}
La polarità è in genere legata alla simmetria della molecola.
\end{quote}

\begin{itemize}
\item
  Legame \textbf{ionico}: cessione definitiva di un elettrone da un
  atomo a un altro, secondo l'elettronegatività (da meno elettronegativo
  -- donatore - a più elettronegativo - accettore)
\item
  Legame \textbf{H}: l'O (più elettronegativo) tende a attrare
  l'elettrone dell'H. H debolmente positivo, O debolmente negativo. La
  conformazione delle molecole d'acqua è caratterizzata e stabilizzata
  da questo legame (angolo particolare). In generale riguarda H e un
  altro atomo debolmente elettronegativo. Questo legame tiene anche
  insieme molecole diverse (legame intermolecolare).
\end{itemize}

\begin{quote}
I legami fra molecole d'acqua avvengono su direzioni preferenziali.

Se vogliamo sciogliere sale in acqua, la molecola del sale si scinde e
si infila nel network dell'acqua.

Una delle cose di cui è responsabile questo legame è la tensione
superficiale.

È un legame più debole (0.2eV) del legame covalente (5eV), ma sempre
stabile.

L'acqua si può dissociare (ionizzare) in ione idronio e ione ossidrile
\end{quote}

\[H_{2}0 \rightarrow H_{3}O^{+} + OH^{-}\]

\begin{quote}
Il legame H è fondamentale nella stabilizzazione delle proteine, in cui
si instaura nella struttura secondaria e caratterizza α-eliche e
β-sheets. Stabilizza le interazioni e ``congela'' la struttura.
\end{quote}

\begin{itemize}
\item
  \textbf{Interazioni idrofobiche}: si instaura fra molecole idrofile e
  idrofobiche. Molecole polari sono idrofile e tendono a circondare gli
  ioni e altre molecole polari. Molecole apolari non hanno interazioni
  elettrostatiche preferenziali con il solvente quindi tendono a
  interagire fra di loro, perché questa configurazione è favorita
  energeticamente. Se gruppi idrofobi sono circondati da acqua tendono
  ad aggregarsi perché così perturbano meno la struttura dell'acqua (i
  legami H).
\end{itemize}

\begin{quote}
Sapone: tensioattivo. In acqua crea micelle perché in questa condizione
minimizza l'interazione con l'acqua.
\end{quote}

\textbf{Energia libera di Gibbs}

\[G = H - TS\]

Il principio di massimizzazione dell'entropia valeva per un sistema
isolato:

\[\Delta S \geq 0\]

All'equilibrio abbiamo un massimo: \(\Delta S = 0\)

Anche per l'energia libera c'è un criterio di massimizzazione, ma vale
per sistemi aperti, quindi in generale è molto più utile.

Per un sistema aperto (prima legge):

\[\Delta G = \Delta H - T\Delta S \leq 0\]

All'equilibrio in un sistema aperto
\(\Delta G = 0 \rightarrow G_{i} = G_{f}\)

Consideriamo il nostro sistema immerso in un bagno molto più grande.
Consideriamo positivo il calore assorbito dal sistema \(Q > 0\) e
positivo il lavoro fatto dal sistema \(W > 0\).

\[\delta Q = dE + \delta W \rightarrow dE = \delta Q - \delta W\]

\(\mathbf{\text{dE}}\) perché l'energia interna è un differenziale
esatto in quanto è una funzione di stato.

\[\int_{A}^{B}{(\delta W - \delta Q)} = E_{B} - E_{A}\]

\textbf{Entropia} (seconda legge)

\[\delta Q_{\text{rev}} = TdS\]

\[\delta Q_{\text{irrev}} > TdS\]

Per una trasformazione reversibile:

\[TdS = dE + pdV\]

Se oltre al lavoro di espansione c'è un altro lavoro? Se c'è del lavoro
irreversibile o di altra natura ne dobbiamo tenere conto.

All'equilibrio (no scambi di energia) e in un sistema isolato:

\[TdS = dE + pdV - \delta W\]

L'energia libera di Gibbs per definizione:

\[G = H - TS = E + PV - TS\]

\textbf{Entalpia}: \(H = E + PV\)

\[dG = dE + PdV + VdP - TdS - SdT\]

\[dG = VdP - SdT + \delta W\]

\[- \delta W = - d{G|}_{P,T}\ \]

Oss. A temperatura e pressione costante la variazione di energia libera
è nulla.

Torniamo al nostro sistema+bagno.
\(m_{\text{bagno}} \gg m_{\text{sistema}}\), quindi possiamo considerare
il nostro sistema a T costante. L'entropia aumenta, e sarà la somma
dell'entropia del bagno termico e dell'entropia del nostro sistema.
\(Q_{P}\) calore assorbito dal sistema.

\[dS_{\text{bagno}} = \frac{\delta Q_{\text{rev}}}{T} = - \frac{Q_{P}}{T}\]

Poiché lavoriamo a pressione costante, il calore scambiato corrisponde
alla variazione di entalpia del sistema:

\[Q_{P} = \Delta H_{\text{sistema}} = \Delta E_{\text{sistema}} + P_{\text{sistema}}\Delta V \rightarrow \ dS_{\text{bagno}} = \frac{\delta Q_{\text{rev}}}{T} = - \frac{\Delta H_{\text{sistema}}}{T}\]

Questo per un sistema aperto.

Per l'universo invece:

\[\Delta S_{\text{bagno}} + \Delta S_{\text{sistema}} = \Delta S_{\text{sistena}} - \frac{\Delta H_{\text{sistema}}}{T} \geq 0\]

L'entropia dell'universo aumenta.

Moltiplico per T, costante:

\[T\Delta S_{\text{sistema}} - \Delta H_{\text{sistema}} = - \Delta G_{\text{sistema}} \geq 0\]

Questo rappresenta il criterio di spontaneità di una reazione.

\[\Delta G_{sistema\ a\ T,P\ costanti} \leq 0\]

Un sistema chiuso tende alla massimizzazione dell'entropia. Un sistema
aperto tende alla minimizzazione dell'energia libera.

Se c'è un lavoro non di espansione fatto sul sistema, il lavoro massimo
fatto dalle forze esterne è uguale alla variazione dell'energia libera
del sistema:

\[W_{\text{ext}} \geq \Delta G_{T,P} \rightarrow W_{\text{ext}}^{\max} = \Delta G_{\text{T.P}}\]

\textbf{Temperatura di melting} \(\mathbf{T}^{\mathbf{*}}\)

All'equilibrio e se non viene fatto lavoro esterno \((\Delta G = 0)\),
consideriamo ghiaccio in fusione:

\[\Delta G = \Delta H_{\text{fusione}} - T^{*}\Delta S_{\text{fusione}} = 0\]

\[T^{*} = \frac{\text{ΔH}}{\Delta S}\]

\(\Delta G\) varia al variare della T linearmente (lineare decrescente).

\(\Delta G_{A \rightarrow B} = G_{B} - G_{A} > 0\) significa che è
favorito lo stato A (stabile). Se \(\Delta G_{A \rightarrow B} < 0\) è
favorito lo stato B.

Il melting del ghiaccio è analogo al DNA melting. Anche in questo caso
viene definita una temperatura di melting, esattamente nello stesso
modo.

Nel melting del DNA si ha il passaggio dal duplex (A - ghiaccio) al
single strand (B - acqua).

\textbf{Energia libera per gas ideali}

A T costante, in assenza di lavoro, in un gas ideale

\[dG = VdP = nRT\frac{\text{dP}}{P}\]

\[\Delta G = nRT\ln\frac{P_{f}}{P_{i}} \rightarrow G_{i} = n_{i}G_{i}^{0} + \Delta G_{i} = n_{i}G_{i}^{0} + n_{i}\text{RT}\ln\frac{P_{i}}{P_{0}}\]

\[G_{i} = n_{i}G_{i}^{0} + n_{i}\text{RT}\ln P_{i},\ P_{0} = 1\ atm\]

\textbf{Energia libera per soluzioni ideali}

Scegliamo come stato di riferimento non più la pressione (sarebbe
scomodo), ma la frazione molare. Quindi prendiamo come stato standard lo
stato a frazione molare unitaria:

\[G_{i} = n_{i}G_{i}^{*} + n_{i}\text{RT}\ln X_{i}\]

\(G_{i}^{*}\) è l'energia libera molare del componente \(i\) puro
(frazione molare unitaria), \(n\) il numero di moli.

In soluzioni diluite \(X_{i} \propto C_{i}\) e vale l'approssimazione

\[X_{B} = \frac{n_{B}}{n_{\text{tot}}} = \frac{n_{B}}{n_{A} + n_{B}} \sim \frac{n_{B}}{n_{A}}\]

\[C_{B} = \frac{n_{B}}{V_{\text{soluzione}}\left( L \right)} \sim \frac{n_{A}X_{B}}{V_{\text{soluzione}}\left( L \right)} = \frac{n_{A}X_{B}}{V_{A}} = \frac{10^{3}\rho_{A}}{M_{A}}X_{B}\]

Il \(10^{3}\) è per pareggiare le unità di misura visto che
\(1L = 1000\ cm^{3}\). \(\rho\) è la densità in \(g/cm^{3}\) e \(C\) è
la concentrazione in \(g/L\).

Essendo concentrazione e frazione molare proporzionali, possiamo
considerarle alternativamente.

\textbf{Legge di Raoult}

\[P_{i} = x_{i}P_{i}^{*}\]

Dove: \(P_{i}\) è la pressione di vapore, \(x_{i}\) è la frazione molare
\(x_{i} = n_{i}/n_{\text{tot}}\) e \(P_{i}^{*}\) è la pressione di
vapore che il componente avrebbe in assenza della altre componenti.

\textbf{Potenziale chimico}

Spesso noi cambiamo il numero di molecole/moli degli oggetti in
soluzione e cambia l'energia interna di ciascuna molecola. La variazione
di energia complessiva deve tenere conto di questa variazione: ecco lo
scopo del potenziale chimico. È la velocità di cambiamento dei
potenziali termodinamici con il numero di molecole.

Dalla definizione di energia libera:

\[dG = - SdT + VdP + \mathbf{\text{μ\ dn}}\]

\[\mu = \frac{\partial G}{\partial n}\]

La formula rimane la stessa dell'energia libera ma per mole unitaria:

\(\mu_{i} = \mu_{i}^{0} + RT\ln P_{i}\) gas ideali dipende dalla
pressione quindi lo stato standard sarà quello a pressione unitaria,
idem gli altri:

\(\mu_{i} = \mu_{i}^{*} + \text{RT}\ln X_{i}\) soluzioni ideali

\(\mu_{i} = \mu_{i}^{\circ} + RT\ln C_{i}\) soluzioni ideali diluite

Soluto A in miscela acqua/solvente: a temperatura e pressione costanti
(all'equilibrio) \(\Delta G = 0\)

\[\mu_{A}\left( \text{water} \right) = \mu_{A}(solution)\]

La condizione di equilibrio per i potenziali chimici ci dice che (regola
della somma):

\[dG_{\text{TP}} = \sum_{i}^{}{\mu_{i}dn_{i}} = 0\]

\[G = n_{A}\mu_{A} + n_{A^{*}}\mu_{A}^{(*)}\]

\[dG = dn_{A}\left( \mu_{A} - \mu_{A}^{*} \right) = 0 \rightarrow \mu_{A} = \mu_{A}^{*}\]

Consideriamo due sistemi 1 e 2 con diverse temperatura, energia, numero
di molecole. Supponiamo che ci sia soltanto un flusso energetico
\(\mathbf{d}\mathbf{E}_{\mathbf{i}}\mathbf{\neq 0}\)\textbf{,}
\(\mathbf{\text{dn}}_{\mathbf{i}}\mathbf{= 0,\ dV = 0}\)

\[dS_{\text{tot}} = \sum_{i}^{}\frac{dE_{i}}{T_{i}} = \frac{dE_{1}}{T_{1}} + \frac{dE_{2}}{T_{2}} = dE_{1}\left( \frac{1}{T_{1}} + \frac{1}{T_{2}} \right)\]

Siccome
\(dS_{\text{tot}} \geq 0 \rightarrow dE_{1} > 0 \rightarrow T_{1} < T_{2}\).
Il flusso va dalla parte più calda a quella più fredda.

Se invece \(\mathbf{d}\mathbf{E}_{\mathbf{1}}\mathbf{= 0}\) ma
\(\mathbf{d}\mathbf{n}_{\mathbf{i}}\mathbf{\neq 0}\)

\[dS_{\text{tot}} = - \frac{1}{T}\left( \mu_{2}dn_{1} + \mu_{2}dn_{2} \right) = \frac{dn_{1}}{T}\left( \mu_{2} - \mu_{1} \right)\]

Anche qui
\(dS_{\text{tot}} \geq 0 \rightarrow dn_{1} > 0 \rightarrow \mu_{2} > \mu_{1}\).
Quindi il potenziale chimico regola la variazione di flusso della
materia (molecole).

In un \textbf{sistema a multicomponenti} abbiamo un potenziale chimico
per ogni specie presente.

La regola della somma vale anche per i volumi, oltre che per l'energia
libera, ma anche per le quantità parziali molari:

\[\mu_{i} = \left( \frac{\partial G}{\partial n_{i}} \right)_{T,P}\]

\[dG = \sum_{i}^{}{\mu_{i}dn_{i}}\]

\[\sum_{i}^{}{\mu_{i}n_{i}} = E + PV - TS = G\]

Relazione di Gibbs-Duhem:

\[\sum_{i}^{}{n_{i}d\mu_{i}} = 0\]

Il numero di gradi di libertà è inferiore al numero delle variabili. I
gdl sono 3 (P,T,\(\mu_{1}\) -- \(\mu_{2}\) non viene considerato perché
dipende dal valore di \(\mu_{1}\) tramite la relazione qui sopra). Vale
anche per sistemi multicomponenti: se ho 9 componenti avrò 10 gdl (P, T
e 8 potenziali chimici -- il nono dipende dagli altri 8).

\textbf{Legge di azione di massa}

Due stati A e B

\[G = \sum_{i}^{}{\mu_{i}^{\circ}n_{i}} + RT\sum_{i}^{}{n_{i}\ln C_{i}}\]

\[G = n_{A}\mu_{A} + n_{B}\mu_{B}\]

\[dG = 0 \rightarrow dn_{A} = - dn_{B}\]

\[dG = dn_{A}\left( \mu_{A} - \mu_{B} \right) = 0 \rightarrow \mu_{A} = \mu_{B}\]

\[\mu_{A}^{\circ} + RT\ln C_{A} = \mu_{B}^{\circ} + RT\ln C_{B}\]

\[\ln\frac{C_{B}}{C_{A}} = - \frac{\mu_{B}^{\circ} - \mu_{A}^{\circ}}{\text{RT}} = - \frac{\Delta G_{B \rightarrow A}^{\circ}}{\text{RT}} \rightarrow \frac{C_{B}}{C_{A}} = e^{- \frac{\Delta G_{B \rightarrow A}^{\circ}}{\text{RT}}} = K_{\text{eq}}\]

I potenziali standard sono tabulati. In condizioni standard:

\[\mathbf{\Delta}\mathbf{G}_{\mathbf{B \rightarrow A}}^{\mathbf{\circ}}\mathbf{= - RT}\ln\mathbf{K}_{\mathbf{\text{eq}}}\]

Se abbiamo più reagenti e più prodotti:

\[\sum_{i = 1}^{R}{\nu_{i}r_{i}} \Longleftrightarrow \sum_{i = R + 1}^{R + P}{\nu_{i}p_{i}}\]

dove \(\nu_{i}\) sono i coefficienti stechiometrici.

\[dG = dn\left( \sum_{i = 1}^{R}{\nu_{i}\mu_{i}} \right) - dn\left( \sum_{i = R + 1}^{R + P}{\nu_{i}\mu_{i}} \right) = 0\]

\[\Delta G_{R \rightarrow P}^{\circ} = - RT\ln K_{\text{eq}}\]

\textbf{Spontaneità delle reazioni}

Il principio di minimizzazione dell'energia interna ci fa capire in che
direzione ci muoviamo rispetto a una generica coordinata di reazione: ci
muoviamo sempre verso l'energia libera minore.

Esempio. Oligonucleotide (tipo DNA con poche coppie di basi) formato da
adenina e uracile. L'energia libera è molto più bassa se ho 12 coppie di
basi rispetto a se ho 10 coppie di basi. Questo perché le due coppie in
più rendono la molecola più stabile (effetto zip).

Conoscendo il valore dell'energia libera posso stimare il contributo
alla stabilità di ciascuna coppia di basi: se con n=5 (10 coppie di
basi) ho \(\Delta G^{0} = - 21\ kJ/mol\), con n=6 ho
\(\Delta G^{0} = - 30\ kJ/mol\). Quindi
\(\delta\Delta G^{0} = - 9\ kJ/mol\) quindi il contributo di una coppia
di basi è \(- 4.5\ kJ/mol\).

\textbf{Pressione} \textbf{osmotica}

È fondamentale per la sopravvivenza, ad esempio, dei globuli rossi in
condizioni saline anomale (eccesso o difetto).

\includegraphics[width=3.98958in,height=1.63542in]{media/image6.emf}

Attraverso una membrana semipermeabile il solvente tende ad entrare nel
contenitore.

\includegraphics[width=1.44792in,height=1.04167in]{media/image7.emf}La
pressione osmotica è la pressione necessaria per raggiungere
l'equilibrio. Se siamo all'equilibrio i potenziali chimici dei due stati
sono gli stessi. Il potenziale chimico del solvente avrà un termine di
riferimento, un termine dovuto alla frazione molare e un termine dovuto
alla pressione osmotica:

\[\mu_{A} = \mu_{A}^{\circ} + RT\ln x_{A} + \tilde{V_{A}}\Pi\]

\[\Pi = \rho gh\]

\[\mu_{A}^{(A + B)} = \mu_{A}^{(A)}\]

\[\frac{V}{n_{\text{tot}}}\Pi = RT\frac{n_{B}}{n_{\text{tot}}} \rightarrow \Pi V = n_{B}\text{RT}\]

Simile all'equazione dei gas ideali.

Se la pressione osmotica è troppo alta si creano dei buchi nella cellula
per diminuirla, e la cellula muore. Viceversa la cellula può essere
schiacciata se la pressione esterna è molto maggiore.

\textbf{pH di una soluzione}

\[pH = - \operatorname{}\left\lbrack H_{3}O^{+} \right\rbrack\]

Per l'acqua

\[\left\lbrack H_{3}O^{+} \right\rbrack + \left\lbrack HO^{-} \right\rbrack = \left\lbrack H_{2}O \right\rbrack\]

\[K_{w} = 1.0 \times 10^{- 14}M^{2}\]

\[\rho_{w} = 1000\frac{g}{L}\]

\[M_{w} = 18\ D\]

\[\left\lbrack H_{2}O \right\rbrack = 55.5\ M\]

\textbf{Reazione di ionizzazione}

\[HA \rightleftharpoons H^{+} + A^{-}\]

Costante di dissociazione

\[K_{d} = \frac{\left\lbrack H^{+} \right\rbrack\left\lbrack A^{-} \right\rbrack}{\left\lbrack \text{HA} \right\rbrack}\]

\[pK_{d} = - \operatorname{}K_{d}\]

Relazione con l'energia libera

\[\frac{\Delta G_{d}}{\text{KT}} = - \ln K_{d}\]

Che vale se la \(K_{d}\) è espresso tramite frazioni molari:

\[K_{d} = \frac{x_{A}x_{H}}{x_{\text{AH}}}\]

Con M specie diverse in soluzione la frazione molare è

\[x_{i} = \frac{n_{i}}{\sum_{k = 1}^{M}n_{k}}\]

Se invece la \(K_{d}\) è espresso come rapporto di concentrazioni molari
K ha una dimensione, quindi si fa una conversione:

\[x_{A} = \frac{n_{A}}{n_{\text{tot}}} = \frac{n_{A}V}{Vn_{\text{tot}}} = \frac{n_{A}}{N_{A}V}\frac{N_{A}V}{n_{\text{tot}}} = \frac{\left\lbrack A^{-} \right\rbrack}{\left\lbrack A^{-} \right\rbrack + \left\lbrack H^{+} \right\rbrack + \left\lbrack \text{AH} \right\rbrack}\]

\[\ln\left( \frac{\left\lbrack A^{-} \right\rbrack\left\lbrack H^{+} \right\rbrack}{\left\lbrack \text{AH} \right\rbrack} \right) = - \frac{\Delta G_{d}}{\text{KT}} + \ln\left( \left\lbrack A \right\rbrack_{\text{tot}} \right)\]

\[\left\lbrack A \right\rbrack_{\text{tot}} = \left\lbrack A^{-} \right\rbrack + \left\lbrack \text{AH} \right\rbrack\]

Con \(\lbrack X\rbrack\) si intende sempre la concentrazione di X in
molare (moli/litro di soluzione). Altrimenti c'è anche il molale
(moli/litro di solvente).

EQUILIBRIO ACIDO-BASE

\[\left\lbrack A \right\rbrack_{\text{tot}} = \left\lbrack \text{AH} \right\rbrack + \left\lbrack A^{-} \right\rbrack\]

\[K_{d} = \frac{\left\lbrack A^{-} \right\rbrack\left\lbrack H^{+} \right\rbrack}{\left\lbrack \text{AH} \right\rbrack} = \frac{\left( \left\lbrack A \right\rbrack_{\text{tot}} - \left\lbrack \text{AH} \right\rbrack \right)\left\lbrack H^{+} \right\rbrack}{\lbrack AH\rbrack} \rightarrow \operatorname{}\left( \frac{\lbrack AH\rbrack}{\left\lbrack A \right\rbrack_{\text{tot}} - \left\lbrack \text{AH} \right\rbrack} \right) = pK - pH\]

\[\left\lbrack \text{AH} \right\rbrack = \frac{\left\lbrack A \right\rbrack_{\text{tot}}}{1 + 10^{pH - pK}}\]

\textbf{Curve di protonazione}

Indicano come varia la concentrazione delle specie al variare del pH.

\[\left\lbrack \text{AH} \right\rbrack = A_{0}\frac{10^{pK - pH}}{1 + 10^{pK - pH}}\]

Se \(pH = pK\) la concentrazione delle due specie è uguale e pari alla
metà della concentrazione totale.

\includegraphics[width=2.90625in,height=2.06250in]{media/image8.emf}

Andamento sigmoidale definito da due parametri:

\begin{itemize}
\item
  Stipness (pendenza). Indica quanto la reazione è cooperativa. Più la
  curva è ripida più la reazione è brusca quindi priva di un regime
  intermedio (è del tipo ``tutto o niente'').
\item
  Punto di mezzo (midpoint).
\end{itemize}

Le basi cedono protoni e gli acidi accettano protoni.

Reazione di condensazione: formazione del legame + rilascio di acqua.

Reazione di condensazione inverse: separazione di oggetti + addizione di
acqua.

\textbf{Macromolecole}

Nelle macromolecole abbiamo 70\% acqua e 30\% sostanze chimiche.

Nelle cellule sono presenti 4 tipi di macromolecole: polimeri
(monomeri): polisaccaridi (zuccheri), grassi (acidi grassi), proteine
(aa), acidi nucleici (nucleotidi).

\includegraphics[width=4.22307in,height=2.90569in]{media/image9.emf}

\textbf{Carboidrati} (idrati di carbonio)

\[\left( \mathbf{C}\mathbf{H}_{\mathbf{2}}\mathbf{0} \right)_{\mathbf{n}}\]

Comprendono composti che hanno gruppi aldeidici (doppio legame con O e
singolo con H) e gruppi chetonici (3 C, quello centrale fa doppio legame
con l'O).

Funzioni:

\begin{itemize}
\item
  Fonte di energia: 4kcal/mol
\item
  Riserva energetica (amido , glicogeno)
\item
  Sostegno ai tessuti (cellulosa..)
\item
  Comunicazione all'interno delle cellule
\item
  Materiale per la sintesi di altri componenti (RNA, DNA)
\item
  Funzionalizzare proteine (glicoproteine)
\end{itemize}

Monosaccaride (molecola singola) più semplice: glucosio. Prodotto dalla
fotosintesi della \(CO_{2}\) nelle piante.

Polisaccaridi: macromolecole, polimeri che hanno unità ripetute di
monosaccaridi semplici.

Funzioni:

\begin{itemize}
\item
  Riserva: glicogeno, amido
\item
  Sostegno: cellulosa (pareti cellulari), chitina (esoscheletro insetti)
\end{itemize}

Tuttavia, a differenza di amido e glicogeno che possono essere
idrolizzati tramite enzimi, la cellulosa non può essere sintetizzata con
gli enzimi che possediamo.

\textbf{Lipidi}

\begin{itemize}
\item
  Insolubili in  in quanto idrofobici (questione di entropia,
  preferiscono formare micelle per minimizzare il contatto con l'acqua)
\item
  Si dividono in:

  \begin{itemize}
  \item
    Semplici: acidi grassi
  \item
    Complessi: fosfolipidi, fosfogliceridi, steroidi.
  \end{itemize}
\end{itemize}

Funzioni:

\begin{itemize}
\item
  Fonte di energia (9kcal/mol). Producono energia (calore)
\item
  Riserva energetica (trigliceridi) senza limiti
\item
  Strutturale (membrane biologiche, mielina)
\item
  Trasduzione del segnale inter e infra cellulare
\item
  Trasporto e regolazione sintesi vitamine
\item
  Antiossidante
\item
  Isolamento termico e protezione da traumi
\end{itemize}

\textbf{Acidi grassi}: catene di idrocarburi non ramificate (lineari)
con testa polare (idrofilica) e cosa apolare (idrofobica). Questa
differenza è responsabile della formazione delle micelle.

\begin{itemize}
\item
  Saturi: legami covalenti C-C (struttura rigida)
\item
  Insaturi: legami diversi che provocano ripiegamenti (meno efficiente
  nella formazione delle micelle).
\end{itemize}

Questi oggetti si uniscono a formare membrane (fosfolipidi).

I saturi riescono a fare una membrana bella fitta e compatta, mentre
quelli insaturi creano membrane con aperture fondamentale per il
passaggio di sostanze per le membrane.

\includegraphics[width=2.81667in,height=1.80077in]{media/image10.emf}

\textbf{PROTEINE}

\begin{itemize}
\item
  Contengono C, H, N, O
\item
  20 aa fondamentali di cui alcuni sono essenziali (non sintetizzati
  dall'organismo, quindi devono essere assunti tramite la dieta)
\end{itemize}

Funzioni:

\begin{itemize}
\item
  Struttura: collagene, cheratina
\item
  Catalizzatori: gli enzimi son proteine
\item
  Movimento: miosina e actina per i muscoli
\item
  Trasporto: emoglobina
\item
  Ormoni: insulina
\item
  Protezione: fibrinogeno, anticorpi
\item
  Scorta di nutrimento: caseina, ovalbumina, ferritina
\item
  Regolazione dell'espressione genica.
\end{itemize}

\textbf{Amminoacidi}

Possono agire da acidi o basi a seconda del valore del pH. Esistono in
soluzione come ioni dipolari. Tutti gli amminoacidi (tranne la glicina)
hanno almeno uno stereocentro e sono chirali (la maggior parte è
levogira), quindi rispondono otticamente con una diversa assorbanza alla
luce polarizzata circolarmente destra e sinistra.

Formano proteine tramite legame peptidico.

\begin{itemize}
\item
  Struttura \textbf{primaria}: sequenza di amminoacidi. Per fare una
  catena di \(n\) aa abbiamo \(20^{n}\) peptidi utilizzabili.
\end{itemize}

Esempio. Insulina. Formata da due catene polipeptidiche tenute insieme
da due ponti disolfuro intercatena. Tale legame è una reazione di
ossidazione fra due cisteine, a formare una cistina ed è non planare.

La sequenza degli aa è molto importante per la funzione: vasopressina e
ossitocina differiscono per un amminoacido nella catena ma hanno
funzioni completamente diverse.

\begin{itemize}
\item
  Struttura \textbf{secondaria}: disposizione degli aa in porzioni
  localizzate di una catena peptidica.
\end{itemize}

L'α elica è solitamente destrorsa (tutte nelle proteine), la catena si
avvolge a spirale. Tuttavia alcuni agglomerati di proteine possono
disporsi a formare eliche sinistrorse. L'N di un aa forma un legame H
con l'O del gruppo carbossilico distante di 4 posizioni della catena.
L'inizio e la fine sono C terminale e N terminale. L'interno è vuoto, i
residui si dispongono verso l'esterno (anche perché sono ingombranti).

α-elica:

\begin{itemize}
\item
  diametro 0,54nm
\item
  3,6 residui per giro
\end{itemize}

β-sheets. Strutture più estese con legami H fra aa distanti nella
catena. Possono essere paralleli o antiparalleli.

I residui sono alternativamente disposti verso l'interno e verso
l'esterno (sempre una questione entropica). Questo fa sì che gli angoli
di torsione abbiano valori preferenziali, rappresentati nel Ramachandran
plot:

\includegraphics[width=1.97917in,height=1.97227in]{media/image11.emf}

\begin{itemize}
\item
  Struttura \textbf{terziaria}. È la conformazione complessiva
  dell'intera catena polipeptidica. Stabilizza la struttura secondaria
  tramite legami covalenti, legame H, ponti salini (interazione
  NH3-COO), interazioni idrofobiche forma compatta della proteina.
\end{itemize}

\includegraphics[width=4.53478in,height=2.18341in]{media/image12.emf}

\begin{itemize}
\item
  Struttura \textbf{quaternaria}: più subunità legate in modo non
  covalente (facilmente distruttibile). Emoglobina: due catene da 141 aa
  e due da 146 aa, ciascuna catena circonda un gruppo eme contenente il
  Fe.
\end{itemize}

\textbf{FOLDING E DENATURAZIONE}

Perché la proteina sia biologicamente attiva è necessario che la catena
polipeptidica si ripieghi (fold) in una struttura 3D. Il folding
(ripiegamento) della catena è necessaria per:

\begin{itemize}
\item
  Formazione di elementi di struttura secondaria (foci di nucleazione)
\item
  Interazione fra nuclei per formare domini
\item
  Unione dei domini nel molten globule (struttura terziaria distorta)
\item
  Sistemazione delle distorsioni fine.
\end{itemize}

Errori nel folding comportano malattie (huntington, fibrosi cistica,
alzheimer).

Gli elementi cruciali nel folding sono i residui interni alla proteina
nello stato ripiegato, ovvero quelli idrofobici.

\textbf{Paradosso di Levinthal}. La proteina non può esplorare
casualmente lo spazio delle configurazioni accessibili: se ci sono n aa,
avremo \(2^{n}\) angoli di torsione ciascuno con 3 conformazioni stabili
quindi \(3^{2n} \sim 10^{n}\) conformazioni possibili se supponiamo di
esplorare tutte le conformazioni possibili il tempo necessario sarà
\(10^{n} \times 10^{- 13}s\) ovvero il numero delle conformazioni
possibili per il tempo necessario per la riorientazione del legame. Se
\(n = 100\) il tempo richiesto è enorme (\(10^{87}s\), vita
dell'Universo \(10^{17}s\))! Le proteine devono ripiegarsi più
velocemente.

Sperimentalmente invece si constata che il folding avviene in alcuni
secondi. La stabilità conformazionale della proteina aumenta ad ogni
passaggio del folding.

\begin{itemize}
\item
  Entro 5 ms: formazione di segmenti locali di struttura IIaria
\item
  5-1000ms: stabilizzazione della struttura IIaria e inizio della
  formazione della struttura IIIaria. Si formano sub-domini, si ha il
  collasso idrofobico e si forma il molten globule.
\item
  Secondi: ci sono riarragiamenti conformazionali (isomerizzazioni CIS
  TRANS), la proteina si compatta.
\end{itemize}

Nel dicroismo, durante lo stato di molten globule gli aromatici hanno
ancora la possibilità di muoversi, quindi lo spettro di dicroismo è
molto piatto ancora. Quando la proteina raggiunge lo stato
conformazionale finale, la sua chiralità è ben definita quindi da
segnale di dicroismo forte e definito.

Si passa da uno stato ad alta energia (ed alta entropia) della proteina
denaturata ad uno di bassa energia e basse entropia nel momento in cui
formiamo contatti di tipo nativo. Questo processo però non è continuo,
ci sono dei minimi locali in cui la proteina rimane intrappolata in
conformazioni intermedie (che non sono a energia minima assoluta);
tuttavia, se forniamo energia termica, queste barriere (piccole) possono
essere superate in favore di conformazioni ad energia minore e quindi si
arriva al minimo assoluto.

\includegraphics[width=2.94523in,height=3.00000in]{media/image13.emf}

Questo è il processo di folding. In lab noi possiamo fare solo il
contrario: farla unfoldare (denaturazione).

Per indurre la denaturazione:

\begin{itemize}
\item
  Temperatura
\item
  Agenti chimici denaturanti che si legano preferenzialmente ad alcuni
  aa, rompendo la struttura secondaria.
\end{itemize}

La denaturazione è in generale un processo reversibile. Tuttavia
esistono agenti che danneggiano permanentemente la proteina.

Il processo di descrive tramite il modello a due stati: nativo e
denaturato. Il presupposto è ammettere che ci siano solo questi due
stati, e che in ogni punto del processo di denaturazione varino
solamente le frazioni di proteina nativa e denaturata. In ogni punto
avremo \(f_{N} + f_{D} = 1\) e il valore dell'osservabile sarà una
combinazione lineare

\[y = y_{N}f_{N} + y_{D}f_{D}\]

\[f_{N} = \frac{y - y_{N}}{y_{D} - y_{N}},\ f_{D} = \frac{y - y_{D}}{y_{N} - y_{D}}\]

\[\frac{f_{N}}{f_{D}} = K_{D} = e^{- \frac{\Delta G_{D}}{\text{RT}}} = \frac{y - y_{N}}{y_{D} - y}\]

Devo però collegarmi alla concentrazione di denaturante, che è indice di
quanto la proteina è denaturata e per farlo uso il LEM (linear
extrapolation method / linear energy model):

\[\Delta G_{D} = \Delta G_{D}^{H_{2}O} - mC_{\text{den}}\]

Il contributo della concentrazione di denaturante è lineare, perché
questo modello è il più semplice possibile e funziona.

Nel mid-point della transizione \(\Delta G_{D} = 0\) quindi
\(K_{D} = 1\) e \(\Delta G_{D}^{H_{2}O} = mC_{\text{mid}}\).

\[\mathbf{y =}\frac{\mathbf{y}_{\mathbf{N}}\mathbf{+}\mathbf{y}_{\mathbf{D}}\mathbf{e}^{\mathbf{-}\frac{\mathbf{\Delta}\mathbf{G}_{\mathbf{D}}}{\mathbf{\text{RT}}}}}{\mathbf{1 +}\mathbf{e}^{\mathbf{-}\frac{\mathbf{\Delta}\mathbf{G}_{\mathbf{D}}}{\mathbf{\text{RT}}}}}\mathbf{=}\frac{\mathbf{y}_{\mathbf{N}}\mathbf{+}\mathbf{y}_{\mathbf{D}}\mathbf{e}^{\mathbf{-}\frac{\mathbf{\Delta}\mathbf{G}_{\mathbf{D}}^{\mathbf{H}_{\mathbf{2}}\mathbf{O}}\mathbf{- m}\mathbf{C}_{\mathbf{\text{den}}}}{\mathbf{\text{RT}}}}}{\mathbf{1 +}\mathbf{e}^{\mathbf{-}\frac{\mathbf{\Delta}\mathbf{G}_{\mathbf{D}}^{\mathbf{H}_{\mathbf{2}}\mathbf{O}}\mathbf{- m}\mathbf{C}_{\mathbf{\text{den}}}}{\mathbf{\text{RT}}}}}\]

\(m\) è la stipness della transizione, cioè la pendenza della curva e
indica la rapidità della transizione.

Denaturazione per effetto dell'aumento di temperatura.

\[\theta = \frac{C_{n}}{C_{n} + C_{d}} = \frac{1}{1 + K\left( T \right)}\]

\[K = e^{- \frac{\Delta G^{0}}{\text{RT}}} = e^{\frac{- \Delta H^{0} + T\Delta S^{0}}{\text{RT}}}\]

I valori sono tabulati quindi posso ragionare solo sulla temperatura.

Il punto di mezzo è la temperatura di melting. Temperatura a
\(\theta = \frac{1}{2}\). La pendenza della transizione è data dai
\(\Delta H^{\circ}\) e \(\Delta S^{\circ}\). Tipico è studiare il
melting del DNA perché si separano le catene.

Nello stato nativo gli aa più idrofobici sono interni alla proteina.
Nello stato denaturato invece vengono esposti all'acqua.

La stabilità di una proteina in acqua è legata alle sue interazioni
preferenziali con il solvente, quindi alla polarità. Un modo semplice di
misurare la polarità è tramite la solubilità (simile scioglie simile).
La solubilità in soluzione può essere misurata dalla differenza di
energia libera fra cristallo puro e soluzione. Ma non è facile fare il
confronto diretto fra cristallo e soluzione. Quindi si può fare una
misura della preferenza in diversi solventi: ad esempio, etanolo e
acqua. Per calcolare la variazione di energia libera ci sono delle
tabelline. Per gli aa singoli le energie libere di solubilità sono tutte
positive, questo perché tutti gli aa sono più idrofilici che idrofobici,
e questo dipende dal legame peptidico più che dalle catene laterali.

Si può costruire una scala di idrofobicità tramite un parametro che è la
somma di tutte le energie libere degli aa moltiplicate per la frequenza
di ciascun aa.

Oppure possiamo separare residui polari e apolari e definire

\[R = \frac{\Sigma\text{res\ carichi}}{\Sigma\text{res\ neutri}}\]

R è una stima di idrofobicità.

\textbf{Funzione delle proteine}

Perché avvengano interazioni dal punto di vista biofisico devono esserci
interazioni fra proteine e macromolecole, che avvengono attraverso
processi di \textbf{legame}, e che ne espletano la funzione.

Il legame è reversibile quindi è un equilibrio chimico, caratterizzato
da una costante di equilibrio.

Esempio. Il trasporto dell'ossigeno nel sangue è un processo di legame,
basato sulla capacità di legare o rilasciare l'ossigeno in condizioni
diverse.

Un altro processo fondamentale di legame è la risposta immunitaria.

La risposta immunitaria è mediata da interazioni tra proteine e si
distingue in:

\begin{itemize}
\item
  Cellulare, tramite cellule T, che distruggono le cellule ospiti
  infettate dai virus
\item
  Umorale, tramite cellule B, che producono anticorpi o immunoglobuline
  per evitare l'infezione.
\end{itemize}

Altro processo che si basa sulle proteine è la contrazione dei muscoli.
Le proteine coinvolte sono actina e miosina che scivolano l'una
sull'altra.

\textbf{ACIDI NUCLEICI}

Gli acidi nucleici si uniscono a formare catene di DNA.

DNA:

\begin{itemize}
\item
  Acido deossiribonucleico
\item
  Scheletro esterno di fosfati di zuccheri
\item
  Eliche (mai perfettamente simmetriche: presentano zone con spazzi più
  grandi e zone con spazi più piccoli; è importante perché altre
  molecole cancerogene o mutageniche si possono infilare in questi spazi
  e disturbare il processo di accoppiamento delle base provocando danni
  nella duplicazione cellulare)
\item
  Basi fondamentali: adenina, citosina, guanina, timina
\item
  Deossiribosio come zucchero fondamentale
\item
  Struttura secondaria: disposizione delle basi ad accoppiamento
  avvenuto. È solo di tipo elica. Ci sono diversi tipi di elica,
  caratterizzate da:

  \begin{itemize}
  \item
    Direzione dell'avvolgimento
  \item
    Angolo fra due coppie di basi successive (definisce la lunghezza del
    gradino)
  \item
    Numero di coppie di basi per giro (definisce la ripidità della
    scala)
  \end{itemize}

  \begin{itemize}
  \item
    Elica B (la più comune): ad avvolgimento destro, 35°, 10bp/giro
  \item
    Elica A (tipica dell'RNA): destrorsa, 33°, 11bp/giro.
  \end{itemize}
\item
  \textbf{Proprietà topologiche}. La maggior parte del DNA
  dell'organismo è contenuto nei nuclei cellulari, e ivi non è in forma
  elicoidale, ma in forma compatta. Si ha spesso la formazione di DNA
  chiusi su se stessi oppure con le estremità bloccate. Dal punto di
  vista fisico-topologico ha interesse la posizione di più filamenti di
  DNA l'uno rispetto all'altro. I due filamenti che lo costituiscono si
  possono avvolgere più volte su se stessi (twisting), e poi possono
  avvolgersi l'uno sull'altro (writhing). Un parametro topologico
  invariante è il \textbf{numero di linking}: numero di avvolgimenti
  attorno a un ipotetica circonferenza.
\end{itemize}

\[Lk = Tw + Wr\]

\begin{quote}
Numero di linking = numero di avvolgimenti (twisting) + numero di
incroci (writhing).
\end{quote}

Ci sono oggetti che possono interagire con gli acidi nucleici: gli
intercalanti, come benzopirene (fumo, bruciato) e bromuro di etidio
(colorante ad eccitazione UV), che sono agenti mutageni e cancerogeni.

Cancerogeno: influenza la replicazione cellulare e la morte programmata
delle cellule. Mutageno: induce variazioni nella sequenza del DNA.

RNA:

\begin{itemize}
\item
  Acido ribonucleico
\item
  Parte doppia elica di tipo A
\item
  Non ha la base timina, ma l'uracile
\item
  Esistono vari tipi di RNA:

  \begin{itemize}
  \item
    mRNA (messaggero): sintesi proteine
  \item
    rRNA (ribosomiale): formazione strutturale ribosomi
  \item
    tRNA (trasporto): trasduce il segnale nei ribosomi, che poi
    attiveranno la sintesi proteica nella cellula.
  \end{itemize}
\end{itemize}

\textbf{Strutture non canoniche del DNA}

Di rado il DNA è perfettamente solo elica; più spesso presenta strutture
non canoniche (cruciform, a zip aperta, etc).

Come fa il DNA a formare queste strutture non convenzionali e ad
arrotolarsi su se stesso? C'è un enzima (topoisomerasi I) che si lega
attorno a un pezzo di DNA e provoca un taglio in uno dei due filamenti
(questo risolve anche problemi di eccessiva tensione). Questo permette
all'elica di svolgersi e ruotare, raggiungere una configurazione ad
energia minore, per poi ricollegare i due filamenti. Così si forma la
struttura ``finale'' del DNA.

\textbf{Elica Z}: unica elica che ha avvolgimento sinistro. All'interno
dei gradini ha ripetizione di coppie di paia di basi invece che di
singole paia di basi.

Esempio. Passaggio da una regione di elica Z a una regione di elica B:
ci sono due basi che vengono espulse dalla struttura del DNA per
accomodare il passaggio da una struttura all'altra.

Le proteine fondamentali che permettono l'avvolgimento del DNA sono gli
\textbf{istoni} (ottameri, proteine con 8 subunità unite a formare
dimeri, tetrameri e infine ottameri). Il DNA interagisce con gli istoni
avvolgendosi a nastro (ribbon) sulla loro superficie esterna. Gli istoni
riescono a formare strutture ordinate di DNA (nucleosomi) che si
impacchettano poi a formare la cromatina.

Esempio di legame stretto fra struttura e funzione: collagene.

\textbf{LEGAME FRA MACROMOLECOLE E LIGANDI}

Il legame fra macromolecole e ligandi (oggetti più piccoli) è un
\textbf{equilibrio} fisico, caratterizzato dalla probabilità di avere lo
stato legato e di avere quello separato, dipendente dalle costanti che
regolano le transizioni, che a loro volta dipendono dalle condizioni al
contorno.

\[M + L \rightleftarrows ML\]

La costante di equilibrio di associazione \(K_{A}\) è il rapporto fra
concentrazione di prodotti e concentrazione di reagenti.

\[K_{A} = \frac{k_{1}}{k_{2}} = \frac{\left\lbrack \text{ML} \right\rbrack}{\left\lbrack M \right\rbrack\lbrack L\rbrack} \rightarrow \frac{\left\lbrack \text{ML} \right\rbrack}{\lbrack M\rbrack} = K_{A}\left\lbrack L \right\rbrack\]

\[K_{D} = \frac{1}{K_{A}}\]

\[\Delta G_{\text{leg}}^{0} = - RT\ln K_{D}\]

A \(K_{D}\) bassa corrisponde legame forte, a \(K_{D}\) alta corrisponde
legame debole.

Il valore di \(\Delta G\) dipende dallo stato scelto come riferimento
(udm scelte).

\textbf{Saturazione} \textbf{frazionaria}

Il parametro utile a descrivere l'interazione (legame) fra macromolecola
(proteina) e ligando è la saturazione frazionaria: frazione di
macromolecole che hanno un ligando legato rispetto alle possibilità
totali di legame (macromolecole presenti in soluzione).

\[Y = \theta = \frac{\left\lbrack \text{ML} \right\rbrack}{\lbrack M_{T}\rbrack} = \frac{\left\lbrack \text{ML} \right\rbrack}{\left\lbrack M \right\rbrack + \lbrack ML\rbrack} = \frac{K_{A}\left\lbrack L \right\rbrack}{1 + K_{A}\lbrack L\rbrack} = \frac{\left\lbrack L \right\rbrack}{K_{D} + \left\lbrack L \right\rbrack}\]

La SF dipende dalla concentrazione di ligando libero in soluzione e
dalla costante di dissociazione.

Quando la saturazione frazionaria è 1/2:

\[\frac{1}{2} = \frac{\left\lbrack L \right\rbrack_{\frac{1}{2}}}{K_{D} + \left\lbrack L \right\rbrack_{\frac{1}{2}}}\]

Quindi dev'essere
\(K_{D} = \left\lbrack L \right\rbrack_{\frac{1}{2}}\): la costante di
dissociazione corrisponde alla concentrazione di ligando necessaria per
saturare metà dei legami.

\includegraphics[width=2.50625in,height=2.53125in]{media/image14.emf}Studiando
il grafico \(\left\lbrack L \right\rbrack - Y\) possiamo valutare la
\(K_{D}\). A bassi valori di \(\lbrack L\rbrack\) rispetto alla
\(K_{D}\) la relazione è lineare
\(Y = \left\lbrack L \right\rbrack/K_{D}\).

La curva di legame completa satura poi molto lentamente.

Nel caso del legame semplice per avere 90\% di saturazione la
concentrazione di ligando deve essere 9 volte la \(K_{D}\); per avere
saturazione del 99\% dev'essere
\(\left\lbrack L \right\rbrack = 99K_{D}\).

Se mettiamo la concentrazione su scala logaritmica possiamo vedere
l'intervallo grande (4 decadi) di concentrazione di ligando libero che
dobbiamo esplorare per caratterizzare completamente la curva di legame.

Dal punto di vista sperimentale, variare molto la concentrazione di
ligando libero è complicato, quindi si fa il contrario: si tiene
costante il ligando e si varia la concentrazione di macromolecola.
Perché conviene?

Gli altri due metodi sono le rappresentazioni

\begin{itemize}
\item
  di Scatchard: informazioni sul numero di siti di legame e sulle
  costanti di legame
\item
  di Hill: informazioni sull'accoppiamento.
\end{itemize}

\(\mathbf{n}\) \textbf{siti identici e indipendenti}

Le cose non sono mai così semplici: non avremo mai un solo sito sulla
molecola. Nel caso più semplice avremo tanti siti identici e
indipendenti (non interagiscono fra loro), ciascuno caratterizzato da
una costante di interazione \(k\).

\[M_{0} + L \rightarrow M_{1}\]

\[M_{1} + L \rightarrow M_{2}\]

\[M_{n - 1} + L \rightarrow M_{n}\]

La macromolecola libera può legare successivamente diversi oggetti, fino
a saturare tutti i siti di legame. \(M_{i}\) sono l'insieme delle specie
microscopiche che hanno \(i\) ligandi legati.

Esempio. Se abbiamo \(n = 4\) siti di legame, \(M_{2}\) sono le specie
microscopiche che hanno 2 oggetti legati, ma essendoci 4 siti
equivalenti, questi oggetti possono essere legati in 6 combinazioni
diverse.

\begin{longtable}[c]{@{}llllllllllll@{}}
\toprule
L & & & & & L & L & L & L & & & L\tabularnewline
\midrule
\endhead
L & & L & L & & L & & & & L & L &\tabularnewline
\bottomrule
\end{longtable}

Il numero di oggetti dipende dal numero di siti \(n\) e dal numero di
occupazione \(i\) tramite il coefficiente binomiale:

\[\Omega_{n,i} = \frac{n!}{\left( n - i \right)!i!}\]

Un parametro equivalente alla saturazione frazionaria ma che utilizza le
concentrazioni è \(v\), che ci indica le moli di ligando legati per moli
di macromolecola. Le moli di macromolecola sono le somme delle
concentrazioni di tutte le specie in soluzione. Le moli di ligando
legato sono il prodotto di ciascuna specie macroscopica con il numero di
oggetti che ha legato.

\[v = \frac{\sum_{i = 0}^{n}{\left\lbrack M_{i} \right\rbrack i}}{\sum_{i = 0}^{n}\left\lbrack M_{i} \right\rbrack}\]

Tuttavia, sperimentalmente ciò che misuro è quanto ligando o quanta
macromolecola metto, quindi devo trovare una relazione che utilizzi la
concentrazione di ligando e di macromolecola.

Sia \(K_{i}\) la costante di associazione macroscopica relativa al
legame \(i\).

\[K_{1} = \frac{\left\lbrack M_{0} \right\rbrack\left\lbrack L \right\rbrack}{\lbrack M_{1}\rbrack}\ \]

\[K_{n} = \frac{\left\lbrack M_{n - 1} \right\rbrack\left\lbrack L \right\rbrack}{\lbrack M_{n}\rbrack}\ \]

Ma ho ancora una dipendenza da \(i\), quindi ripeto la stessa operazione
fino a:

\[\left\lbrack M_{i} \right\rbrack = \frac{\left\lbrack M_{i - 1} \right\rbrack\left\lbrack L \right\rbrack}{\left\lbrack M_{n} \right\rbrack} = \frac{\left\lbrack M_{0} \right\rbrack\left\lbrack L \right\rbrack^{i}}{\prod_{j = 1}^{i}K_{j}} = \left\lbrack M_{i - 1} \right\rbrack\frac{n - i + 1}{i}\frac{\left\lbrack L \right\rbrack}{k} = \left\lbrack M_{0} \right\rbrack\left( \prod_{j = 1}^{i}\frac{n - j + 1}{j} \right)\left( \frac{\left\lbrack L \right\rbrack}{k} \right)^{i}\ \]

Il problema è che la \(k\) è macroscopica, non tiene conto di quale
configurazione microscopica (nell'esempio, delle 6 possibili) sia quella
effettiva. Introduciamo \(k\), la costante di associazione microscopica,
supponendo che la \(k\) sia la stessa quando aggiungo il primo ligando o
quando aggiungo un qualsiasi altro ligando (il terzo dopo che ne sono
stati aggiunti due, ad esempio).

La \(k\) microscopica si riferisce al legame di un singolo ligando a
livello molecolare, indipendentemente dal tipo di configurazione, mentre
la \(K\) macroscopica la normalizza rispetto a tutte le configurazioni
possibili.

Il legame fra \(k\) e \(K\) è dato dalla numerosità delle configurazioni
microscopiche che abbiamo, nel passaggio da \(i - 1\) ligandi legati a
\(i\) ligandi legati.

\[\mathbf{v} = \frac{\sum_{i}^{n}{\left( \prod_{j = 1}^{i}\frac{n - j + 1}{j} \right)\left( \frac{\left\lbrack L \right\rbrack}{k} \right)^{i}i}}{1 + \sum_{i}^{n}{\left( \prod_{j = 1}^{i}\frac{n - j + 1}{j} \right)\left( \frac{\left\lbrack L \right\rbrack}{k} \right)^{i}}} = \frac{\frac{\mathbf{n}\left\lbrack \mathbf{L} \right\rbrack}{\mathbf{k}}}{\mathbf{1 +}\frac{\left\lbrack \mathbf{L} \right\rbrack}{\mathbf{k}}}\]

Relazione che deriva da:

\[\left( \prod_{j = 1}^{i}\frac{n - j + 1}{j} \right) = \Omega_{n,i} = \frac{n!}{\left( n - i \right)!i!}\]

\[1 + \sum_{i}^{n}{\left( \prod_{j = 1}^{i}\frac{n - j + 1}{j} \right)\left( \frac{\left\lbrack L \right\rbrack}{k} \right)^{i}} = \left( 1 + \frac{\left\lbrack L \right\rbrack}{k} \right)^{n}\]

Derivando:

\[n\frac{\left\lbrack L \right\rbrack}{k}\left( 1 + \frac{\left\lbrack L \right\rbrack}{k} \right)^{n - 1} = \sum_{i}^{n}{i\frac{n!}{\left( n - i \right)!i!}\left( \frac{\left\lbrack L \right\rbrack}{k} \right)^{i}}\]

La saturazione frazionaria la possiamo quindi esprimere in funzione
della costante microscopica e del numero di siti di legame, ed è
rappresentabile con lo Scatchard Plot.

Esso rappresenta la frazione di ligandi legati, normalizzata per la
concentrazione di ligando libero e ciò è una retta con pendenza
\(- \frac{1}{k}\) e intercetta \(\frac{n}{k}\) su y e \(n\) su x.

Osservazioni. Ho un parametro in funzione di se stesso, cosa strana.
Inoltre, richiede una certa elaborazione dei dati perché richiede la
conoscenza della concentrazione di ligando libero.

\textbf{Classi multiple di} \(\mathbf{n}_{\mathbf{i}}\) \textbf{siti
identici e indipendenti}

Possiamo scrivere le stesse equazioni, ma otteniamo uno Scatchard Plot
che non è più una retta.

\(\mathbf{n}\) \textbf{siti identici e interagenti}

Se interagiscono la costante di associazione è funzione del grado di
saturazione, ovvero il legame di un oggetto facilita o inibisce il
legame dei successivi.

\[k = k(v)\]

\[\Delta G^{0} = \Delta G_{0}^{0} + RT\ \phi(v)\]

\(\phi(v)\) è una funzione che per semplicità assumiamo lineare, con
l'unica condizione che

\[\phi\left( v \right) = 0\ \text{se}\ v = 0\]

\[k\left( v \right) = k_{0}e^{- \phi\left( v \right)}\]

Nello Scatchard Plot la \(k\) non sarà più costante. Se la \(\phi\) è
decrescente, il suo esponenziale sarà crescente, quindi la costante di
dissociazione \(k\) cresce, e ciò implica che il legame è sfavorito e
nel plot avrò una curvatura verso l'alto.

Il fenomeno contrario è la cooperatività: processo per cui il legame di
un oggetto facilita il legame dei successivi.

\textbf{MODELLO DI HILL}

Descrive la cooperatività.

Parte da un'assunzione: il legame può essere descritto con delle
funzioni simili a quelle ottenute in un sistema all-or-none, ovvero un
sistema con cooperatività infinita (una volta che uno si lega, pure
tutti gli altri possibili si legano, si hanno 0 oppure n oggetti
legati).

\[M_{0} + nL \leftrightarrow M_{n}\]

Definiamo la saturazione frazionaria come

\[\overset{\overline{}}{y} = \frac{v}{n} = \frac{\frac{\left\lbrack L \right\rbrack^{n}}{K^{n}}}{1 + \frac{\left\lbrack L \right\rbrack^{n}}{K^{n}}}\]

In un buon range del legame posso esprimere la saturazione frazionaria
in funzione di un altro coefficiente con la stessa espressione, che
porta, al posto di \(n\), il coefficiente di Hill \(\alpha_{H}\), che
può assumere un valore variabile da 1 a \(n\) a seconda del grado di
cooperatività del sistema. Per \(\alpha_{H} = n\) ho cooperatività
massima.

\[\overset{\overline{}}{\mathbf{y}}\mathbf{=}\frac{\frac{\left\lbrack \mathbf{L} \right\rbrack^{\mathbf{\alpha}_{\mathbf{H}}}}{\mathbf{K}^{\mathbf{\alpha}_{\mathbf{H}}}}}{\mathbf{1 +}\frac{\left\lbrack \mathbf{L} \right\rbrack^{\mathbf{\alpha}_{\mathbf{H}}}}{\mathbf{K}^{\mathbf{\alpha}_{\mathbf{H}}}}}\]

\includegraphics[width=2.13542in,height=2.16806in]{media/image15.emf}Il
grafico (Hill Plot) ha andamento sigmoidale, la cui pendenza rappresenta
il grado di cooperatività. All'aumentare di \(\alpha_{H}\) aumenta la
pendenza della curva.

La larghezza del range di transizione è anche un indice di questo: se
abbiamo cooperatività infinita il range sarà praticamente nullo.

Una volta che ho l'espressione analitica della mia curva, posso ricavare
il parametro di Hill tramite:

\[\frac{\overset{\overline{}}{y}}{1 - \overset{\overline{}}{y}} = \frac{\left\lbrack L \right\rbrack^{\alpha_{H}}}{K^{\alpha_{H}}} \rightarrow \alpha_{H} = \frac{d\left( \ln\left( \frac{\overset{\overline{}}{y}}{1 - \overset{\overline{}}{y}} \right) \right)}{d\left( \ln\left\lbrack L \right\rbrack \right)}\]

Tuttavia l'equazione evidenziata non vale per tutti i valori di \(y\)
(in tutto il range), poiché \(\alpha_{H}\) è funzione del grado di
saturazione, quindi si definisce il parametro di Hill corrispondente
alla concentrazione che si ha quando c'è saturazione pari a 0,5. Così la
\(\alpha_{H}\) la possiamo estrapolare dal Plot di Hill tramite la
pendenza del grafico.

\[\frac{\alpha_{H,\frac{1}{2}}}{\left\lbrack L \right\rbrack_{\frac{1}{2}}} = \frac{d\left( \frac{\overset{\overline{}}{y}}{1 - \overset{\overline{}}{y}} \right)}{d\left( \left\lbrack L \right\rbrack \right)}_{\overset{\overline{}}{y} = \frac{1}{2}}\]

Possiamo invece estrapolare la \(K\) di legame tramite l'intercetta del
grafico \emph{con l'ascissa. }

A concentrazione di ligando bassa abbiamo un sistema non cooperativo,
quindi possiamo estrapolare la \(K_{1}\) di legame del primo oggetto
legato; anche quando abbiamo \(\left\lbrack L \right\rbrack\) alta
abbiamo non-cooperatività, perché la concentrazione è spostata verso la
specie con tutti gli oggetti legati, quindi possiamo estrapolare la
\(K_{n}\) di legame con tutti i siti occupati.

\includegraphics[width=2.23194in,height=1.97917in]{media/image16.emf}In
questo caso possiamo osservare anche lo Scatchard:

\[\frac{v}{\left\lbrack L \right\rbrack} = \frac{\frac{n\left\lbrack L \right\rbrack^{\alpha_{H} - 1}}{K^{\alpha_{H}}}}{1 + \frac{\left\lbrack L \right\rbrack^{\alpha_{H}}}{K^{\alpha_{H}}}}\ \]

Ha andamento a cupola con un massimo, legato alla \(K\), va a 0 quando
abbiamo saturazione nulla e massima.

Quindi comunque posso usare anche questo.

\[K^{\alpha_{H}} = \frac{\left\lbrack L \right\rbrack^{\alpha_{H}}\left( n\left\lbrack M \right\rbrack_{0} - \left\lbrack L \right\rbrack_{b} \right)}{\left\lbrack L \right\rbrack_{b}}\]

\(\left\lbrack L \right\rbrack_{b}\) concentrazione di ligando legato.

Non posso esprimerlo in funzione di concentrazione di proteina e
macromolecola, perché potrei averne molte legate ad una stessa proteina
e altre che non legano nulla.

Insieme a queste, possiamo anche plottare il \(\ln{\lbrack L\rbrack}\) e
\(\ln\left\lbrack \frac{n}{v - 1} \right\rbrack\) che avrà pendenza
\(- \frac{1}{\alpha_{H}}\) e intercetta \(\ln K\).

Se esiste cooperatività significa che esiste cooperazione fra i siti, e
questo altera l'energia di legame. Supponiamo di avere una proteina con
due siti di legame, e due ligandi a disposizione. Ciascun oggetto ha una
sua energia libera di legame.

\begin{longtable}[c]{@{}ll@{}}
\toprule
\(M + L_{1} \rightarrow ML_{1}\) & \(\Delta G_{1}^{0}\)\tabularnewline
\midrule
\endhead
\(M + L_{2} \rightarrow L_{2}M\) & \(\Delta G_{2}^{0}\)\tabularnewline
\(L_{2}M + L_{1} \rightarrow L_{2}ML_{1}\) &
\(\Delta G_{1}^{0}\left( 2 \right)\)\tabularnewline
\(ML_{1} + L_{2} \rightarrow L_{2}ML_{1}\) &
\(\Delta G_{2}^{0}\left( 1 \right)\)\tabularnewline
\(M + L_{1} + L_{2} \rightarrow L_{2}ML_{1}\) &
\(\Delta G^{0}\left( 1,2 \right)\)\tabularnewline
\bottomrule
\end{longtable}

Ovviamente dovremo avere
\(\Delta G_{1}^{0} + \Delta G_{2}^{0}\left( 1 \right) = \Delta G_{2}^{0} + \Delta G_{1}^{0}\left( 2 \right) = \Delta G^{0}\left( 1,2 \right)\),
perché in entrambi i casi arrivo allo stesso risultato: la molecola ha
due oggetti legati.

Se c'è interazione, che sia cooperativa o non cooperativa, c'è una
differenza fra energia libera con entrambi gli oggetti legati e la somma
delle energie libere di legame dei due oggetti separati, quindi

\[\Delta G_{12}^{0} = \Delta G^{0}\left( 1,2 \right) - \Delta G_{1}^{0} - \Delta G_{2}^{0}\]

Se \(\Delta G_{12}^{0} = 0\) si ha che
\(\Delta G^{0}\left( 1,2 \right) = \Delta G_{1}^{0} + \Delta G_{2}^{0}\)
e ciò significa che non importa l'ordine con cui si legano i due
oggetti, posso prima legare uno e poi l'altro o viceversa: non c'è
interazione.

Se invece \(\Delta G_{12}^{0} < 0\) il sistema è favorito, quindi c'è
cooperatività.

Se \(\Delta G_{12}^{0} > 0\) il legame è sfavorito, quindi c'è
interazione antagonistica.

Se consideriamo i vari equilibri singolarmente, per ciascuno di essi
possiamo scrivere l'equazione della saturazione frazionaria:

\begin{longtable}[c]{@{}ll@{}}
\toprule
\(M + L_{1} \rightarrow ML_{1}\) &
\(\overset{\overline{}}{y_{1}} = \frac{\left\lbrack L_{2}ML_{1} \right\rbrack + \left\lbrack ML_{1} \right\rbrack}{\left\lbrack M \right\rbrack_{T}}\)\tabularnewline
\midrule
\endhead
\(M + L_{2} \rightarrow L_{2}M\) &
\(\overset{\overline{}}{y_{2}} = \frac{\left\lbrack L_{2}ML_{1} \right\rbrack + \left\lbrack L_{2}M \right\rbrack}{\left\lbrack M \right\rbrack_{T}}\)\tabularnewline
\(M + L_{1} + L_{2} \rightarrow L_{2}ML_{1}\) &
\(\overset{\overline{}}{y_{12}} = \frac{\lbrack L_{2}ML_{1}\rbrack}{\left\lbrack M \right\rbrack_{T}}\)\tabularnewline
\bottomrule
\end{longtable}

Consideriamo la reazione, invece che come sequenziale, come di
sproporzionamento, esaminando le due speci (quella con legato \(L_{1}\)
e quella con legato \(L_{2}\)) e valutiamo l'equilibrio di queste speci
confrontandolo con quello delle speci saturate e senza oggetti legati.

La trasformazione si ottiene facendo
\(ML_{1} + L_{2}M \rightarrow M + L_{2}ML_{1}\) e la sua energia libera
è esattamente l'energia libera di accoppiamento \(\Delta G_{12}^{0}\).

\[K_{12} = e^{- \frac{\Delta G_{12}^{0}}{\text{RT}}} = \frac{\left\lbrack M \right\rbrack\lbrack L_{1}ML_{1}\rbrack}{\left\lbrack ML_{1} \right\rbrack\left\lbrack L_{2}M \right\rbrack}\]

\includegraphics[width=3.27153in,height=2.76042in]{media/image17.emf}Se
\(\Delta G_{12}^{0} = 0\) abbiamo un sistema all'equilibrio.

Se \(\Delta G_{12}^{0} < 0\) significa che il sistema preferisce
saturare entrambi i siti.

Quando \(y_{1} = y_{2} = \frac{1}{2}\) si ha
\(K_{12} = \frac{{\overset{\overline{}}{y_{12}}}^{2}}{\left( \frac{1}{2} - \overset{\overline{}}{y_{12}} \right)^{2}}\)
e
\(\Delta G_{12}^{0} = - 2RT\ln\frac{2\overset{\overline{}}{y_{12}}}{1 - 2\overset{\overline{}}{y_{12}}}\),
funzione facile da plottare.

Se \(2\overset{\overline{}}{y_{12}} = 1\) abbiamo l'unica specie con
entrambi legati. Se \(2\overset{\overline{}}{y_{12}} = 0\) abbiamo le
due specie intermedie, ciascuna con un oggetto legato. Se
\(2\overset{\overline{}}{y_{12}} = \frac{1}{2}\) abbiamo
\(\Delta G_{12}^{0} = 0\) e non c'è accoppiamento, non c'è nulla che
favorisca una condizione rispetto alle altre.

Oss. Tali energie sono molto basse! Basta poca energia per spostare
l'equilibrio verso il doppio legame.

Tutto ciò è alla base del funzionamento delle proteine allosteriche.
L'effetto allosterico (quando le caratteristiche di legame vengono
alterate dal legame stesso) può essere:

\begin{itemize}
\item
  omotropico: il legame di un oggetto cambia l'affinità del legame di
  altri oggetti dello stesso tipo;
\item
  eterotropico: il legame cambia l'affinità per altri tipi di legame.
\end{itemize}

Gli enzimi funzionano così: c'è un sito regolatore (dove si lega
l'oggetto che dà l'effetto allosterico) e un sito catalitico (dove si
lega il ligando primario che dà la sua funzione alla proteina).
Tipicamente si lega prima un oggetto nel sito regolatore, esso cambia la
conformazione della proteina rendendola più favorevole al legame col
ligando primario e questo cambio di affinità caratterizza l'azione
catalitica.

Se abbiamo più siti di legame allosterico può diventare catalitico.

Gli effetti allosterici richiedono l'esistenza di due forme della
macromolecola:

\begin{itemize}
\item
  T taut (tesa)
\item
  R relaxed (rilassata)
\end{itemize}

Il cambiamento da T a R può avvenire:

\begin{itemize}
\item
  All'unisono: modello di Monod-Wyman-Changeaux
\item
  Sequenzialmente: modello di Koshland
\end{itemize}

\textbf{Emoglobina}

Il passaggio dalla forma ossi- alla forma deossi- è una transizione
conformazionale RT.

L'emoglobina ha un anello eterociclico che trasporta il Fe, il cui stato
di ossidazione è cruciale per il legame con l'O.

Tutte le proteine in grado di trasportare l'ossigeno possiedono un
gruppo eme. La differenza è il tipo di legame che fanno con l'O le varie
proteine. Ci sono infatti più siti di legame, che possono essere
indipendenti o interagire.

Inotre, l'emoglobina ha comportamento di tipo cooperativo (il legame di
un oggetto favorisce il legame di altri) mentre la mioglobina no.
Infatti il legame dell'emoglobina con il Fe comporta un cambiamento
conformazionale, cioè uno spostamento di un residuo (istidina); in
questo modo si facilita l'accesso dell'ossigeno al ferro. Il legame con
il Fe provoca una modifica nella conformazione inducendo il movimento di
una subunità β.

L'emoglobina è tetramerica, formata da 4 catene quindi 4 siti di legame
per l'ossigeno.

Il valore della costante di Hill varia fra 1 e il numero di siti di
legame ed è un indice della cooperatività.

Il legame di tipo allosterico omotropico (modifica conformazione, e sono
i legami sono tutti uguali). Gli enzimi invece hanno meccanismi
eterotropici.

Posso osservare il legame perché ci sono grandezze fisiche osservabili
che cambiano.

Richiede due conformazioni, che sono sensibili a condizioni esterne:

\begin{itemize}
\item
  T tense, bassa \(pO_{2}\), ossiemoglobina, minore affinità quindi l'O
  verrà rilasciato.
\item
  R relaxed, maggiore affinità per l'O quindi in questa conformazione
  avrà l'O legato.
\end{itemize}

L'alternanza delle due conformazioni è cruciale per una corretta
distribuzione dell'O.

La transizione da uno stato all'altro è regolata dalla pressione
parziale dell'O.

Anche la concentrazione di \(CO_{2}\) influisce sul cambiamento
conformazionale.

Con il legame cambia anche lo spettro di assorbimento.

Problema: metaemoglobina, non dovremmo averla, nel gruppo eme il ferro
di trova nello stato di ossidazione \(Fe^{3 +}\) e non lega l'\(O_{2}\)
(conseguenza: favismo).

Ciò che si studia è la curva di rilascio (saturazione frazionaria --
frazione di siti occupati - in funzione di \(pO_{2}\)): sigmoide con
pendenza elevata, plateau ad alta pressione parziale \(O_{2}\).

La zona ad alto legame è quella nelle arterie, perché il sangue
arterioso è quello ricco di ossigeno, quella a basso legame è quella
delle vene, il sangue venoso è meno ricco di ossigeno. Il cambiamento di
colore è dovuto al cambiamento dello spettro di assorbimento.

Rispetto alla mioglobina, quella dell'emoglobina è più sigmoidale,
quindi caratterizzabile con un valore di costante di Hill, anche se qui
non abbiamo meccanismo del tipo all-or-none. La costante di Hill non è
in realtà costante, ma ha un andamento a campana con un massimo a 2.5.

Ci sono fattori che spostano la curva, che rendono quindi il rilascio
più efficiente a pressioni diverse. Ad esempio, i muscoli stanno a
pressioni parziali da 20 a 40 mmHg.

Questi effettori allosterici si distinguono in:

\begin{itemize}
\item
  Positivi: spostano la curva verso sinistra
\item
  Negativi: spostano la curva verso destra
\end{itemize}

Esempi. Produzione di \(CO_{2}\) diminuisce il pH e favorisce il
rilascio dell'O. La DPG (proteina che ha legame preferenziale con la
forma T) sposta l'equilibrio verso la forma T (a sinistra) -- i muscoli
si abituano a lavorare in presenza di meno O (allenamento ad alta
quota).

Quindi in scarsa presenza di ossigeno (bassa concentrazione) tendiamo a
far spostare la curva a sinistra, perché questo favorisce il legame
anche a basse concentrazioni.

L'emoglobina fetale non ha la stessa affinità con l'O rispetto a quella
della mamma. A parità di pressione deve mostrare maggiore affinità
perché si abbia il trasferimento di O, altrimenti l'O rimarrebbe nel
sangue della mamma.

La curva della mioglobina è più in alto perché è molto efficiente per il
legame con l'O ma pessima per il rilascio nei tessuti.

4 siti identici e indipendenti effetto statistico sulle K macroscopiche
di anticooperatività.

Siti non indipendenti effetto di cooperatività (considero le K
microscopiche).

Apparente \(\Delta G_{i}^{0}\) di legame: la solita
\(\Delta G_{i}^{0} = RT\ln K_{i}\)

Intrinseca \(\Delta{\overset{\overline{}}{G}}_{i}^{0}\) di legame

\[\Delta{\overset{\overline{}}{G}}_{i}^{0} = RT\ln K_{i} - RT\ln\frac{\Omega_{n,i - 1}}{\Omega_{n,i}}\]

Se \(\Delta G_{\text{ij}} < 0,\ j > i\) cooperatività, se
\(\Delta G = 0\) indipendenza.

\textbf{SPETTROSCOPIA DI ASSORBIMENTO UV -- VIS}

Studia l'interazione radiazione-materia (molecola).

150-700nm

Spettro: misura dell'intensità del parametro interessante in funzione di
frequenza o lunghezza d'onda.

\begin{itemize}
\item
  Spettroscopia di assorbimento. L'assorbimento è un processo
  praticamente istantaneo.
\item
  Spettroscopia di emissione. Si basa sulla diseccitazione radiativa
  delle molecole e i livelli energetici eccitati vengono occupati per
  più tempo, quindi è più interessante.
\end{itemize}

\textbf{Interazione radiazione materia}

La radiazione elettromagnetica interagisce con le cariche di qualsiasi
corpo interessato. Le energie in gioco sono molto diverse, quindi si
separano spettroscopia ottica e spettroscopia magnetica. L'effetto del
CE sulle cariche dell'oggetto è garantito dall'interazione con il
momento di dipolo.

Massa nucleare ed elettronica sono molto diverse quindi risposte molto
diverse al CE. L'approssimazione di Born-Oppenheimer è alla base della
separazione moti nucleari -- moti elettronici per la divisione fra
spettroscopie elettroniche e spettroscopie vibrazionali (degli atomi
nella nuvola elettronica).

{[}L'approssimazione di Born-Oppenheimer è una tecnica usata al fine di
disaccoppiare i moti dei nuclei e degli elettroni, ovvero per separare
le variabili corrispondenti al moto nucleare e le coordinate
elettroniche nella equazione di Schrödinger associata alla hamiltoniana
molecolare. Si basa sul fatto che le tipiche velocità elettroniche sono
molto maggiori di quelle nucleari.{]}

Nel modello classico c'è trasferimento di energia all'oggetto e tale
energia, se non ci sono risonanze, dà scattering, altrimenti viene
assorbita (se ci sono risonanze). Poi ci sono processi dissipativi di
vario tipo.

Nel modello quantistico l'energia varia in quanti \(\text{hν}\) e il
sistema è descritto dall'equazione di Schrodinger:

\[i\mathcal{h}\frac{d}{\text{dt}}\Psi = \overset{\overline{}}{H}\Psi\]

\(\Psi\) è la funzione d'onda, \(H\) l'hamiltoniana del sistema.

\(P = \Psi^{2}\) è la probabilità di trovare l'elettrone (e si parla di
elettroni perché trascuriamo il moto nucleare). La probabilità è
normalizzata.

L'hamiltoniana ha come valore di aspettazione l'energia, che è la somma
delle energie cinetiche degli elettrone e di un termine di potenziale
che indica le interazioni con il circondario.

\[\left\langle \Psi\left| \overset{\overline{}}{H} \right|\Psi \right\rangle = E\ \]

\[\overset{\overline{}}{H} = \overset{\overline{}}{T} + \overset{\overline{}}{V}\]

Se l'hamiltoniana è indipendente dal tempo allora la funzione d'onda ha
la forma:

\[\overset{\overline{}}{H}\Psi = E\Psi\]

E questi sono stati stazionari perché la probabilità è costante nel
tempo.

Gli autostati sono ortogonali:

\[\left\langle \Psi_{1} \middle| \Psi_{2} \right\rangle = \left\langle \Psi_{2} \middle| \Psi_{1} \right\rangle = 0\]

In un sistema a due stati la funzione d'onda è la sovrapposizione delle
funzioni d'onda corrispondenti ai due stati. Se le autofunzioni sono
ortogonali, l'integrale di sovrapposizione è nullo e la probabilità di
trovare una molecola nello stato a (o b) è il modulo quadro dei
coefficienti che compaiono nella funzione d'onda.

\[\left\langle \Psi_{a} \middle| \Psi_{b} \right\rangle = 0\]

\[\Psi = C_{a}\Psi_{a} + C_{b}\Psi_{b}\]

\[P_{a} = \left| C_{a} \right|^{2},\ P_{b} = \left| C_{b} \right|^{2}\]

La soluzione esatta sfrutta l'approssimazione di Born-Oppenheimer,
quindi la funzione d'onda avrà una parte elettronica e una nucleare:

\[\Psi = \Psi_{e}\left( r,R \right)\Phi_{N}\left( R \right)\]

I nuclei (R) si muovono all'interno di un potenziale determinato dalle
posizioni medie degli elettroni mentre gli elettroni (r) vedono i nuclei
fermi.

\[{\overset{\overline{}}{H}}_{\text{el}}\Psi_{\text{El}} = E_{\text{el}}\Psi_{\text{el}}\]

Dalla soluzione si ottengono i livelli elettronici del sistema
imperturbato.

Quando invece abbiamo un CM esterno che perturba il sistema abbiamo
l'hamiltoniana modificata da un termine che esprime l'interazione col
CM.

Si chiama cromoforo l'oggetto che produce colore, cioè che assorbe. I
cromofori hanno dimensioni più piccole della λ della radiazione
utilizzata. La variazione spaziale del CE può essere trascurata. Quindi
il CE si descrive indipendentemente dallo spazio.

L'espressione di V determina il tipo di interazione e l'elemento che
determina l'intensità dell'interazione è

\[\left| \left\langle \Psi\left| \overset{\overline{}}{V} \right|\Psi \right\rangle \right|^{2} = \left| \left\langle \overset{\overline{}}{V} \right\rangle \right|^{2} = \left| \int_{}^{}{\Psi_{\text{fin}}^{*}\overset{\overline{}}{V}\Psi_{\text{in}}\text{dτ\ }} \right|^{2}\]

\textbf{Approssimazione di dipolo}

Si fa l'espansione in serie della distribuzione molecolare di carica
elettrica.

Il termine che regola l'accoppiamento fra onda em e sistema molecolare è

\[\overset{\overline{}}{V} = \overset{\overline{}}{\mu} \cdot E\left( t \right) = \overset{\overline{}}{\mu} \cdot E\left( 0 \right)e^{\text{iωt}}\]

\[\overset{\overline{}}{\mu} = \sum_{i}^{}{e_{i}\mathbf{r}_{i}}\]

Perché vi sia trasferimento di energia il momento di dipolo di
transizione (valore di aspettazione) dev'essere:

\[\left| \left\langle \Psi_{\text{fin}}\left| \overset{\overline{}}{V} \right|\Psi_{\text{in}} \right\rangle \right|^{2} = \left| \left\langle \overset{\overline{}}{V} \right\rangle \right|^{2} = \left| E\left( 0 \right) \cdot \int_{}^{}\Psi_{\text{fin}}^{*}\overset{\overline{}}{\mu}\Psi_{\text{in}}\text{dτ} \right|^{2} = \left| \left\langle \Psi_{\text{fin}}\left| \overset{\overline{}}{\mu} \right|\Psi_{\text{in}} \right\rangle E\left( 0 \right) \right|^{2}\]

Soltanto per stati iniziali e finali per cui avrò \(\neq 0\) si può
avere transizione (perché devo cambiare stato elettronico del sistema).
Le regole di selezione selezionano quali transizioni fra quali stati
assicurano
\(\left\langle \Psi_{\text{fin}}\left| \overset{\overline{}}{\mu} \right|\Psi_{\text{in}} \right\rangle \neq 0\).
Tali regole vengono da simmetrie degli stati coinvolti (dopotutto era un
integrale spaziale).

Di tutta la molecola, quello che interagisce con campo EM è l'elettrone,
quindi considero solo i due orbitali interessati dalle transizioni (che
possiedono i termini non costanti). L'esistenza della transizione può
essere dedotta da considerazioni di simmetria. Conviene considerare il
momento di dipolo separato sui tre assi. Se l'integrando è dispari per
riflessione su uno dei tre piani, tutto l'integrale è nullo.

Perché quel termine non deve essere nullo?

Consideriamo un sistema a due stati a e b:

\[\Psi\left( t \right) = C_{a}\left( t \right)\Psi_{a}e^{- \frac{iE_{a}t}{h}} + C_{b}\left( t \right)\Psi_{b}e^{- \frac{iE_{b}t}{h}}\]

La funzione d'onda è indipendente dal tempo.

\[i\mathcal{h}\frac{d}{\text{dt}}\Psi = \left( \overset{\overline{}}{H} + \overset{\overline{}}{V} \right)\Psi\]

Se inseriamo l'espressione di \(\Psi\), abbiamo il termine nelle
derivate delle \(C\) e il termine di derivata dell'esponenziale, che
però risolviamo sapendo che \(\overset{\overline{}}{H}\Psi = E\Psi\). I
termini si cancellano e tutto diventa più semplice. Rimangono i termini
delle derivate delle \(C\) e della funzione d'onda per il potenziale.
L'oggetto che rimane però è mescolato, contiene i coefficienti di
entrambi gli stati, ma a me ne interessa solo uno dei due perché
inizialmente il sistema è nello stato a, e mi interessa sapere cosa fa
nello stato b, sotto sollecitazione del campo EM.

Per separarli si moltiplica per il complesso coniugato di uno dei due
termini, ottenendo due funzioni pulite, simmetriche che contengono
entrambe un termine del tipo
\(\left\langle \Psi_{i}\left| \overset{\overline{}}{V} \right|\Psi_{i} \right\rangle\).
Ricordando che \(\mu\) è dispari, \(\left| \Psi \right|^{2}\) è pari,
abbiamo che
\(\left\langle \Psi_{i}\left| \overset{\overline{}}{V} \right|\Psi_{i} \right\rangle = 0\)
quindi quei termini spariscono e rimane solo il termine del tipo

\[C_{a}\left( t \right)\left\langle \Psi_{b}\left| \overset{\overline{}}{\mu} \right|\Psi_{a} \right\rangle E\left( 0 \right)e^{- it\left( \frac{E_{b}}{\mathcal{h}} - \frac{E_{a}}{\mathcal{h}} - \omega \right)}\]

Ora imponiamo due approssimazioni:

\[\left\langle \Psi_{b}\left| \overset{\overline{}}{V} \right|\Psi_{a} \right\rangle = \left\langle \Psi_{a}\left| \overset{\overline{}}{V} \right|\Psi_{b} \right\rangle\]

\[\left| C_{b}\left( 0 \right) \right|^{2} = 0\]

Cioè lo stato fondamentale è lo stato a, a tempo 0 la probabilità di
osservare molecole nello stato b è nulla.

Quella che mi interessa è la probabilità di trovare molecole nello stato
b a tempo t

\[P_{b} = \left| C_{b}\left( t \right) \right|^{2} = \frac{\left| \left\langle \Psi_{b}\left| \overset{\overline{}}{\mu} \right|\Psi_{a} \right\rangle E(0) \right|^{2}}{h^{2}}\frac{\ldots}{2\left\lbrack \frac{\left( \frac{E_{b}}{\mathcal{h}} - \frac{E_{a}}{\mathcal{h}} - \omega \right)t}{2} \right\rbrack^{2}}\]

Il numeratore dipende dal momento di dipolo di transizione. Quindi
all'aumentare di esso aumenta la probabilità di trovare molecole nello
stato b. Se questo termine è 0 non ho possibilità di trovare molecole
nello stato b, che significa che non c'è possibilità di transizione!
Quello che dicevamo prima. Al denominatore invece c'è un termine che
dipende dalla differenza di energia fra i due livelli. Quindi la
probabilità sarà maggiore tanto più vicini sono i livelli. Quindi devo
assorbire radiazione con energia pari alla differenza di energia fra i
livelli energetici della transizione, quindi devo essere in risonanza.

\[\mathcal{h}\omega = h\nu = E_{b} - E_{a}\]

Capitolo 7 del Cantor

\textbf{LIVELLI ENERGETICI}

L'ampiezza delle linee spettrali è conseguenza di:

\begin{itemize}
\item
  Transizioni elettroniche a diversi livelli vibrazionali (invece di una
  singola riga ha molte righe)
\item
  Tempo di vita, perché una molecola trascorre solo un breve periodo di
  tempo nello stato eccitato, che definisce appunto il tempo di vita
  dello stato eccitato \(\tau\). Se una molecola cambia stato alla
  velocità di \(1/\tau\) allora i livelli energetici si allargano e il
  corrispondente allargamento in energia attorno a E è dato da
  \(\delta E \sim \mathcal{h/}\tau\) (principio di indeterminazione di
  Heisenberg).
\item
  Doppler shift, causato dal moto termico degli atomi. Quelli che si
  muovono verso il rivelatore con velocità \(v\) avranno frequenze di
  transizione che differiscono da quelle degli atomi a riposo per un
  fattore
  \(\Delta\omega = \frac{2\omega_{0}}{c\sqrt{2\ln 2\frac{\text{kT}}{m_{0}}}}\)
  dove \(m_{0}\) è la massa atomica e \(\omega_{0}\) la frequenza degli
  atomi a riposo.
\item
  Interazione col solvente
\end{itemize}

\textbf{Livelli energetici elettronici}. A T ambiente le molecole
occupano il livello energetico più basso (stato fondamentale). Quando
assorbono radiazione (UV-Vis) uno degli elettroni più esterni (o un
elettrone spaiato) viene promosso ad un livello energetico superiore.
Questa è una transizione elettronica. La spaziatura energetica fra
livelli elettronici è di decine di kcal/mol (30-80). Nelle molecole ogni
livello energetico è suddiviso in sottolivelli, i \textbf{livelli}
\textbf{energetici} \textbf{vibrazionali}. Energia minore rispetto agli
elettronici, spaziature più piccole (0.01-10 kcal/mol). Energia minore
significa λ maggiore quindi possiamo indurre transizioni vibrazionali
tramite assorbimento di radiazione IR. A loro volta, questi livelli si
suddividono in sottolivelli con spaziatura ancora minore, i
\textbf{livelli} \textbf{energetici} \textbf{rotazionali}. Sono
quantizzati e discreti. Sono gli unici che possono essere eccitati a
temperatura ambiente.

Una molecola può assorbire una radiazione a una certa λ solo se esiste
nella molecola una transizione di pari energia a quella trasportata
dalla radiazione.

\includegraphics[width=5.28125in,height=3.20833in]{media/image22.emf}

\[E = h\nu = \frac{h}{\lambda}\]

\[c = 3 \times 10^{8}\frac{m}{s}\]

\[h = 6.63 \times 10^{- 34}\text{\ J\ s}\]

\textbf{Assorbimento multi-fotone}

La transizione può essere fatta assorbendo un fotone di energia
opportuna. Tuttavia si può fare anche assorbendo più fotoni di energia
minore. Doppio fotone: i fotoni devono avere metà energia rispetto al
singolo fotone, quindi lunghezza d'onda doppia.

Sono processi con probabilità di avvenire molto bassa (sezione d'urto
molto piccola).

Perché questi processi possano avvenire c'è necessità di un flusso
costante ed elevato di fotoni, concentrato spazialmente e temporalmente.
Nonostante fossero teorizzati molto prima, è stato possibile osservarli
sperimentalmente solamente dopo l'invenzione dei laser pulsati. Questi
permettono la transizione della molecola ad uno stato metastabile, in
cui rimane per un tempo molto breve, per poi ricevere il fotone
successivo che la fa transire al livello eccitato.

Se a singolo fotone abbiamo
\(\sigma_{1} \sim \Delta x \sim 10^{- 17} - 10^{- 16}cm^{2}\) e
\(\Delta t \sim 10^{- 16}s\) (tempo medio di vita), a doppio fotone la
sezione d'urto sarà:

\[\sigma_{2} \sim \sigma_{1}\Delta t\Delta x\]

Scoperto da Goeppert-Mayer, a cui è stata dedicata l'unità di misura che
rappresenta la sezione d'urto dell'assorbimento a doppio fotone:

\[1GM = 10^{- 50}cm^{4}s\]

In generale

\[\sigma_{n} \sim \sigma_{1}^{n}\Delta t^{n - 1}\]

dove \(\Delta t\) è il tempo di vita dello stato metastabile e \(n\) è
il numero di fotoni (quindi di salti) che mandiamo.

Il vantaggio dell'eccitazione a due fotoni è che gli spettri di
assorbimento si allargano, non sono la copia moltiplicata per 2 del
singolo fotone (causa: interazioni fra livelli energetici). Ci sono casi
in cui questo è vero (cumarina) ma altri no (rodamina).

Utile perché la maggior parte delle sonde ha buona probabilità di
eccitazione su un grande range, quindi con una stessa lunghezza d'onda
riesco a eccitare diversi coloranti e quindi a studiare la fluorescenza
di tanti aspetti diversi della cellula contemporaneamente, non è
necessario modificare il setup.

Altro vantaggio è che l'assorbimento multi fotone minimizza gli effetti
indesiderati di scattering: infatti lo scattering dipende da
\(1/\lambda^{4}\), quindi lo scattering in IR è molto ridotto rispetto
all'UV. Quindi la radiazione viene focalizzata più precisamente
soprattutto quando si ha a che fare con oggetti opachi.

\textbf{LEGGE DI LAMBERT-BEER}

In spettroscopia di assorbimento misuriamo gli spettri mandando
radiazione monocromatica di intensità \(I_{0}\) verso un campione e
osservando la radiazione uscente \(I\).

Se N è il numero di molecole per \(cm^{3}\) (numero di centri capace di
assorbire):

\[N = \frac{CN_{A}}{1000}\]

\includegraphics[width=2.28125in,height=1.17708in]{media/image23.emf}

Per la legge dell'alternazione

\[dI = - I_{z}\text{σN\ dz}\]

\[\frac{\text{dI}}{I} = - \sigma N\ dz\]

\[\ln{I - \ln I_{0} = - \sigma}\text{NL}\]

Passo al logaritmo in base 10 (fattore 2.303):

\[\mathbf{-}\log\frac{\mathbf{I}}{\mathbf{I}_{\mathbf{0}}}\mathbf{=}\sigma NL = \sigma\frac{N_{A}}{1000}CL = \mathbf{\varepsilon CL = A}\]

Coefficiente di estinzione molare:

\[\varepsilon = \frac{\sigma}{2,303}\frac{N_{\text{AV}}}{1000}\]

L'assorbanza è una quantità adimensionale quindi \(\varepsilon\) sarà
misurato in \(M^{- 1}cm^{- 1}\).

Trasmittanza: \(T = \frac{I}{I_{0}}\)

\[A = - \log T\]

Quanto è veloce una transizione dallo stato fondamentale allo stato
eccitato? La situazione è dinamica: eccitiamo molecole che, dopo il lor
tempo di vita, si diseccitano. L'evoluzione temporale della probabilità
di trovare la particella nello stato eccitato (stato b) ci dà
informazioni:

\[\frac{dP_{b}}{\text{dt}} = \frac{d}{\text{dt}}\int_{}^{}{\text{dν}\left| C_{b}\left( t \right) \right|^{2}} = \frac{1}{2\mathcal{h}^{2}}\left| \left\langle \Psi_{b}\left| \overset{\overline{}}{\mu} \right|\Psi_{a} \right\rangle E\left( 0 \right) \right|^{2}\ \]

La integriamo su tutte le frequenze. La probabilità di avere molecole
nello stato eccitato è data da un coefficiente per l'intensità di
radiazione incidente. Il \(4\pi\) è un fattore che dipende dalle
orientazioni.

\[\frac{dP_{b}}{\text{dt}} = B_{\text{ab}}I\left( \nu \right) \rightarrow I\left( \nu \right) = \frac{\left| E\left( 0 \right) \right|^{2}}{4\pi}\]

\[\left\langle \left| \left\langle \Psi_{b}\left| \overset{\overline{}}{\mu} \right|\Psi_{a} \right\rangle E\left( 0 \right) \right|^{2} \right\rangle_{\text{orientazioni}} = \frac{1}{3}\left| \left\langle \Psi_{b}\left| \overset{\overline{}}{\mu} \right|\Psi_{a} \right\rangle \right|^{2}\left| E\left( 0 \right) \right|^{2}\]

Il fattore 1/3 è per la distribuzione omogenea.

\[B_{\text{ab}} = \frac{2}{3}\frac{\pi}{\mathcal{h}^{2}}\left| \left\langle \Psi_{b}\left| \overset{\overline{}}{\mu} \right|\Psi_{a} \right\rangle \right|^{2}\ \]

Il coefficiente è detto coefficiente di Einstein per l'assorbimento
stimolato.

Inoltre \(B_{\text{ab}} = B_{\text{ba}}\) cioè si ha indipendenza dalla
direzione del processo, quindi è anche il coefficiente del processo di
emissione stimolata. Il termine del momento di dipolo fra b e a è uguale
a quello del processo da a a b. L'emissione stimolata, a differenza di
quella spontanea, è quello che accade quando un fotone provoca la
diseccitazione anziché l'eccitazione (STED).

La diminuzione dell'intensità ad una certa frequenza nel tempo sarà il
valore dell'intensità di partenza per l'intensità ``catturata'' dalle
molecole, e subirà una variazione anche perché le molecole dello stato
eccitato fanno emissione stimolata, aumentando il numero di fotoni. Lo
stesso vale per la popolazione degli stati eccitati. La popolazione
dello stato a diminuirà perché alcune molecole passano allo stato b ma
aumenterà perché ci sono altre molecole che tornano allo stato a.

\[- \frac{\text{dI}\left( \nu \right)}{\text{dt}} = h\nu\left( N_{a}B_{\text{ab}} - N_{b}B_{\text{ba}} \right)I(\nu)\]

Per valutare quanto un oggetto assorba dobbiamo valutare l'intensità
della banda, ovvero l'intensità del momento di dipolo, che quindi andrà
collegato ad un'osservabile. Se supponiamo che in condizioni normali la
popolazione dello stato eccitato sia trascurabile, allora possiamo
trascurare il contributo di discesa delle molecole dallo stato eccitato:
\(N_{b} \sim 0\)

\[- \frac{\text{dI}\left( \nu \right)}{\text{dt}} = h\nu N_{A}B_{\text{ab}}I\left( \nu \right) = h\nu\frac{N_{A}}{1000}B_{\text{ab}}I(\nu)\]

La luce in un tempo \(\text{dt}\) percorre \(\text{dl}\):

\[- dI\left( \nu \right) = - \frac{\text{dI}\left( \nu \right)}{\text{dt}}\frac{\text{dl}}{c} = h\nu\frac{N_{A}}{1000c}B_{\text{ab}}I\left( \nu \right)\text{dl}\]

\[- dI = I\varepsilon C\ dl\]

Il coefficiente di estinzione molare dipende da \(\nu\) perché
l'assorbimento non è uguale a tutte le λ. Sarà tabulato quindi a
specifiche λ.

\textbf{LASER} (Light amplification by stimulated emission of radiation)

Il funzionamento del laser si basa sul fatto che abbiamo un grande
numero di fotoni coerenti che si moltiplicano fra di loro grazie ai
componenti del laser.

Le popolazioni dei livelli cambiano in base alle probabilità dei
processi, date dai coefficienti di Einstein. I processi di emissione
stimolata dipendono dall'intensità di radiazione, mentre quelli
spontanei avvengono naturalmente.

\includegraphics[width=4.36458in,height=1.42708in]{media/image24.emf}

La popolazione del secondo livello avrà le stesse equazioni ma con i
segni cambiati perché ciò che depopola un livello, popola l'altro.

La popolazione di un livello all'equilibrio avrà una distribuzione che
segue la statistica di Boltzmann. L'energia ha una distribuzione in
frequenza del tipo

\[\rho\left( \nu \right) = \frac{8\pi h\nu^{3}}{c^{3}}\frac{1}{e^{\frac{\text{hν}}{\text{kT}}} - 1}\]

Il numero di oggetti di un livello dipenderà da

\[N_{i} = g_{i}N_{0}e^{- \frac{E_{i}}{\text{kT}}}\]

All'equilibrio:

\[B_{12}N_{1}\rho\left( \nu \right) - A_{21}N_{2} - B_{21}N_{2}\rho\left( \nu \right) = 0\]

Quindi necessariamente (deve valere a tutte le frequenze)

\[\frac{A_{21}}{B_{21}} = \frac{8\pi h\nu^{3}}{c^{3}}\]

\[\frac{B_{21}}{B_{12}} = \frac{g_{1}}{g_{2}}\]

Emissione stimolata:

\includegraphics[width=4.02649in,height=1.96850in]{media/image25.emf}

Prima dell'emissione l'elettrone è eccitato, arriva un fotone e
l'elettrone si diseccita per emissione di due fotoni. I fotoni emessi
hanno la stessa λ, la stessa direzione e sono coerenti. Questo è il
vantaggio del laser sulla lampadina. Due sono le cose fondamentali:

\begin{itemize}
\item
  Bisogna avere inversione di popolazione. Infatti per avere emissione
  stimolata devo avere il livello eccitato popolato, quindi il sistema
  deve essere in qualche misura eccitato.
\item
  Lo stato eccitato deve essere uno stato metastabile con tempo di vita
  lungo (a differenza di quello dell'assorbimento multifotone), dove i
  fotoni si accumulano. Questo perché se non ci fosse tale stato,
  mandando un fotone questo andrebbe a rieccitare l'elettrone, anziché a
  diseccitarlo e si entrerebbe un ciclo di eccitazione, diseccitazione
  inutile. In questo modo la transizione in giù avviene per emissione
  stimolata e non spontaneamente.
\end{itemize}

\textbf{TRANSIZIONI ELETTRONICHE}

Con la spettroscopia UV-Vis osserviamo essenzialmente le transizioni
elettroniche.

Ci sono due tipi di orbitali molecolari:

\begin{itemize}
\item
  Localizzati σ
\item
  Delocalizzati \(\pi\)
\end{itemize}

Si parla di stati di legame quando gli elettroni favoriscono
l'avvicinamento dei nuclei. Altrimenti si parla di stati di antilegame.

Ogni orbitali può contenere al massimo due elettroni con spin
antiparallelo. A parità di energia è più stabile lo stato a massima
molteplicità.

In sistemi contenenti CNO lo stato elettronico fondamentale è quello di
singoletto (tutti gli orbitali sono occupati da coppie di elettroni). Si
ha doppietto quando si hanno singoli elettroni spaiati (radicali) e si
ha tripletto quando ci sono due elettroni spaiati.

Dal punto di vista elettronico si possono avere le seguenti transizioni:

\begin{itemize}
\item
  \(\sigma \rightarrow \sigma^{*}\) transizione di un elettrone fra il
  suo stato legante σ allo stato antilegante sempre σ, è quella a
  maggiore energia. Non riusciamo a osservarla in lab (λ bassissima:
  125nm)
\item
  \(\pi \rightarrow \pi^{*}\) uguale ma fra gli elettroni pigreco.
  Tipica dei composti con legami multipli (aromatici, carbonili). Da 170
  a 205nm (già è più osservabile)
\item
  \(n \rightarrow \sigma^{*}\) transizione da non legante a antilegante
  σ; non richiede grande energia, tipica di composti saturi contenenti
  atomi con elettroni spaiati come ONS. Da 150 a 250nm, utile se non
  fosse che non ci sono gruppi funzionali con picchi nella regione UV.
\item
  \(n \rightarrow \pi^{*}\) anche qui legami doppi CO, NO, e triplo CN.
  Richiedono energie molto basse. Picchi attorno ai 300nm.
\item
  \(\sigma \rightarrow \pi^{*},\ \pi \rightarrow \sigma^{*}\) sono
  transizioni proibite (possibili solo dal punto di vista teorico).
\end{itemize}

In definitiva solo le transizioni \(n \rightarrow \pi^{*}\) e
\(\pi \rightarrow \pi^{*}\) mostrano picchi di assorbimento sopra i
200nm, regione accessibile agli spettrofotometri UV-Vis.

Strumento: \textbf{spettrofotometro}. Viene mandata luce monocromatica
(singola λ) che attraversa la sostanza in esame e la luce che ne esce
viene mandata al detector. I componenti fondamentali sono: monocromatore
di eccitazione (fenditura) e il tubo fotomoltiplicatore in emissione. I
dinodi del fototubo sono fotoemissivi quindi un elettrone che ci sbatte
sopra produce altri elettroni, moltiplicandoli in numero.

Se \(n\) è il numero di passaggi e ogni dinodo moltiplica per 4 il
numero di elettroni avremo che la resa del fototubo è il fattore di
amplificazione elevato al numero di dinodi. In questo esempio
\(resa = 4^{n}\).

L'assorbanza è definita come \(A = - \log T\). Un'assorbanza pari a 1
significa che l'intensità iniziale è 10 volte quella raggiunta al
detector. Un'assorbanza di 2 comporta un fattore 100, un'assorbanza di 3
comporta un fattore 1000, che sono quantità di luce trascurabili. Sopra
un certo range gli strumenti non sono affidabili.

Quindi devo stare bassa con l'assorbanza, e per farlo mi basta cambiare
il parametro concentrazione. Misurare la concentrazione passando per
l'assorbanza è utile se l'assorbanza rimane molto bassa (sotto 1).

Il cromoforo è la parte che dà il colore alla molecola. Sono utili gli
auxocromi, gruppi funzionali che alterano l'emissione del cromoforo.
Possono alterarla in 4 modi.

\includegraphics[width=3.35417in,height=2.37246in]{media/image26.emf}

\begin{itemize}
\item
  Effetto batocromico: spostamento verso il rosso (diminuzione di
  energia)
\item
  Effetto ipsocromico: spostamento verso il blu
\item
  Ipercromico: aumento di intensità
\item
  Ipocromico: diminuzione di intensità
\end{itemize}

Tali shift sono causate da interazioni con il solvente oppure da
interazioni con altre molecole.

Le interazioni col solvente sono dovute al fatto che gli orbitali
\(\pi^{*}\) indicano stati eccitati più polari, stabilizzati da
associazioni con solventi polari. Quindi un passaggio da solvente non
polare a solvente polare si ha effetto batocromico. Aumenta λ e
diminuisce energia.

Cosa assorbe nelle proteine? Triptofani, fenilalanina, tirosina
(aromatici), gruppi cisteina, legame peptidico. Negli acidi nucleici
invece assorbono tutte le basi (ATCG).

Quindi dividendo grossolanamente:

\begin{itemize}
\item
  \textbf{Legame} \textbf{peptidico}. Abbiamo picchi molto alti a 190nm
  (\(\pi \rightarrow \pi^{*}\)) e un picco più basso a 210nm
  (\(n \rightarrow \pi^{*}\)). Il problema è che ci sono tanti legami
  peptidici in una molecola: se anche il singolo legame peptidico ha un
  coefficiente di estinzione molare basso, considerandoli
  complessivamente ci sarà un assorbimento elevato e questo è un
  problema perché abbiamo detto che l'assorbanza deve stare bassa.
  Quindi ciò implica concentrazioni molto piccole e difficili da
  preparare.
\item
  Gli \textbf{aa} \textbf{aromatici} hanno picchi a λ molto diverse da
  quelle del legame e inoltre, anche se hanno coefficienti di estinzione
  molare alti, ce ne sono pochi nelle proteine. Alcuni residui hanno
  picco di assorbimento a 230nm, ma è molto vicino a quello del legame
  peptidico e quindi vengono ``oscurate'' dalla predominanza del legame
  sugli effetti di assorbimento. Ci sono alcune proteine che possiedono
  gruppi prostetici (tipo il gruppo EME nell'emoglobina) che assorbono.
\end{itemize}

Il valore del coefficiente di estinzione è sensibile alle variazioni
conformazionali della proteina, quindi possiamo studiare il grafico di
\(\varepsilon\) in funzione di λ sapendo che α-eliche, β-sheets e random
coil provocano grafici diversi.

Le catene aromatiche sono più interessanti. Tipicamente è l'assorbimento
del triptofano e il picco associato che consente di misurare le
concentrazioni tramite i coefficienti di estinzioni tabulati in quella
posizione (280nm). Tirosina e fenilalanina assorbono invece a 260nm.

I carotenoidi presentano 3 picchi nel Vis grazie ai tanti doppi legami
che possiedono.

Negli acidi nucleici le bande sono tutte nella stessa zona. Il picco è
molto sensibile al pH. Quando le basi sono aggregate in polimeri le
interazioni elettroniche fra le basi causano un pesante effetto
ipocromico per cui diminuisce l'assorbanza. Il melting del DNA si misura
tramite assorbimento e quando la denaturazione raggiunge il melting
(separazione della catena) si ha un aumento dell'intensità di
assorbimento sigmoide.

Il coefficiente di estinzione molare è dato per coppia di paia di basi.

\textbf{FLUORESCENZA DI BIOMOLECOLE}

\includegraphics[width=3.07110in,height=2.75591in]{media/image27.jpeg}

\textbf{Diagramma di Jablonski}. Coordinata spaziale -- coordinata
energetica. S stati di singoletto, T stato di tripletto. Livelli
vibrazionali a cui si sovrappongono i livelli rotazionali.
L'assorbimento porta da uno stato vibrazionale qualsiasi dello stato
fondamentale a uno stato vibrazionale qualsiasi, ma dello stato
eccitato. Poi ci sono processi di perdita di energia interna senza
emissione di radiazione (conversione interna), che può avvenire:

\begin{itemize}
\item
  Fra stati con la stessa molteplicità conversione interna
\item
  Fra stati con molteplicità diversa: da singoletto a tripletto
  intersystem crossing.
\end{itemize}

Tali processi (conversione interna e rilassamenti vibrazionali) sono
molto veloci (picosecondi, \(10^{- 12}s\)) e, una volta arrivati nello
stato vibrazionale fondamentale del primo singoletto eccitato, per fare
il salto energetico per tornare allo stato elettronico fondamentale,
possiamo avere interazioni con diseccitazioni di tipo:

\begin{itemize}
\item
  Non radiativo (FRET, quenching..)
\item
  Radiativo fluorescenza
\end{itemize}

Fluorescenza se viene dal primo stato (fondamentale vibrazionale) di
singoletto eccitato, dura decine/centinaia di ns.

Fosforescenza se viene dal tripletto, può durare da
\(10^{- 3} - 10^{2}\)s. Questo processo è vietato dalle regole di
selezione.

Inizialmente parleremo di processi unimolecolari, interazione di una
molecola con se stessa (conversione interna e intersystem crossing), poi
di processi bimolecolari (quenching, energy transfer), che coinvolgono
processi di diseccitazione per interazione con altre molecole.

CARATTERISTICHE DELLA FLUORESCENZA

\begin{itemize}
\item
  \textbf{Stokes shift}: l'emissione avviene sempre a energie minori
  rispetto all'assorbimento quindi a lunghezze d'onda più alte; questo
  provoca uno shift del picco. Infatti, l'assorbimento porta la molecola
  a uno stato vibrazionale eccitato di uno stato elettronico eccitato,
  poi la molecole per conversione interna rilassa allo stato
  fondamentale del livello eccitato \(S_{1}\). La fluorescenza si
  verifica esclusivamente per il passaggio dallo stato vibrazionale
  fondamentale dell'\(S_{1}\) allo stato fondamentale vibrazionale
  dell'\(S_{0}\).
\end{itemize}

\begin{quote}
L'ampiezza dello shift dà informazioni sull'energia dei livelli
vibrazionali, perché se c'è un grande Stokes shift vuol dire che c'è
grande differenza energetica fra lo stato eccitato e quello
fondamentale; se invece è piccolo significa che lo spread energetico fra
gli stati è piccolo.

Ci sono molecole che hanno Stokes shifts molto diversi: alcune hanno
shift quasi sovrapposti (perilene), altre hanno uno spread fra i due
picchi molto più grande (solfato di quinino).
\end{quote}

\begin{itemize}
\item
  \textbf{Regola dello specchio}. lo spettro di emissione è l'immagine
  speculare dello spettro di assorbimento. Questo perché i livelli
  \(S_{0}\) e \(S_{1}\) presentano strutture vibrazionali simili.
\item
  \textbf{Regola di Kasha}: lo spettro di fluorescenza è indipendente
  dalla lunghezza d'onda di eccitazione. La forma dello spettro non
  cambia, ma cambia la probabilità di assorbimento, quindi il numero di
  fotoni che danno fluorescenza e quindi l'intensità.
\item
  \textbf{Tempo di vita naturale}. Esistono processi di assorbimento
  stimolato e di emissione, stimolata o spontanea. All'equilibrio il
  numero di molecole che transiscono da uno stato all'altro è uguale a
  quello del processo inverso. Una molecola occupa il suo stato
  fondamentale a meno che non venga fornita energia sufficiente per
  effettuare la transizione. Tenendo conto della distribuzione in
  frequenza della radiazione (Planck) e del fattore di Boltzmann si
  possono ottenere le espressioni dei coefficienti.
\end{itemize}

\begin{quote}
Conoscendo il coefficiente di estinzione molare, posso calcolare il
coefficiente di estinzione spontanea.

In assenza di radiazione esterna non ci sono né assorbimento né
emissione stimolata, quindi la popolazione del livello eccitato decadrà
nel tempo solo a seconda del coefficiente di emissione spontanea, il cui
inverso è il tempo di vita naturale o radiativo:
\(\tau_{R} = \frac{1}{A_{10}}\) :
\(\frac{dn_{1}}{\text{dt}} = - A_{10}n_{1} \rightarrow n_{1}\left( t \right) = n_{1}\left( 0 \right)e^{- A_{10}t}\).

Supponiamo invece che adesso sulla mia molecola arrivino un tot di
fotoni.

La popolazione dello stato eccitato nel tempo cambia perché assorbe
fotoni nel passaggio allo stato eccitato (contributo di \(I_{0}\)), poi
il passaggio inverso può essere radiativo o non radiativo:
\(\frac{dS_{1}}{\text{dt}} = I_{0} - \left( k_{R} + k_{\text{NR}} \right)S_{1}\).
\(I_{0}\) è il numero di fotoni assorbiti dal sistema,
\(k_{R},k_{\text{NR}}\) sono i coefficienti delle diseccitazioni
radiativa (emissione spontanea) e non radiativa (ic, rilassamenti..)
rispettivamente, che contribuiranno a cambiare la popolazione in modo
negativo.

In condizioni stazionarie, \(\frac{dS_{1}}{\text{dt}} = 0\), quindi
possiamo definire la resa quantica.
\end{quote}

\begin{itemize}
\item
  \textbf{Resa quantica}. All'equilibrio è il rapporto fra i fotoni
  emessi per fluorescenza (numero di diseccitazioni radiative) e il
  numero di diseccitazioni totali. Quindi è la tendenza di una molecola
  a diseccitarsi con emissione di radiazione:
\end{itemize}

\[\phi = \frac{k_{R}S_{1}}{I_{0}} = \frac{k_{R}}{k_{R} + k_{\text{NR}}}\]

\begin{quote}
Ha valore compreso fra 0 e 1.
\end{quote}

\begin{itemize}
\item
  \textbf{Tempo di vita di fluorescenza}. Viceversa in condizioni
  transienti, abbiamo inizialmente una certa \(I_{0}\) che viene poi
  spenta \(\rightarrow I_{0} = 0\). La popolazione decadrà nel tempo
  secondo
  \(S_{1}\left( t \right) = S_{1}\left( 0 \right)e^{- \frac{t}{\tau_{F}}}\)
  quindi introduciamo il tempo di vita di fluorescenza.
\end{itemize}

\begin{quote}
È quello che osserviamo sperimentalmente perché non tutte le
diseccitazioni avvengono per emissione di fotoni:
\(\tau_{F} = \frac{1}{k_{R} + k_{\text{NR}}}\) .
\end{quote}

Il tempo di vita radiativo è quello che osserveremmo se non avessimo
alcun altro modo di diseccitare la molecola per una molecola con resa
quantica ideale pari a 1. Il tempo di vita di fluorescenza è quello che
osserviamo perché esistono altri tipi di diseccitazione disponibili per
la molecola e che non comportano emissione di fotoni.

Fra i due il tempo di vita di fluorescenza è sempre più piccolo.

\[\phi = \frac{\tau_{F}}{\tau_{R}} < 1\]

Ciò che misuriamo sperimentalmente è l'intensità di fluorescenza al
variare della lunghezza d'onda. L'intensità di fluorescenza dipende
dalle probabilità di avere assorbito fotoni e di avere emissioni di
fotoni.

\[F\left( \lambda \right) \div P_{\text{abs}} \cdot P_{\text{em}}\]

La probabilità di emissione dei fotoni a una certa λ dipende dalla
probabilità che la diseccitazione avvenga per emissione di fotoni e
dalla probabilità che i fotoni vengano emessi proprio a tale λ, quindi
dalla frazione di fotoni emessi a quella λ. Inoltre c'è dipendenza anche
da altri fattori strumentali di sensibilità del rilevatore.

\[P_{\text{em}} \div \phi \cdot f\left( \lambda_{\text{em}} \right) \cdot d\]

La probabilità di assorbimento dei fotoni dipende dall'intensità di
radiazione che viene assorbita a tale λ, che è la differenza fra quella
incidente e quella trasmessa.

\[P_{\text{abs}} \div I_{0} - I\]

La relazione fra le due intensità è di tipo esponenziale, per la
definizione di assorbanza:

\[I = I_{0}e^{- 2.303\varepsilon\left( \lambda_{\text{abs}} \right)\text{CL}}\]

Se \(\varepsilon CL < 0.05\) è piccolo, posso fare lo sviluppo di Taylor
e trovare

\[I_{0} - I = 2,303 \cdot \varepsilon\left( \lambda_{\text{abs}} \right)\text{CL}I_{0}\]

\[F\left( \lambda \right) = \varepsilon\left( \lambda_{\text{abs}} \right)\text{ϕf}\left( \lambda_{\text{em}} \right)\text{\ C\ }I_{0}\text{\ K}\]

Quindi la fluorescenza dipende in generale da tre tipi di parametri:

\begin{itemize}
\item
  Quelli che dipendono dal colorante usato: \(\phi,\varepsilon\)
\item
  Quelli che dipendono da come è stato preparato il campione: \(C,L\)
\item
  Quelli che dipendono dallo strumento: \(I_{0},\ K\)
\end{itemize}

Oss. La fluorescenza è lineare nella concentrazione a basse assorbanze e
dipende da \(K\), un parametro strumentale, che rende la misura non
assoluta. Non ha senso confrontare dati presi con strumenti diversi e la
fluorescenza non ha perciò un'unità di misura.

La fluorescenza ha una lunghezza massima di osservazione, che invece non
dovrebbe dipendere dal fluorimetro usato.

\includegraphics[width=3.71780in,height=2.75591in]{media/image28.emf}

\begin{itemize}
\item
  La frazione di fotoni raccolta rispetto al numero di fotoni
  complessivamente disponibili è il 3\%, e rappresenta l'efficienza
  dello strumento (tuttavia Webb aveva lavorato su microscopi, che hanno
  efficienza minore dei fluorimetri in generale).
\item
  Perché proprio 0.05 come limite superiore al valore di \(\text{εCL}\)?
  È per l'effetto dell'inner filtering: l'intensità di fluorescenza ha
  un picco in corrispondenza di 0.05 e poi diminuisce, quindi individua
  un range in cui misurare la fluorescenza è conveniente.
\end{itemize}

Questo ci impone di non usare concentrazioni troppo elevate.

Sbiancamento (\textbf{photobleaching}). Altro problema. Le molecole sono
in un ciclo di eccitazione-emissione ma può accadere che passino in
stato di tripletto, ed è possibile anche che finiscano in uno stato
``dark'', dove non emettono più. A questo punto le molecole sono fuori
dal ciclo, di eccitazione-emissione e si avrà una diminuzione apparente
della concentrazione della sostanza. Il photobleaching dà un limite alla
quantità di fotoni che possiamo ottenere dalla nostra soluzione nel
tempo. È una distruzione irreversibile del fluoroforo che avviene allo
stato eccitato quindi riguarda processi di diseccitazione non radiativi
che avvengono nello stato fondamentale e portano il fluoroforo in uno
stato dark.

È proporzionale all'intensità di eccitazione integrata nel tempo, che è
la quantità di fotoni che stiamo sparando sulla molecola, ovvero il
numero di cicli eccitazione-emissione che effettua. Per evitare si può

\begin{itemize}
\item
  minimizzare l'intensità di eccitazione, per una mera questione di
  probabilità
\item
  massimizzare l'efficienza della rivelazione
\end{itemize}

Se definiamo \(\phi_{B}\) la resa di bleaching, il rapporto
\(\phi_{F}/\phi_{B}\) dà il numero totale di fotoni emessi per molecola,
ovvero il numero di cicli di fluorescenza che può fare prima del
bleaching.

Si può calcolare anche il rate di bleaching dalla diminuzione
dell'intensità di fluorescenza.

Rate dei cicli di eccitazione-rilassamento: \(R = I_{0}\text{OD}\)
dipende dall'assorbanza e dalla quantità di fotoni che stiamo mandando.

\(OD = A\) a una certa λ e \(L = \varepsilon_{\lambda}c_{0}\), \(c_{0}\)
concentrazione prima del bleaching.

La velocità limite del processo di eccitazione è legata al coefficiente
di estinzione molare e al tempo di vita:

\[R_{\max} \propto \frac{1}{\varepsilon_{\lambda}\tau_{F}}\]

Il rate di fluorescenza dipende da quante cose si sono eccitate e dalla
resa di fluorescenza:

\[k_{F} = \phi_{F}R\]

Analogamente per il bleaching:

\[k_{\text{Bleaching}} = \phi_{B}R\]

L'effetto del bleaching è quello di diminuire nel tempo la quantità di
molecole che danno fluorescenza, in modo esponenziale:

\[F\left( t \right) = k_{F}e^{- k_{B}t}\]

Semplicemente per effetti di bleaching, la fluorescenza diminuirà
esponenzialmente, senza che ci siano altri effetti dinamici. I nuovi
fluorimetri ovviano a questo problema mandando radiazione per il minimo
tempo possibile.

\textbf{Strumento}: fluorimetro, come l'assorbimento. A singolo raggio.
Sorgente, filtri, monocromatori (doppio grating, doppia fenditura),
portacampioni, rivelatori. La differenza fondamentale è che la
fluorescenza viene osservata a 90° perché i fotoni emessi hanno la
stessa probabilità di emissione in tutto l'angolo solido, quindi tanto
vale mettersi a 90° così siamo sicuri che non ci arrivi anche luce
diretta dalla sorgente, che arriva anche con intensità superiore
rispetto a quella di fluorescenza, e quindi darebbe abbastanza fastidio.

I fluorimetri fanno sia spettri di eccitazione che di emissione. Mando
una λ per eccitare e vario tutto il range di λ in cui osservo
fluorescenza per vedere dove ho il picco.

Sorgenti tipiche: xenon, tungsteno.

Poi ci sono i filtri:

\begin{itemize}
\item
  Di eccitazione: deve garantire che al campione arrivi solo la
  lunghezza d'onda che vogliamo. L'ideale sarebbe avere un tot di filtri
  interferenziali che coprano solo 3-4nm. Dovrebbe tagliare la luce
  proveniente dall'esterno, tutta quella diversa dalla mia luce di
  eccitazione.
\item
  Di emissione. Filtro passa alto: fa passare lunghezze d'onda superiori
  a quella di eccitazione. Filtro passa banda: fanno passare un grosso
  range di λ, centrato sul massimo di emissione. Sono migliori dei
  filtri passa alto. Scongiurano anche i problemi di scattering,
  emissione del fotone alla stessa λ, ma anche le armoniche superiori si
  vedono. A volte però lo scattering è così forte che non c'è nulla da
  fare.
\end{itemize}

Il fluorimetro del lab non richiede filtri ma si possono mettere filtri
passa alto, anche se non quelli passa banda.

Portacampioni. Il materiale deve essere trasparente alla radiazione che
gli arriva. Devono avere 4 facce ottiche (2 facce ottiche per
l'assorbimento). Facce smerigliate diffonderebbero ulteriormente la luce
quindi aggiungerebbero scattering.

Fotomoltiplicatori. Non sono ugualmente efficienti a tutte le λ,
soprattutto sopra i 1600?nm l'efficienza dei fotomoltiplicatori crolla.
Se il sistema non corregge la probabilità di rivelazione del fm noi
vediamo gli spettri spostati verso λ più basse, quindi pure la lunghezza
d'onda al picco la vedremmo sbagliata.

È fondamentale correggere lo spettro sottraendo il contributo della
lampada.

L'altro parametro da scegliere è l'ampiezza delle fenditure: 1,2,10,20nm
è la risoluzione del monocromatore, e dice quando integra sulle λ.
L'ampiezza delle fenditure incide sulla risoluzione osservabile sullo
specchio. Se aumento l'ampiezza delle fenditure faccio una media del
segnale e se ho dei picchi fini finisce che li perdo. In spettri che non
hanno picchi piccoli, non cambia nulla, però aumenta l'intensità di
radiazione che arriva al rivelatore.

Proprietà misurabili con la fluorescenza:

\begin{itemize}
\item
  Spettri
\item
  Tempi di vita della fluorescenza
\item
  Polarizzazione
\item
  Trasferimento eccitazione
\item
  Localizzazione della fluorescenza immagini di fluorescenza
\end{itemize}

Vantaggi dell'uso della fluorescenza:

\begin{itemize}
\item
  Sensibile a basse concentrazioni
\item
  Sensibile a piccole variazioni dell'environment
\item
  Ci sono molti fluorofori sensibili ognuno a una cosa diversa
\end{itemize}

Informazioni ottenibili:

\begin{itemize}
\item
  Forma spettri rilassamento solvente, polarità
\item
  Rese quantiche relative cambiamenti environment
\item
  Informazioni sul legame con sonde che cambiano fluorescenza da libere
  a legate
\end{itemize}

Fluorofori:

\begin{itemize}
\item
  Intrinseci: fenilalanina, tirosina, triptofano (molto sensibile alla
  polarizzazione)

  \begin{itemize}
  \item
    Formano un legame fisico
  \item
    Formano un legame chimico, legati indissolubilmente
  \item
    Proteine fluorescenti che possono essere legate ad altre proteine di
    interesse nella cellula, facendo da marker. Hanno fenilalanina,
    tirosina e triptofano.
  \end{itemize}
\end{itemize}

\begin{quote}
Il triptofano ha uno spettro che dipende molto dalla polarità
dell'ambiente circostante. Viene usato tipicamente per misure di
denaturazione delle proteine. In ambiente apolare ha λ molto corta fino
a 350nm in acqua. Tale grande variazione è un buon indicatore di cosa
sta succedendo alla proteina. Tirosina e triptofano hanno spettri
sovrapposti fino ai 290nm. Quindi se voglio studiare solo il triptofano,
devo isolare la tirosina, quindi devo eccitarlo dove son sicura di non
eccitare anche la tirosina, quindi eccito da 295nm. Non eccitare il trpt
al suo massimo non ha effetto sul suo spettro, ma solo sull'intensità
(regola di Kasha).
\end{quote}

\begin{itemize}
\item
  Sintetici. Aggiunti dall'esterno. Fluorescina (verde), Texas Red
  (rosso), DAPI (blu).

  \begin{itemize}
  \item
    ANS: altissima sensibilità al solvente per polarità. Provoca forte
    aumento intensità fluorescenza + spostamento leggero del picco verso
    il blu.
  \item
    PRODAN: anche qui elevata sensibilità al solvente. Utile per
    misurare quanto siano permeabili le membrane.
  \item
    Fluorescina, rodamina: poca sensibilità alla polarità
    dell'environment ma alte rese quantiche. Possono essere
    funzionalizzate in modo che si leghino ad anticorpi specifici
    all'interno delle cellule (labeling di anticorpi). Ne esistono vari
    tipi che si differenziano per posizione del picco di emissione. È
    utile comunque averne diversi anche se si legano alle stesse cose
    perché così posso avere, in un'unica immagine, componenti diverse
    legate a fluorofori diversi. Questo permette di valutare le
    interazioni fra oggetti.
  \item
    Famiglia BODIPY.
  \item
    Famiglia ALEXA. Nominate a seconda della loro λ di emissione.
  \item
    Famiglia CY.
  \end{itemize}
\end{itemize}

Come funziona il legame fra fluorofori e oggetti che devo osservare? È
utile avere un legame stabile, covalente. Tre tipi di legame:

\begin{itemize}
\item
  Isotiocianato: reagisce con le amine (sicuramente quella terminale,
  più eventualmente nei residui). A seconda del pH si seleziona il
  legame esclusivo con l'amina primaria (terminale).
\item
  Maleimide. Si lega ai gruppi SH quindi a cisteine spaiate.
\item
  Succinimidilestere.
\end{itemize}

Capitolo 8 Cantor

\textbf{FRET: TRASFERIMENTO RISONANTE DI ENERGIA}

{[}All'aumentare della temperatura, si ha una diminuzione intrinseca
dell'intensità di fluorescenza data dall'aumento dell'agitazione
molecolare che favorisce i processi di diseccitazione non radiativa. Ad
esempio, nella denaturazione, si espone il triptofano al solvente. Se
guardiamo la lunghezza d'onda quella rimane invariata; se invece
utilizziamo delle altre sonde (tipo l'ANS) allora il parametro non è più
solo la lunghezza d'onda ma anche l'intensità di fluorescenza. Ci sarà
una baseline indipendente dal cambiamento conformazionale della
proteina, ma dovuta intrinsecamente a questo aumento della probabilità
di diseccitazione n radiativa. Ciò che si osserva è una variazione
dovuta alla transizione e una variazione dovuta alla variazione di
temperatura, tipicamente un'esponenziale. Se la variazione è monotona
allora significa che non sta succedendo nulla. Se invece abbiamo una
diminuzione normale, e poi andamento fortemente diverso, questo indica
un cambiamento conformazionale. Dove non c'è cambiamento conformazionale
dovrei ottenere curve parallele in funzione della temperatura, e dove ho
deviazioni invece ci sarà un qualche tipo di cambiamento
conformazionale.{]}

\includegraphics[width=3.42708in,height=3.35613in]{media/image29.emf}

Processo di diseccitazione non radiativo che coinvolge l'interazione fra
due molecole, donore e accettore. Il donore trasferisce in modo
risonante la sua energia all'accettore; il donore non emette radiazione
e trasferisce, tramite interazione dipolo-dipolo, la sua energia
all'accettore. Poi noi possiamo eccitare il donore alla sua lunghezza
d'onda di assorbimento e osserviamo l'emissione di fluorescenza
dell'accettore (che è a lunghezza d'onda diversa rispetto all'emissione
del donore), che è la luce FRET. Osserveremo: emissione di fluorescenza
dell'accettore a seconda della sua resa di fluorescenza (indice della
probabilità dell'accettore di fluorescere) e una diminuzione della
fluorescenza del donore (o non lo vediamo più oppure diminuirà il suo
tempo di vita come conseguenza della diminuzione della resa quantica).

Ci sono delle condizioni perché si verifichi il FRET.

\begin{itemize}
\item
  Essendo accoppiamento dipolo-dipolo ci sarà una distanza
  caratteristica, ideale perché avvenga il processo; non c'è un range
  infinito.
\item
  L'energia che perde il donore deve essere utile per l'eccitazione
  dell'accettore. Lo spettro di emissione del donore si deve sovrapporre
  (almeno in parte) allo spettro di assorbimento dell'accettore.
  L'ampiezza della sovrapposizione degli spettri è l'integrale di
  sovrapposizione \(J\left( \lambda \right)\), dipende da:

  \begin{itemize}
  \item
    Probabilità di assorbimento dell'accettore della radiazione a quella
    λ
  \item
    Frazione di emissione del donore che avviene a quella λ
  \item
    Alla λ stessa
  \end{itemize}
\end{itemize}

\begin{quote}
Il parametro caratteristico \(R_{0}\ \)che descrive la probabilità di
accadimento del FRET dipende, oltre che dall'integrale di
sovrapposizione e da altri parametri numerici, anche da un
\end{quote}

\begin{itemize}
\item
  Fattore di orientazione, che indica l'orientazione del dipolo di
  emissione del donore rispetto a quello di assorbimento dell'accettore.
\end{itemize}

La probabilità reale di FRET dipende dal tempo di vita del donore e
dalla distanza effettiva fra le molecole, e dipende dalla sesta potenza
dei parametri:

\[k_{\text{ET}} = \frac{1}{\tau_{D}}\left( \frac{R_{0}}{R} \right)^{6}\]

Da dove vengono questi parametri?

Regola d'oro di Fermi. Lega la probabilità di avere una transizione da
uno stato a a uno stato b al momento di dipolo. Supponiamo di applicare
questa regola alla transizione in cui il donore passa dallo stato
eccitato a quello fondamentale e l'accettore passa dal fondamentale
all'eccitato. Consideriamo una singola frequenza emessa dal donore e
presa dall'accettore. Il rate della transizione a questa λ sarà
proporzionale al valore d'aspettazione per il potenziale di interazione,
che è dipolo-dipolo. Introduciamo un fattore di orientazione \(\kappa\):

\[k_{\text{ET}} = \frac{\kappa^{2}\phi_{D}}{R^{6}\tau_{D}}\int_{}^{}{\varepsilon_{A}\left( \nu \right)f_{D}\left( \nu \right)\nu^{- 4}\text{dν}}\]

Quello che ci interessa quantificare è l'efficienza di transfer E
rispetto a tutti i processi di diseccitazione (fluorescenza, conversione
interna, intersystem crossing): frazione di donori eccitati che decadono
attraverso il processo di FRET.

\[E = \frac{k_{\text{ET}}}{k_{\text{ET}} + k^{D}} = \frac{k_{\text{ET}}}{k_{\text{ET}} + k_{F}^{D} + k_{\text{CI}}^{D} + k_{\text{IC}}^{D}}\]

\includegraphics[width=3.28125in,height=2.50000in]{media/image30.emf}

Quando le molecole sono a distanza caratteristica E=0,5.

Come si ricava E?

\begin{itemize}
\item
  Fluorescenza in stato stazionario. Analisi di spettri. Prima eccitiamo
  e osserviamo solo donore, poi donore+accettore. Quando c'è anche
  l'accettore la resa quantica sarà
\end{itemize}

\[\frac{\phi_{D + A}}{\phi_{D}} = \frac{k_{F}^{D}}{k_{F}^{D} + k_{\text{ci}}^{D} + k_{\text{ic}}^{D} + k_{\text{ET}}} \times \frac{k_{F}^{D} + k_{\text{ci}}^{D} + k_{\text{ic}}^{D}}{k_{F}^{D}} = 1 - E\]

\begin{quote}
Se c'è solo l'accettore. Eccitiamo il donore ma andiamo a osservare alla
λ di emissione dell'accettore.
\end{quote}

\[\frac{F_{D + A}}{F_{A}} = \frac{\varepsilon_{A}C_{A}\phi_{A} + \varepsilon_{D}C_{D}\phi_{A}E}{\varepsilon_{A}C_{A}\phi_{A}} = 1 + \frac{\varepsilon_{D}C_{D}}{\varepsilon_{A}C_{A}}E\]

\begin{itemize}
\item
  Fluorescenza risolta nel tempo. Si misurano i tempi di vita.
\end{itemize}

\[\frac{\tau_{D + A}}{\tau_{D}} = 1 - E\]

A cosa serve il FRET? Studiare i cambiamenti conformazionali in una
proteina; vedere se enzimi provocano la rottura di una macromolecola,
che quindi non è più in grado di fare FRET; vedere se un ligando si lega
e provoca il FRET.

Distinguiamo fra:

\begin{itemize}
\item
  FRET intramolecolare, quando leghiamo donore e accettore a una stessa
  moelcola per osservare i cambiamenti conformazionali (in una
  conformazione fanno FRET, nell'altra no)
\item
  FRET intermolecolare.
\end{itemize}

Tirosina e triptofano fanno FRET.

Altra applicazione è per stimare l'angolo di piegamento in una giunzione
di DNA. Dipende dalla concentrazione di sali in soluzione. Si studia con
fluorescina e rodamina, applicate alle giunzioni. Esse possono fare
inizialmente FRET e poi allontanarsi e non farlo più.

Nelle cellule invece si usano le proteine fluorescenti, purché siano
soddisfatte le condizioni di FRET (sovrapposizione).

Quindi il FRET, essendo molto sensibile alla distanza, è una sorta di
righello spettroscopico.

Sperimentalmente, in assenza di FRET si osserva forte emissione del
donore e bassa dell'accettore; in presenza di FRET avviene l'opposto.

\includegraphics[width=2.84375in,height=2.12031in]{media/image31.emf}

Bisogna stare attenti a dove si raccoglie la fluorescenza, in modo tale
da non includere la fluorescenza dell'accettore. Per il donore non è
difficile, ma per l'accettore sì, perché ha bande di fluorescenza con
lunghe code, molto sovrapposte al donore.

Cameleon. Sonda a base FRET (yellow+ciano). All'aggiunta di calcio le
proteine si avvicinano e fanno FRET. Cambia colore e ha una lunga lingua
che ritrae o si estende a seconda che il calcio sia legato o meno.

\textbf{QUENCHING}

Altro fenomeno bimolecolare che avviene è il quenching: diminuzione
dell'intensità di fluorescenza a seguito dovuta all'interazione con
altre molecole presenti in soluzione. Un tipico quencher è l'ossigeno
\(O_{2}\), quindi riusciremmo a evitare il fenomeno semplicemente
pompando fuori l'O a favore di N, ad esempio.

Ci sono diversi tipi di quenching:

\begin{itemize}
\item
  Auto-quenching
\item
  Quenching chimico
\item
  Statico, legato alla formazione di un complesso
\item
  Dinamico-collisionale, come quello dell'O.
\end{itemize}

L'auto-quenching è un auto-assorbimento, per cui i primi strati di
soluzione assorbono talmente tanto la luce incidente che, procedendo, ai
fluorofori non rimane più luce di eccitazione.

Il quenching chimico è la diminuzione dell'intensità di fluorescenza
dovuta a variazioni di pH, presenza di O, alogeni o metalli pesanti.

\textbf{Quenching dinamico}

Il quenching dinamico è la diseccitazione dovuta a un'interazione.
L'oggetto che provoca quenching diffonde in soluzione, interagisce con
il fluoroforo e ne provoca la diseccitazione non radiativa.

L'intensità statica diminuisce perché c'è diseccitazione quindi ritorno
allo stato fondamentale, secondo la legge:

\[\frac{I_{0}}{I} = 1 + K_{D}\left\lbrack Q \right\rbrack\]

La probabilità di quenching collisionale dipende dalla concentrazione
del quencher in soluzione.

L'intensità dinamica o risolta nel tempo invece riguarda il rapporto fra
i tempi di vita del fluoroforo legato e non legato al quencher e
anch'essa diminuisce nel quenching collisionale, perché abbiamo
esplorato un nuovo processo di diseccitazione.

\[\frac{\tau_{0}}{\tau} = 1 + K_{D}\left\lbrack Q \right\rbrack,\ \ \tau = \frac{1}{k_{F} + k_{q}\left\lbrack Q \right\rbrack},\ \ \tau_{0} = \frac{1}{k_{F}}\]

Dove:

\begin{itemize}
\item
  \(K_{D} = k_{q}\tau_{0}\) è la costante di Stern-Volmer
\item
  \(k_{q} = \gamma\kappa\) è la costante di quenching dinamico

  \begin{itemize}
  \item
    \(\gamma\) è l'efficienza di quenching
  \item
    \(\kappa = \frac{4\pi dR_{0}N_{A}}{1000}\) è la costante di reazione
    bimolecolare per il processo di collisione conseguente alla
    diffusione

    \begin{itemize}
    \item
      \(D\) e \(R_{0}\) sono la somma dei coefficienti di diffusione
      (Stokes-Einstein) e dei raggi di fluorofori e quenchers.
    \end{itemize}
  \end{itemize}
\end{itemize}

All'aumentare della temperatura aumentano le collisioni quindi aumenta
l'efficienza di quenching.

\includegraphics[width=4.57292in,height=2.96762in]{media/image32.emf}

\textbf{Quenching statico}

Il quenching statico comporta la formazione di un complesso non
fluorescente nello stato eccitato. C'è una reazione di associazione
quencher-fluoroforo:

\[Q + F \rightarrow F^{*}Q\]

L'asterisco indica che il fluoroforo è nello stato eccitato.

L'intensità statica si riduce a seconda della concentrazione del
quencher:

\[\frac{I_{0}}{I} = 1 + K_{S}\left\lbrack Q \right\rbrack\]

Intensità risolta nel tempo

\[\frac{\tau}{\tau_{0}} = 1\]

dove \(K_{S}\) è definita come

\[K_{S} = \frac{\left\lbrack F^{*}Q \right\rbrack}{\left\lbrack F \right\rbrack\left\lbrack Q \right\rbrack}\]

Il tempo di vita non cambia: le molecole che si diseccitano continuano a
farlo per fluorescenza, quelle che hanno fatto il complesso non
contribuiscono più. I fluorofori non legati continuano a fare il loro
ciclo eccitazione-emissione. L'effetto del quencher è solo una
diminuzione di concentrazione dei fluorofori. Quindi il tempo di vita
non è utile nel quenching statico. Misure dinamiche (intensità risolta
nel tempo) non rivelano l'eventuale presenza di quenching statico.

È utile dunque abbinare le misure per distinguere quenching statico e
dinamico.

Aumentando la temperatura la \(K_{S}\) diminuisce (è meno probabile la
formazione del complesso) quindi l'efficienza diminuisce.

\textbf{Quenching misto}

È la composizione di quenching statico e dinamico.

\[\frac{I_{0}}{I} = \left( 1 + K_{D}\left\lbrack Q \right\rbrack \right)e^{V\left\lbrack Q \right\rbrack} \sim \left( 1 + K_{D}\left\lbrack Q \right\rbrack \right)\left( 1 + K_{S}\left\lbrack Q \right\rbrack \right)\]

dove \(V\) è l'elemento di volume attivo che circonda il fluoroforo.

L'effetto è una curvatura verso l'alto del grafico di \(I_{0}/I\) in
funzione di \(\lbrack Q\rbrack\).

\textbf{DICROISMO CIRCOLARE}

Il dicroismo circolare è la differenza di assorbimento di luce
polarizzata circolarmente destrogira e levogira. È necessario avere un
cromoforo chirale otticamente attivo (non simmetrica), perché la sua
capacità di assorbire è diversa per la luce polarizzate circolarmente
nei due sensi.

\textbf{Polarizzazione}

Luce non polarizzata: il vettore campo elettrico va in una direzione
qualsiasi.

Luce polarizzata piana: il vettore campo elettrico è limitato a un piano
che non contiene la direzione di propagazione.

Polarizzazione circolare, polarizzazione ellittica. Si chiamano tutte
così in relazione alla figura che traccia il vettore campo elettrico su
un piano perpendicolare alla direzione di propagazione.

Tutti le polarizzazioni della luce si possono considerare come somma di
due componenti polarizzate linearmente: una verticale e uno orizzontale,
perpendicolari. Se prendiamo onde polarizzate linearmente, con stessa
ampiezza e fase otteniamo un'onda polarizzata linearmente posta a 45°.
Le proprietà dell'onda risultante dipendono da intensità delle onde
iniziali e dalla loro differenza di fase.

Se le sfasiamo di λ/4 otteniamo luce polarizzata circolarmente. Per
sfasare la luce di λ/4 si usa una lamina quarto d'onda, che ritarda una
delle due onde.

La sovrapposizione di due onde polarizzate circolarmente una destrogira
e l'altra levogira dà un'onda polarizzata linearmente.

Quando abbiamo una radiazione polarizzata linearmente che attraversa un
mezzo otticamente attivo possiamo avere:

\begin{itemize}
\item
  Il vettore che descrive la massima ampiezza (in uscita) descrive
  un'ellisse, quindi posso definire la grandezza ellitticità:
  arcotangente del rapporto fra semiasse maggiore e semiasse minore.
\item
  L'asse maggiore dell'ellisse viene ruotato (rispetto alla direzione
  della radiazione incidente): rotazione ottica a una sola λ, altrimenti
  ORD (optical rotatory dispersion).
\item
  Luce polarizzata destrogira e levogira vengono assorbite in modo
  diverso: dicroismo circolare. Alla ricombinazione di due componenti
  che risultano (per l'assorbimento diverso) avere ampiezza diversa,
  otteniamo luce polarizzata ellitticamente in uscita.
\item
  Le onde vedono indici di rifrazione diversi, quindi si muovono con
  velocità diverse e questo causa uno spostamento di fase fra le due:
  alla ricombinazione l'asse dell'ellisse è spostato. È la birifrangenza
  circolare, ed è equivalente alla rotazione ottica.
\end{itemize}

La rotazione è dovuta alla differenza dell'indice di rifrazione:

\[\phi = \frac{180L}{\lambda}(n_{L} - n_{R})\]

Dove \(L\) è la lunghezza del cammino ottico e 180 è per convertire da
radianti a gradi.

L'ellitticità è legata alla differenza di assorbimento:

\[\theta = 2.303\ \left( A_{L} - A_{R} \right)\frac{180}{4\pi}\]

Se dobbiamo confrontare campioni che sono in concentrazioni diverse si
fa una normalizzazione per la concentrazione (il fattore 100 è per far
tornare le udm):

\[\left\lbrack \phi \right\rbrack = \frac{100\phi}{\text{CL}}\]

\[\left\lbrack \theta \right\rbrack = \frac{100\theta}{\text{CL}}\ \left\lbrack \frac{}{\text{M\ cm}} \right\rbrack\]

Con il DNA si usa la normalizzazione alla concentrazione (molarità di
fosfati).

Con le proteine si normalizza così:

\[\left\lbrack \theta \right\rbrack = \frac{\theta}{\text{cLN}}\]

Dove N è il numero di residui (aa presenti). Alle volte si normalizza
per \((N - 1)\) che non è il numero di residui ma il numero di legami
peptidici presenti.

La direzione di rotazione della luce ellittica che viene emessa e il suo
grado di ellitticità è determinata da quale componente viene assorbita
maggiormente.

Quello che si verifica sperimentalmente sono entrambi i processi: la
luce esce in polarizzazione ellittica con asse ruotato. Le nostre
misurabili sono ellitticità (info sulla differenza di assorbanza) e
angolo di rotazione ottica (info sugli indici di rifrazione). Entrambi
questi processi avvengono solo se la molecola è asimmetrica nella sua
struttura molecolare, quindi l'attività ottica ci dà informazioni sulle
proprietà strutturali della macromolecola.

Un campione con \(C = 10^{- 4}M,\ L = 1cm,\ \lambda = 300nm\) fornisce
\(\phi = 0.01\) quindi
\(\Delta n = \frac{\text{λϕ}}{180L} = 1.67 \times 10^{- 9}\) ma il
misurabile ci dà
\(\left\lbrack \phi \right\rbrack = 10^{4}\frac{}{\text{M\ cm}}\),
valore misurabilissimo.

Se vedo un'ellitticità di 0.01° ci dà \(\tan\frac{a}{b} \sim 10^{- 4}\),
difficile da misurare ma \(\left\lbrack \theta \right\rbrack = 10^{4}\).

In realtà queste due misurabili sono legate dalle trasformate di
Kronig-Kramers.

Essendo il dicroismo dovuto a una differenza di assorbimento, non posso
osservarlo al di fuori della regione di assorbimento. La forma della
banda del dicroismo sarà la stessa della banda di assorbimento, in senso
positivo o negativo (effetto Cotton).

Lo spettro di ORD può essere ricavato a partire dallo spettro del
dicroismo ed è una specie di derivata di esso, con code molto lunghe.

\textbf{Origine fisica del dicroismo circolare}

Non esiste CD al di fuori della banda di assorbimento, ma non solo: la
probabilità di dicroismo dipende dalla parte immaginaria del prodotto
scalare fra il momento di dipolo elettrico e il momento di dipolo
magnetico. L'effetto del momento di dipolo elettrico è lo spostamento di
carica, quello del momento di dipolo magnetico è la circolazione della
carica. La combinazione dei due produce uno spostamento elicoidale della
carica, ma il momento di dipolo magnetico deve avere una componente
nella direzione del momento di dipolo elettrico.

La circolazione elicoidale di carica è favorita da molecole
asimmetriche.

Essendo il momento di dipolo magnetico una quantità solo immaginaria, il
prodotto scalare sarà solo immaginario quindi osservandone la parte
immaginaria, essa è reale.

\textbf{Strumento }

Lo strumento deve mandare la luce polarizzata circolarmente nei due
modi, alternativamente. Serve una pompa di azoto per rimuovere
l'ossigeno perché, agendo nell'UV, se l'O viene sottoposto a raggi UV
produce ozono, pericoloso per noi e per lo strumento. Inoltre, assorbe
un sacco e quindi darebbe un fondo indesiderato.

La differenza di assorbimento è piccola, quindi bisogna isolare il
contributo del solvente. Ci sono infatti delle λ al di sotto delle quali
il solvente assorbe talmente tanto che il segnale di dicroismo diventa
casuale.

La regione di interesse più interessante è FarUV-NearUV.

Nel Far assorbe il legame peptidico, nel Near assorbono gli aa
aromatici, il gruppo EME..

Le transizioni che danno più contributo sono le
\(n \rightarrow \pi^{*},\ \pi \rightarrow \pi^{*}\) del legame
peptidico.

L'intensità delle transizioni dipende dal valore degli angoli diedri fra
gli amminoacidi, quindi dalla struttura secondaria. Strutture secondarie
diverse danno bande di CD diversi. Ci sono spettri di riferimento.

\includegraphics[width=2.69792in,height=4.08333in]{media/image33.emf}Le
α-eliche hanno tre picchi: uno positivo e due negativi perché fanno
accoppiamento eccitonico: accoppiamento dipolo-dipolo fra molecole allo
stato eccitato che provoca una separazione delle bande energetiche.

Quello che si può fare è anche utilizzare spettri di riferimento per
deconvolvere lo spettro misurato sulla base di tali spettri di
riferimento, per trovare quale sia la proporzione di strutture presenti
nella proteina esaminata.

Il problema è che funziona bene con lunghezze d'onda molto basse,
attorno ai 180nm, che richiedono sottovuoto (non ce la si fa in lab).

\textbf{NearUV}

Assorbono aa aromatici e legami disolfuro.

Il segnale si ha solamente se gli aa sono immobilizzati in una
conformazione asimmetrica.

In questo senso gli spettri ci danno informazioni sulla struttura
terziaria e sulla sua rigidità. Ampiezza e λ delle bande dipendono
fortemente dall'environment elettronico. Lo spettro è molto sensibile
alla denaturazione, perché la proteina si spacchetta e perde rigidità.

Anche gli acidi nucleici sono interessanti al CD. Diverse eliche hanno
spettri di CD diversi. Inducendo un cambiamento strutturale di tipo
dell'elica possiamo osservare spettri diversi.

Quindi gli utilizzi del CD sono: studio di rigidità, ovvero struttura
terziaria (Near) e struttura secondaria (Far), verificare se abbiamo
purificato correttamente una proteina, verificare le mutazioni
(confronto mutante-wild type), verificare se è corretto il folding di un
prodotto biofarmaceutico (per la funzione).

In sintesi:

\begin{itemize}
\item
  Studio della conformazione di una proteina / acido nucleico
\item
  Determinazione di cambiamenti conformazionali dovuti a interazioni di
  molecole asimmetriche (tra proteine, DNA e ligandi)
\item
  Determinazione della termodinamica di folding e unfolding
\item
  Determinazione delle costanti di legame
\item
  Cinetiche di folding e unfolding.
\end{itemize}

Vantaggi:

\begin{itemize}
\item
  Necessita di poco campione
\item
  Non invasivo sul campione
\item
  Osservazione accurata di cambiamenti relativi dovuti a effetti del
  circondario (pH, denaturanti, temperatura)
\end{itemize}

Svantaggi:

\begin{itemize}
\item
  Interferenza con l'assorbimento del solvente nell'UV necessità di
  tamponi molto diluiti e non assorbenti (soprattutto sotto i 200nm)
\item
  Misure affette da vari errori sperimentali
\item
  Accuratezza media del fit circa 10\%
\item
  Spettropolarimetro costoso.
\end{itemize}

\textbf{SPETTROSCOPIA IR}

Riguarda transizioni fra stati che differiscono poco in energia perché
riguarda transizioni fra i livelli vibrazionali. Consideriamo le
molecole con baricentro fisso, escludendo traslazioni, supponendo che i
nuclei si possano muovere solamente lungo la direzione del legame.

Si lavora in numeri d'onda=reciproco lunghezza d'onda.

A diverse zone dell'IR (FarIR, NearIR, intervalli di numeri d'onda)
rispondono diversi legami.

\[E_{\text{molecola}} = E_{\text{elettronica}} + E_{\text{vibrazionale}} + E_{\text{rotazionale}}\]

Vibrazioni molecolari:

\begin{itemize}
\item
  Bending (piegamento)
\item
  Stretching (allungamento)
\end{itemize}

Lo spettro della luce solare è composto in gran parte da radiazione IR
corpo nero e dipendenza dello spettro da T.

Radiazione e spettroscopia IR utili in campo industriale, medico e
scientifico, trasporto dati: sensori di connessione IR, permette di
determinare la conformazione delle molecole.

\textbf{Assorbimento IR}

In una molecola gli atomi sono legati da legame covalente, non fisso,
quindi ci sono oscillazioni di legame, infatti solitamente si parla di
lunghezza media.

L'assorbimento IR provoca cambiamenti vibrazionali nella molecola e la
comparsa di un picco nello spettro. Non tutta la radiazione IR viene
assorbita dalla molecola; ci sono \textbf{regole di selezione}:

\begin{itemize}
\item
  La frequenza applicata deve corrispondere a una delle frequenze
  normali di oscillazione (modi normali)
\item
  Deve esistere un momento di dipolo elettrico, quindi il legame
  covalente deve essere polare, e quindi la molecola deve essere
  asimmetrica
\item
  Ma soprattutto, in seguito all'assorbimento deve essere variazione di
  tale dipolo elettrico.
\end{itemize}

Maggiore è la variazione del momento di dipolo, quindi maggiore è la
variazione di lunghezza del legame (al variare della lunghezza del
legame varia il momento di dipolo perché le cariche si trovano a
distanze diverse), più intenso è il campo em. Se mandiamo radiazione
alla stessa frequenza del campo, le due onde si accoppiano e vengono
assorbite.

Con molecole diverse si hanno gruppi funzionali diversi e quindi
oscillazioni caratteristiche diverse. Essendo altamente specifici, si ha
poca sensibilità all'environment, quindi questo metodo rappresenta un
buon modo per avere un'impronta della molecola.

L'oscillazione si può modellare, usando la legge di Hook:
\(\nu = \frac{1}{2\pi}\sqrt{\frac{K}{\mu}}\),
\(\mu = \frac{m_{1}m_{2}}{m_{1} + m_{2}}\) massa ridotta, \(K\) costante
di forza. Un legame doppio ha \(K\) più elevata quindi avrà oscillazioni
a maggiore frequenza. Questa però è una descrizione grezza.

Una descrizione migliore è quella quantistica, basata
sull'approssimazione di Born-Oppenheimer. Lo spettro energetico di una
molecola è approssimato da una parabola attorno alla posizione di
equilibrio l'hamiltoniana che descrive il movimento è quella di un
oscillatore armonico.

\[H = T + \frac{1}{2}kx^{2}\]

I livelli energetici sono equispaziati proporzionalmente a
\(\text{hν}\).

\[E_{n} = \left( n + \frac{1}{2} \right)\text{hν}\]

Perché si osservi un picco per assorbimento di radiazione, serve una
variazione del momento di dipolo, che è basato sulle funzioni d'onda
prima e dopo l'assorbimento.

\[\left( T + \frac{1}{2}kx^{2} \right)\psi_{n}\left( x \right) = E_{n}\psi_{n}\left( x \right)\]

\[\mu = \mu_{0} + \left( \frac{\text{dμ}}{\text{dx}} \right)_{0}x + ..\]

\[\mu_{\text{trans}} = \left\langle \psi_{n^{*}}\left| \mu_{0} \right|\psi_{n} \right\rangle + \left( \frac{\text{dμ}}{\text{dx}} \right)_{0}\left\langle \psi_{n^{*}}\left| x \right|\psi_{n} \right\rangle\]

Funzioni d'onda ortogonali termine nullo
\(\left\langle \psi_{n^{*}}\left| \mu_{0} \right|\psi_{n} \right\rangle = \mu_{0}\langle\psi_{n^{*}}|\left| \psi_{n} \right\rangle = 0\)

Perché esista dipolo di transizione devo avere

\[\mu_{\text{trans}} \neq 0 \rightarrow \left( \frac{\text{dμ}}{\text{dx}} \right)_{0} \neq 0\ e\ \left\langle \psi_{n^{*}}\left| x \right|\psi_{n} \right\rangle \neq 0\]

\begin{itemize}
\item
  Il termine di dipolo deve variare con la coordinata di vibrazione
  \(x\) perché avvenga la transizione.
\item
  Di tutte le transizioni ce ne sono solo alcune possibili, e sono
  quelle che hanno una differenza di livello energetico di \(\pm 1\)
  (sono possibili solo salti fra livelli vicini).
\end{itemize}

\textbf{Potenziale di Morse}

Questa era un'approssimazione grezza di potenziale di tipo armonico.
Tuttavia tale approssimazione vale soltanto per un piccolo intervallo
attorno alla posizione di equilibrio.

Un'approssimazione migliore è quella al potenziale di Morse:

\[V\left( x \right) = D_{e}\left( 1 - e^{- \alpha x} \right)^{2},\ \alpha = \sqrt{\frac{k}{2D_{e}}}\]

Quindi i livelli energetici in una forma più accurata non dipendono solo
da un termine in \(n\) ma c'è anche un termine in \(n^{2}\):

\[E_{n} = \left( n + \frac{1}{2} \right)h\nu - \left( n + \frac{1}{2} \right)^{2}\text{hνb}\]

dove \(b = \frac{\mathcal{h}a^{2}}{2\text{μω}}\) è la costante di
anarmonicità.

Il nuovo potenziale sarà:

\[V = V_{0} + \left( \frac{\text{dV}}{\text{dx}} \right)_{0}x + \frac{1}{2}\left( \frac{d^{2}V}{dx^{2}} \right)_{0}x^{2}\]

All'equilibrio, a potenziale minimo abbiamo che la derivata prima è
nulla, e se poniamo \(V_{0}\) all'origine dell'asse delle energie:

\[V = \frac{1}{2}\left( \frac{d^{2}V}{dx^{2}} \right)_{0}x^{2}\]

In realtà ci sono dei problemi di anarmonicità che esulano
dall'approssimazione armonica, e per considerarli non possiamo fermarci
allo sviluppo di secondo ordine. I due tipi di anarmonicità sono:

\begin{itemize}
\item
  Anarmonicità meccanica. Abbiamo un termine in più nel potenziale
  (sviluppo terzo ordine) che rappresenta una perturbazione che comporta
  un mescolamento delle funzioni, gli stati del sistema sono una
  combinazione lineare delle oscillazioni, non più solo quelle
  dell'oscillatore. La forza di richiamo (Hook) non è più proporzionale
  allo spostamento, e la frequenza di oscillazione diventa dipendente
  anche dall'ampiezza del movimento che viene fatto, la relazione non è
  più lineare.
\item
  Anarmonicità elettrica. La variazione del momento di dipolo non è più
  proporzionale allo spostamento dei nuclei, ma il momento avrà anche un
  termine in cui si mescolano le funzioni d'onda. In questo caso si
  possono osservare transizioni fra livelli distanti, anche arrivando a
  una differenza di 3 livelli energetici.
\end{itemize}

\textbf{Armoniche superiori}. Si osservano sperimentalmente anche le
transizioni da \(n = 0 \rightarrow 2,\ n = 0 \rightarrow 3\), anche se
sono più deboli della transizione fondamentale perché non sono così
favorite. Se valesse l'approssimazione armonica con livelli energetici
equispaziati, allora avremmo energie e frequenze multiple l'una
dell'altra. Ma col potenziale di Morse c'è un termine in \(n^{2}\) e
confrontando con i valori attesi possiamo stimare l'anarmonicità.

\textbf{Tipi di vibrazioni}

\begin{itemize}
\item
  Fondamentali

  \begin{itemize}
  \item
    Stretching: non cambia l'angolo di legame, ma cambia la lunghezza di
    esso.

    \begin{itemize}
    \item
      Simmetriche. I legami aumentano contemporaneamente la loro
      lunghezza.
    \item
      Non simmetriche. Un legame aumenta in lunghezza, l'altro
      diminuisce.
    \end{itemize}
  \item
    Bending: cambia l'angolo del legame. Energie più basse rispetto alle
    precedenti.

    \begin{itemize}
    \item
      Nel piano

      \begin{itemize}
      \item
        Scissoring: atomi si avvicinano, angoli diminuiscono.
      \item
        Rocking: salutare. Atomi alla stessa distanza, si muovono nella
        stessa direzione.
      \end{itemize}
    \item
      Nel piano della molecola

      \begin{itemize}
      \item
        Twisting
      \item
        Wagging
      \end{itemize}
    \end{itemize}
  \end{itemize}
\item
  Non fondamentali:

  \begin{itemize}
  \item
    Sovratoni: armoniche superiori.
  \item
    Combinazioni di toni: armoniche sferiche, frequenze somma di quelle
    fondamentali.
  \item
    Risonanze di Fermi: misto delle vibrazioni fondamentali e dei
    sovratoni.
  \end{itemize}
\end{itemize}

\begin{quote}
\(n = 0 \rightarrow 1\) transizione fondamentale,
\(n = 0 \rightarrow 2\) sovratoni, \(n = 1 \rightarrow 2\) hot band:
dipendono dalla temperatura, si osservano man mano che vengono occupati
i livelli energetici superiori, quindi all'aumento della T.
\end{quote}

\textbf{Spettrometria IR}

L'obiettivo della spettroscopia FTIR è ottenere lo spettro di
assorbimento di una sostanza, cioè valutare quanto assorba a diverse λ.

Il metodo ordinario usa una sorgente monocromatica e varia continuamente
λ nel range di interesse. Tuttavia, questo procedimento richiede molto
tempo e soprattutto non funziona se è richiesto uno spettro ampio e non
a singole λ, perciò si è reso necessario l'utilizzo di sorgenti a banda
larga, che contengano tutte le λ di interesse.

Per distinguere poi l'assorbimento delle diverse λ da parte del campione
si utilizza un interferometro di Michelson, che permette la scansione
grazie a uno specchio mobile e ad uno specchio fisso. Lo specchio mobile
induce una differenza di cammino ottico che origina un'interferenza
costruttiva o distruttiva con il raggio riflesso dallo specchio fisso,
creando una figura di interferenza. In questo modo si crea
l'interferogramma (intensità rispetto alla posizione dello specchio).
Per ogni posizione dello specchio avremo un interferogramma diverso, e
tutti verranno sommati in un inteferogramma-somma che poi viene mandato
sul campione. La sua forma cambia perché il campione assorbe. La
trasformata di Fourier a questo punto permette di convertire
l'interferogramma in spettro (intensità in funzione delle frequenze).

Anni '40. Spettrometri a doppio raggio: la radiazione viene separata in
due e attraversa campione e riferimento.

Monocromatori a prisma (obsoleti) monocromatori a reticolo.

Problemi con questa strumentazione sostituzione con strumenti che
operano attraverso la trasformata di Fourier. Vantaggio: l'informazione
su tutte le λ viene raccolta simultaneamente o comunque in un breve
intervallo di tempo.

Componenti interferometro di Michelson: beam splitter, specchio fisso,
specchio in movimento.

Tutte le λ vengono scansionate in un ``ciclo'' dello specchio (max 2
secondi).

Interferometro serve per formare l'interferogramma, che viene poi
analizzato tramite la FT una volta passato sul campione ho lo spettro
per tutte le frequenze.

Ovviamente poi bisogna normalizzare per le condizioni ambientali,
l'efficienza della lampada etc quindi viene passato un laser He-Ne, che
fa da fascio di riferimento per ricostruire l'interferogramma ideale
dell'oggetto che non assorbe.

L'interferogramma misurato è una sovrapposizione spettrale delle diverse
onde; l'ampiezza che pesa le onde è legata alla probabilità di
assorbimento, quindi avrò un'ampiezza di partenza che verrà modificata a
seconda di quanto assorbe il campione.

Con un fascio monocromatico è tutto più semplice, perché c'è una sola
funzione seno.

Con un fascio policromatico invece c'è sovrapposizione di praticamente
infinite onde.

\textbf{Componenti FT-IR}

\begin{itemize}
\item
  Sorgente. Approssimazione dell'ideale corpo nero. Si usano materiali
  solidi riscaldati da corrente elettrica che emettano radiazione IR
  nell'intervallo desiderato. Per λ maggiori si usa globar (cilindro
  carburo di silicio), per λ più piccole filamento di Nerst. Noi in lab
  abbiamo un tocco di ceramica.
\item
  Interferometro Michelson
\item
  Campione. Portacampione: deve essere trasparente all'IR e deve
  resistere a solventi polari ATR: cella a riflessione totale interna.
  La radiazione arriva da sotto e fa riflessione totale (il materiale
  della cella è progettato apposta per farla), si forma un'onda
  evanescente al confine fra il portacampione e il campione. Tale onda
  penetra poco, ma abbastanza per interagire con i primi strati
  atomici/molecolari del campione, e può essere assorbita. Riusciamo
  comunque a fare l'esperimento ma dobbiamo avere concentrazioni
  altissime (elevato numero di centri che assorbono) perché altrimenti
  non ho segnale visibile riuscendo a penetrare solo nei primi strati di
  soluzione.
\item
  Rivelatore, che deve essere in grado di misurare interferogrammi. Non
  può essere semplicemente un fotomoltiplicatore.

  \begin{itemize}
  \item
    Termici, che misurano l'energia radiante per un vasto range di λ
  \item
    Selettivi su λ
  \end{itemize}

  \begin{itemize}
  \item
    Bolometri
  \item
    Termistori e termocoppie
  \item
    Rivelatori piroelettrici: c'è polarizzazione che dipende dalla T.
    Mandano segnale elettrico se c'è differenza di T su facce opposte.
    Sotto la T di Curie (49°C) hanno polarizzazione spontanea, che varia
    al variare di T.
  \end{itemize}
\end{itemize}

\begin{quote}
Interferogramma. Una frequenza una riga, due onde a frequenze diverse
che interferiscono due righe etc fino allo spettro continuo,
corrispondente a una banda di frequenze.
\end{quote}

\begin{itemize}
\item
  Computer
\end{itemize}

\textbf{Vantaggi FT-IR}:

\begin{itemize}
\item
  Veloce
\item
  Sensibile
\item
  Tutte le frequenze simultaneamente
\item
  Diversi composti contemporaneamente. Autocalibrazione con luce laser.
\item
  Etc
\end{itemize}

\textbf{Spettri IR}

Si registrano in \% di trasmittanza. A 100\%T non c'è assorbimento no
picchi.

Regioni:

\begin{itemize}
\item
  Regione gruppi funzionali. 4000-1300\(cm^{- 1}\).
\item
  Regione impronte digitali. 1300-650\(cm^{- 1}\).
\end{itemize}

Dobbiamo sapere quanti picchi ci aspettiamo, dalla struttura della
molecola.

Una molecola non lineare con \(n\) atomi possiede \(3n - 6\) modi
vibrazionali fondamentali (\(3n - 5\) se la molecola è lineare).

\textbf{Proteine e IR}

Si osservano delle bande, che corrispondono a diversi modi di vibrazione
del legame peptidico. Solo il legame peptidico genera 9 bande: amide
A,B,1,2,3,4,5,6,7.

Amide I: stretching del C=O

Amide II: bending di N--H

Caratterizzate da intervalli di numeri d'onda in cui si osservano tali
fenomeni.

Nell'amide I e II si vedono anche i contribuiti dei diversi tipi di
struttura secondaria.

Non avendo quasi mai una singola struttura secondaria, osserveremo una
banda enorme con varie spallette che caratterizzano le varie bande fit
multigaussiana per ``dividere'' tali picchi. Dalla posizione dei picchi
e dalla loro altezza relativa siamo in grado di dedurre la struttura
secondaria.

Per capire quanti picchi ci sono è utile fare la derivata seconda.

\textbf{Spettroscopia Raman}. A differenza della spettroscopia IR, è un
processo fuori risonanza. Di manda radiazione con frequenza diversa da
una di quelle caratteristiche. Parte di quell'energia viene usata per
cambiare il livello vibrazionale a cui ricade la molecola.

In IR induco una transizione fra stati vibrazionali per assorbimento, in
Raman ho dispersione della luce inelastica. Tipicamente si considerano
processi di scattering quasi elastici, ma in questo caso abbiamo
scattering inelastico.

\textbf{SCATTERING}

Berne \& Pecora

\begin{itemize}
\item
  Statico. Misura della media temporale dell'intensità della luce
  diffusa al variare di:

  \begin{itemize}
  \item
    Angolo di diffusione, quindi vettore d'onda scambiato Q
  \item
    Concentrazione/pH
  \end{itemize}
\item
  Dinamico. Si studiano le fluttuazioni di intensità che arrivano in una
  certa posizione quindi si studia la dinamica. Spettroscopia di
  fotocorrelazione.
\end{itemize}

Concetto fondamentale. L'intensità di luce diffusa nel volume di
eccitazione è proporzionale alla concentrazione di scatteratori nel
volume e al peso molecolare.

\[\frac{\mathbf{i}_{\mathbf{S}}\left( \mathbf{t} \right)}{\mathbf{V}_{\mathbf{\text{eccitazione}}}}\mathbf{\propto n}\mathbf{M}^{\mathbf{2}}\]

\(i_{s}\) intensità di luce diffusa, \(n\) concentrazione numerica, M
peso molecolare.

Questo vale in approssimazione di particelle di dimensione molto minore
di λ e in generale non interagenti.

Questo permette di distinguere bene fra monomeri \((n = 2,\ M = 1)\) e
dimeri \((n = 1,M = 2)\).

Tipicamente è lo scattering alla base del selettore di dimensioni.

\textbf{Strumento}

\begin{itemize}
\item
  Sorgente laser monocromatica che incide sul
\item
  Portacampione immerso in un bagno che fa da termostato
\item
  Serie di ottiche per selezionare il volume di eccitazione (iride,
  fenditure)
\item
  Fototubo rivelatore
\item
  Analizzatore che digitalizza il segnale e lo invia all'elaboratore.
\end{itemize}

Le osservazioni si fanno su un piano orizzontale.

\textbf{Metodi}

\begin{itemize}
\item
  Omodino: solo luce diffusa arriva al catodo
\item
  Eterodino: radiazioni miste (diffusa + eccitante) arrivano al catodo
\end{itemize}

\textbf{Tipi di scattering}

\begin{itemize}
\item
  Raggi X e neutroni a basso angolo forma delle molecole
\item
  Di micro-onde
\item
  Di luce stellare
\item
  Di molecole peso e diffusione molecolare
\end{itemize}

\textbf{Teoria classica}

Il campo di eccitazione è polarizzato verticalmente e esercita una forza
sulle cariche nel volume di scattering, che vengono accelerate e
irraggiano.

Il campo diffuso è la sovrapposizione dei campi diffusi dalle varie
regioni: campo speckle. Se gli oggetti nel campione diffondono cambia la
figura di interferenza che si forma e si osserva il suo brulicare di
quello che viene chiamato campo speckle.

\includegraphics[width=1.96875in,height=1.96875in]{media/image34.jpeg}

Siccome il campo è polarizzato eccitiamo di più le cariche che hanno
momento di dipolo parallelo alla direzione di polarizzazione del campo
selezione.

Se tutto il volume di eccitazione lo dividiamo in sottoregioni in cui
c'è lo stesso campo incidente, il campo diffuso è la sovrapposizione dei
campi diffusi dalle varie sottoregioni (campo speckle: serie di puntini
più o meno luminosi). Se l'oggetto che scattera è immobile, il campo
speckle sarà immobile. Se invece gli oggetti che scatterano sono in
movimento allora il campo speckle brulica.

Se le molecole sono otticamente identiche lo scattering avviene solo in
avanti (le onde secondarie con stessa ampiezza e fase/posizione diversa
si annullano a coppie quindi rimane solo il contributo in avanti).

I cristalli si studiano facendo scattering a piccolo angolo, perché
facendo scattering lineare è difficile perché dovremmo trovare poi un
modo per cancellare completamente la radiazione trasmessa, che dà un
contributo notevole.

Se le molecole non sono otticamente identiche si producono onde
secondarie con diversa ampiezza, quindi scatterano in tutte le
direzioni.

Einstein dopotutto diceva che lo scattering è il risultato di
fluttuazioni della costante dielettrica del mezzo.

\textbf{Il ritardo di fase}

Consideriamo due punti \(P_{0},P_{1}\) all'interno del campione.
Supponiamo che il punto \(P_{1}\) sia più vicino. Il campo (vettore
d'onda) che arriva a \(P_{0}\) arriverà quindi con un certo ritardo,
dovuto alla differenza di cammino \(d_{0}\) del vettore d'onda, quindi
avrà uno sfasamento di
\(\varphi_{0} = - \left| k_{\text{in}} \right|d_{0}\). Anche i campi
diffusi saranno sfasati: il campo che lascia \(P_{1}\) è anticipato di
un fattore \(\varphi_{1}\) rispetto a quello che proviene da \(P_{0}\) a
causa della differenza di cammino
\(d_{1}:\ \varphi_{1} = - \left| k_{\text{scat}} \right|d_{1}\).

\includegraphics[width=4.05208in,height=2.35202in]{media/image35.emf}

Lo sfasamento complessivo sarà la somma dei due contributi:
\(\Delta\varphi = k_{0}\left( d_{0} + d_{1} \right)\).

Quello che interessa a noi è lo scattering \textbf{elastico}: luce
incidente e luce scatterata hanno lo stesso modulo del vettore d'onda.
L'energia scambiata \(\Delta E = \mathcal{h}\Delta\omega\) è quasi
nulla.

Vettore d'onda scambiato invece non è nullo perché i vettori d'onda sono
vettori -.- e quindi anche se non varia il modulo sono comunque diversi.

\[k_{\text{in}} = k_{\text{scat}} = \frac{2\pi n}{\lambda} = k_{0}\]

Non cambia la λ della radiazione osservata rispetto a quella incidente.
I vettori d'onda scambiati sono uguali in modulo ma la loro differenza
\(\overrightarrow{q}\) non è nulla.

Il ritardo di fase c'è in ogni caso ed è dovuto al fatto che stiamo
prendendo i contributi da punti diversi.

\[\overrightarrow{q} = {\overrightarrow{k}}_{\text{in}} - {\overrightarrow{k}}_{\text{scat}}\]

\[q = \frac{4\pi n}{\lambda}\sin\frac{\theta}{2}\]

\[q^{2} = 2k_{0}^{2} - 2k_{0}^{2}\cos\theta \rightarrow q = 2k_{0}\sin\frac{\theta}{2}\]

\(q\) è proporzionale alla lunghezza d'onda. L'approssimazione da fare è
che le cose che vogliamo osservare devono essere più piccole della
lunghezza d'onda degli strumenti che stiamo utilizzando.

Se invece consideriamo scattering Raman non possiamo approssimare
\(\left| k_{\text{in}} \right| \sim \left| k_{\text{out}} \right|\).

Definire un angolo di scattering significa definire un valore del
vettore d'onda scambiato.

Siccome il vettore d'onda scambiato dipende dall'inverso di λ della
radiazione incidente significa che, utilizzando una certa sorgente con
una certa λ, individuiamo un intervallo di vettori d'onda scambiati che
possiamo andare ad indagare. Quindi ci sarà anche un certo intervallo di
dimensioni delle molecole che potremo andare ad osservare. Utilizzando
raggi X o luce potremo osservare particelle di dimensioni diverse.
Perché non c'è un solo valore di \(q\)? perché comunque poi q dipende
dall'angolo \(\theta\), che genera appunto l'intervallo.

Il modulo del vettore d'onda scambiato quindi dipende dalla dimensione
degli oggetti che stiamo osservando.

Con una singola particella l'intensità del campo diffuso dipende da:

\begin{itemize}
\item
  Distanza R a cui metto il rivelatore (inverso) onda piana
\item
  Intensità del campo incidente
\item
  Termine di fase legato ai parametri del fascio incidente
\item
  λ di scattering (quadrato)
\item
  Come interagisco con la molecola. Quindi dalla polarizzabilità, in
  particolare dalla componente del tensore di polarizzabilità relativa
  alle direzioni di incidenza e di scattering. La polarizzabilità dice
  come il dipolo elettrico della molecola risponde al campo elettrico
  esterno. Infatti, la molecola prediligerà alcune direzioni del campo
  elettrico a seconda del suo momento di dipolo. Dà un informazione di
  ampiezza, che è funzione del tempo. Questo perché la molecola nel
  tempo ruota e vibra e modifica tale termine.
\item
  Altro termine di fase, legato alla posizione della molecola quando
  osservo. Siccome la molecola trasla, la sua posizione varia nel tempo,
  assieme a questo termine di fase.
\end{itemize}

\[{\overrightarrow{E}}_{\text{scat}}\left( R,t \right) \sim \frac{E_{\text{in}}k_{\text{scat}}^{2}e^{ik_{0}R - i\omega t}}{R}\underset{\begin{matrix}
\mathbf{\text{ampiezza}} \\
\text{varia\ nel\ tempo} \\
perche\ la\ molecola\  \\
\text{ruota\ e\ vibra} \\
\end{matrix}}{}\underset{\begin{matrix}
\mathbf{\text{fase}} \\
\text{varia\ nel\ tempo} \\
perche\ la\ molecola \\
\text{trasla} \\
\end{matrix}}{}\]

\(\alpha_{in,scat}\) è la componente del tensore polarizzabilità della
molecola nelle direzioni \(\text{in\ }\) e \(\text{scat}\):

\[\mu\left( t \right) = \alpha \cdot E\left( t \right)\]

Finché abbiamo una sfera, che ruoti o vibri non è importante, il tensore
di polarizzabilità è una matrice identità.

Il campo diffuso che osservo ha andamento temporale
\(\alpha\left( t \right)e^{i\overrightarrow{q} \cdot \overrightarrow{r}(t)}\),
da cui posso ottenere, tramite media, informazioni statiche su
concentrazioni, masse, oppure lo osservo nel tempo e ottengo
informazioni sulla dinamica, rotazioni e vibrazioni.

Per un certo numero di molecole, purché debolmente accoppiate si fa una
sommatoria, fatta solo su molecole contenute nel volume di eccitazione:

\[{\overrightarrow{E}}_{\text{scat}}\left( R,t \right) \sim \frac{E_{\text{in}}k_{\text{scat}}^{2}e^{ik_{0}R - i\omega t}}{R}\sum_{{j \in V}_{\text{exc}}}^{}{\alpha_{in,scat}^{j}\left( t \right)e^{i\overrightarrow{q} \cdot {\overrightarrow{r}}_{j}\left( t \right)}}\]

Si suppone che le molecole siano non troppo concentrate, il movimento
sia di tipo casuale e che la polarizzabilità non dipenda dal movimento
delle molecole.

Tuttavia il rivelatore misura l'intensità, che è il quadrato
dell'energia, quindi dipende dalla quarta potenza del vettore d'onda:

\[{\overrightarrow{I}}_{\text{scat}}\left( R \right) = \left| {\overrightarrow{E}}_{\text{scat}}\left( R \right) \right|^{2} \sim \frac{I_{\text{in}}k_{\text{scat}}^{4}}{R^{2}}\]

Cielo e mare sono blu perché la lunghezza d'onda più corta è quella che
viene diffusa di più.

In realtà la polarità dipende da due termini:

\begin{itemize}
\item
  Un termine rotazionale. Approssimazione quasi elastica, trascurabili
  variazioni di frequenza, solo rotazioni e traslazioni, dovuto a
  variazioni di fase scattering Rayleigh
\item
  Un termine vibrazionale. Piccole variazioni di frequenza nel range di
  spettroscopia IR (ma va?!). Ci sono vibrazioni caratteristiche di
  Stokes o anti-Stokes (frequenza maggiore, λ minore). Perché sia
  osservabile dev'essere che la variazione del tensore di
  polarizzabilità rispetto al vettore d'onda scambiato deve essere non
  nulla scattering Raman.
\end{itemize}

\textbf{Dimensioni} \(\mathbf{\sim \lambda}\)

Se le dimensioni degli oggetti di interesse \textbf{non} è minore della
λ incidente, lo scattering permette di avere info sulla forma
dell'oggetto tramite:

\begin{itemize}
\item
  Fattore di forma.
\item
  Fattore di struttura (tipico nei cristalli, a lungo raggio, e nei
  liquidi, a corto raggio).
\end{itemize}

Al rivelatore osserviamo la sovrapposizione di campi provenienti da
punti diversi della molecola.

\[{\overrightarrow{E}}_{\text{scat}}\left( R,t \right) \sim \frac{E_{\text{in}}}{R}\sum_{k = 1}^{N}e^{i\overrightarrow{q} \cdot {\overrightarrow{r}}_{k}\left( t \right)}\]

La sommatoria è fatta sui monomeri costituenti la proteina. La
sommatoria equivale a fare l'integrale, dove i primi termini sono il
volume:

\[\int_{}^{}{d\overrightarrow{r}\rho\left( \overrightarrow{r} \right)e^{i\overrightarrow{q} \cdot \overrightarrow{r}(t)}}\]

La quantità rapporto fra luce diffusa e luce incidente ci permette di
calcolare:

\[P\left( Q \right) = \frac{I_{\text{scat}}\left( Q \right)}{I_{0}} = \left| \int_{}^{}{\left\langle \rho\left( r \right) \right\rangle e^{i\overrightarrow{q} \cdot \overrightarrow{r}}d\overrightarrow{r}} \right|^{2}\]

Il fattore di forma è
\(\left\langle \rho\left( r \right) \right\rangle\).

Esempio. Dimerizzazione della lactoglobulina.

Lo scattering usato per questo scopo è quello statico si scopre quando
avviene la dimerizzazione perché varia l'intensità di scattering +
variazione del fattore di forma.

\textbf{Interazione fra molecole}

L'altra approssimazione era la non-interazione fra le molecole: non c'è
più linearità dell'intensità di scattering nella concentrazione, ma c'è
un termine quadratico in più, effetto delle interazioni, dovuto alla
carica superficiale (dipende da pH e da forza ionica):

\[\frac{i_{\text{scat}}\left( t \right)}{V_{\text{ecc}}} = CM + \alpha C^{2}\]

Si osserva infatti che l'intensità in funzione di C non è lineare.

Forza ionica

\[\mu = \frac{1}{2}\sum_{\text{ioni}}^{}{z_{i}^{2}C_{i}}\]

L'effetto degli ioni in soluzione è di fare uno schermo attorno alle
parti cariche della proteina. Le interazioni a lungo raggio non saranno
più così efficaci. Quindi fanno decadere il campo elettrostatico secondo
un parametro di screening \(\chi\), più velocemente rispetto al modello
Coulombiano:

\[E \sim \frac{e^{- \chi r}}{r}\]

\textbf{FUNZIONE DI AUTOCORRELAZIONE}

Del segnale studiamo solo le parti interessanti: le fluttuazioni di
intensità. Sono descritte dalla funzione di correlazione: si prende il
valore del segnale iniziale e quello dopo un certo \(\tau\) di ritardo,
poi si confrontano. Quindi la funzione di correlazione dice quanto
rapidamente cambia il segnale.

La funzione di correlazione per il campo elettrico è

\[g^{1}\left( \tau \right) = \frac{\left\langle E_{\text{scat}}\left( t \right)E_{\text{scat}}^{*}\left( t + \tau \right) \right\rangle}{\left\langle E_{\text{scat}}^{2}\left( t \right) \right\rangle}\]

Il campo scatterato è quello diffuso dalle molecole nelle diverse
posizioni. Il campo diffuso da un singolo oggetto dipendeva da un
termine di fase:

\[E_{\text{scat}}\left( t \right)E_{\text{scat}}^{*}\left( t + \delta t \right) \propto e^{i\overrightarrow{Q} \cdot \Delta\overrightarrow{r}\left( t \right)}\]

Si ha il massimo sfasamento quando
\(Q^{2}\left| \Delta r\left( t \right) \right|^{2} = \tau^{2}\).

Il modello di Einstein per la descrizione del moto browniano dice che lo
spostamento quadratico medio è legato al coefficiente di diffusione
browniano:

\[\left| \Delta r\left( t \right) \right|^{2} = \ 6D\ \delta t\]

Possiamo quindi definire un tempo caratteristico associato a un oggetto
che diffonde con moto browniano:

\[\delta t = \frac{1}{DQ^{2}}\]

Dimensionalmente torna.

Se osserviamo la funzione di correlazione a tempi maggiori dei tempi
delle fluttuazioni non va bene, solo in caso contrario avremo un picco
interessante. I due campi osservati non avranno alcuna correlazione
reciproca. Si ha correlazione solamente per tempi
\(\tau \ll tempo\ delle\ fluttuazioni\).

Dal punto di vista fisico è interessante il fatto che la media temporale
dell'ampiezza della funzione di scattering al tempo 0 è maggiore della
stessa media fatta fra il tempo 0 e un certo tempo \(\tau\).

\[\left\langle E_{\text{scat}}^{2}\left( 0 \right) \right\rangle \geq \left\langle E_{\text{scat}}\left( 0 \right)E_{\text{scat}}\left( \tau \right) \right\rangle\]

La correlazione va da un massimo (\(\tau = 0\)) e poi decade. Osservando
il segnale dopo intervalli di tempo più lunghi si perde correlazione,
perché stiamo confrontando valori di segnale molto distanti nel tempo.

Possiamo definire un tempo caratteristico, il tempo di correlazione, che
è il tempo necessario perché la funzione di correlazione decada.

\[\tau_{C} = \int_{0}^{\infty}{\frac{\left\langle \delta E_{\text{scat}}\left( 0 \right)\delta E_{\text{scat}}\left( \tau \right) \right\rangle}{\left\langle \delta E_{\text{scat}}^{2}\left( t \right) \right\rangle}\text{dτ}}\]

\[\delta E_{\text{scat}}\left( t \right) = E_{\text{scat}}\left( t \right) - \left\langle E_{\text{scat}}\left( t \right) \right\rangle\]

Il correlatore in lab normalizza i valori (divide per il quadrato del
valore medio e toglie 1) in modo tale che la funzione di correlazione
vada a zero.

Le fluttuazioni del campo sono fluttuazioni di fase, che inducono
fluttuazioni di intensità.

Il campo diffuso ha distribuzione gaussiana a media nulla (è la somma di
variabili indipendenti casuali), quindi posso calcolare la funzione di
correlazione dell'intensità:

\[g^{2}\left( \tau \right) = \frac{\langle I_{\text{scat}}\left( t \right)I_{\text{scat}}(t + \tau)\rangle}{\left\langle I_{\text{scat}}^{2}\left( t \right) \right\rangle}\]

Il Teorema di Bloch-Siegert ci dice che
\(g^{2}\left( \tau \right) = 1 + \left| g^{1}\left( \tau \right) \right|^{2}\),
dove \(g^{2}\) è la funzione di correlazione dell'intensità, che
calcoliamo, \(g^{1}\) quella del campo, predetta/calcolata su un modello
creato per descrivere tali oggetti.

Ricordando i parametri da cui dipendeva il campo (fase, ampiezza), la
funzione di correlazione del campo sarà il prodotto di due termini, uno
di correlazione dell'ampiezza e uno di correlazione della fase:

\[g^{1}\left( \tau \right) = e^{\text{iωt}}C_{\alpha}\left( \tau \right)C_{\phi}(\tau)\]

L'ampiezza cambia se ci sono rotazioni o vibrazioni, la fase cambia se
c'è spostamento (informazioni traslazionali).

Se le particelle sono molto piccole rispetto al reciproco del vettore
d'onda scambiato, la funzione di correlazione è dominata dal termine di
fase informazioni sulle traslazioni.

\(C_{\phi}(\tau)\) è l'integrale della soluzione dell'equazione di
diffusione di Fick, dipendente dal coefficiente di diffusione
traslazionale D. L'equazione di Fick ci dà informazioni sulla dinamica.
La soluzione dell'integrale è del tipo \(e^{- Q^{2}\text{Dτ}}\).

Il fattore 2 entra perché \(g^{2}\) è al quadrato:

\[g^{2}\left( \tau \right) = 1 + \left| C_{\alpha}\left( \tau \right) \right|^{2} = e^{- 2DQ^{2}\tau}\]

Con oggetti molto estesi abbiamo che
\(\left| C_{\phi}\left( \tau \right) \right| \neq 1\).

Imponendo il modello di Einsten, l'approssimazione di Stokes e le
condizioni al contorno otteniamo che

\[D = \frac{k_{B}T}{6\pi\eta R}\]

\textbf{Polidispersità}. Presenza di più oggetti di dimensioni diverse
in soluzione. Ci sarà una distribuzione di raggi. La funzione di
correlazione sarà più complessa.

Avendo due oggetti in soluzione bisognerà tenere conto di due andamenti
esponenziali perché la \(C_{\phi}\) è data dalla somma di due termini.
Questo fornisce due termini nel campo, che messi al quadrato producono 3
termini nell'intensità.

Questo problema si risolve utilizzando un algoritmo di massima entropia.
Le proteine singole sono quelle che hanno raggio più piccolo, gli
aggregati hanno raggio maggiore. Il programma restituisce curve con
colori diversi che corrispondono a oggetti diversi + aggregati di varie
dimensioni.

\includegraphics[width=2.32292in,height=2.27083in]{media/image36.emf}

L'altro metodo è quello dei cumulanti, in cui si studia la funzione di
correlazione del campo. Il primo cumulante è una somma pesata dei vari
coefficienti di diffusione pesati su massa e concentrazione. Il secondo
cumulante è legato alla varianza. Il valore relativo di questi due dà
una misura della polidispersità (fattore di qualità). Se tale fattore è
\textless{}0.02 il campione è monodisperso.

Scattering:

\begin{itemize}
\item
  Polarizzato: eccitazione e rivelazione allo stesso angolo
\item
  Depolarizzato: la molecola ruota e perde memoria della polarizzazione
  eccitante. Il segnale è inferiore a quello polarizzato.
\end{itemize}

\textbf{Forme molecole}

Molecole di forme diverse presentano coefficienti di diffusione diversi

\begin{longtable}[c]{@{}lll@{}}
\toprule
Forma & Coefficiente traslazionale & Coefficiente
rotazionale\tabularnewline
\midrule
\endhead
Sferica & \(D = \frac{k_{B}T}{6\pi\eta R}\) &
\(\theta = \frac{k_{B}T}{8\pi\eta R^{3}}\)\tabularnewline
Rod (L, d) & \(D = k_{B}T\ln\frac{\frac{L}{d}}{3\pi\eta L}\) &
\(\theta = 3k_{B}T\ln\frac{\frac{L}{d}}{\text{πη}L^{3}}\)\tabularnewline
Ellissoidi (a\textgreater{}b) &
\(D = \frac{k_{B}T}{6\pi\eta R}f\left( \rho \right)\) &
\(\theta = \frac{k_{B}T}{8\pi\eta R^{3}}\text{funz}\left( f\left( \rho \right) \right)\)\tabularnewline
\bottomrule
\end{longtable}

\(f\left( \rho \right)\) dipende da se sono prolati o oblati.

\textbf{NANOPARTICELLE METALLICHE}

Lo scattering ci permette di ottenere informazioni sui coefficienti di
diffusione.

La coppa di Licurgo conteneva 40ppm di oro e 300ppm di argento e questo
le fa assumere colori diversi a seconda che sia illuminata da fuori o da
dentro. Una delle prime applicazioni era nel Medioevo come pigmento del
colore rubino nelle vetrate colorate.

Dimensioni: fra le proteine semplici e gli agglomerati più grandi,
quindi sopra gli amminoacidi e sotto le dimensioni cellulari.

Utili perché cambiano proprietà elettriche, ottiche, termiche passando
da dimensioni di aggregati grossi (bulk) a piccoli. Questo perché cambia
notevolmente il rapporto superficie-volume.

L'oro è preferibile perché è inerte e resiste all'ossidazione e
ciononostante è molto reattivo ai gruppi tiolici, quindi possiamo
funzionalizzare le particelle con oggetti fluorescenti, con anticorpi.
Per eliminarle bisogna usare l'acqua regia, perché sono molto sticky
alla plastica e al vetro.

Presentano aumento di proprietà radiative.

Usate per scopi biologici come agenti di contrasto per imaging, rilascio
di farmaci, terapia fototermica (assorbono radiazione che rilasciano
come calore).

Esempi. La temperatura di melting cambia di 400°C da 20nm a 5nm e di 50°
da bulk a 20nm.

Il primo a sintetizzarle fu Paracelso, Faraday ha per primo studiato le
sospensioni di nanoparticelle, Mie ha scritto per primo un trattato
sulle proprietà delle particelle. Successivamente la sintesi è stata
migliorata e semplificata.

Caratteristica più interessante: \textbf{risonanza plasmonica
superficiale}.

La differenza principale fra metalli (conduttori) e semiconduttori è la
diversa distanza fra la banda di conduzione e la banda di valenza: nei
metalli è piccola o inesistente (sovrapposizioni). Nei metalli c'è anche
la presenza delle zone di Brillouin: in base alle geometrie i livelli
energetici non sono costanti ma cambiano a seconda della posizione
all'interno dei cristalli o delle strutture metalliche.

Sono risonanze elettromagnetiche (a lunghezze d'onda particolari) che
inducono oscillazioni coerenti collettive degli elettroni di conduzione
attorno alla loro posizione di equilibrio. Modello di metallo: una
struttura di bulk circondata da elettroni liberi, in grado di muoversi
nelle zone di Brillouin o di passare fra le bande di conduzione e di
valenza. Alcune λ particolari possono provocare, in presenza di campo
magnetico eccitante, oscillazione degli elettroni attorno alla loro
posizione di equilibrio. Avviene anche un'analoga oscillazione della
parte ionica positiva, che però è molto più piccolo, quindi generalmente
è trascurabile e l'oscillazione è perlopiù determinata dal moto degli
elettroni. Quindi di crea un dipolo oscillante.

Non è un fenomeno che avviene a tutte le λ (come ad esempio lo
scattering), ma solo alle λ dei plasmoni. Qui la sezione d'urto ottica
anziché coincidere con la sezione d'urto geometrica, è molto più grande:
si crea un accumulo di linee di campo, quindi un aumento di intensità
del campo elettrico. Quindi aumento delle oscillazioni.

Il valore delle λ utili dipende da:

\begin{itemize}
\item
  Dimensione nanoparticelle
\item
  Forma delle particelle
\item
  Composizione chimica
\item
  Effetti di polarizzazione in nanoparticelle asimmetriche (effetti di
  orientazione)
\item
  Ambiente circostante.
\end{itemize}

\includegraphics[width=2.80764in,height=1.86458in]{media/image37.emf}

I picchi di assorbimento plasmonico sono localizzati a cavallo della λ
caratteristica e permettono di valutare il diametro delle particelle.
Nel range 2-100nm di dimensione la banda di assorbimento sta nel
visibile, quindi hanno colori diversi a seconda della dimensione. Hanno
coefficienti di estinzione particolarmente elevati (1mln di volte quelli
di un cromoforo classico) perché ci sono molti elettroni coinvolti.

Le studiamo perché sono molto fotostabili e hanno una rapida ed
efficiente conversione della luce in calore (riscaldano l'ambiente
circostante).

Apoptosi: morte cellulare programmata (42°). Necrosi: morte cellulare
che rischia di estendersi alle cellule vicine (temperature più alte).

Che origine hanno le risonanze plasmoniche? Fenomeni di assorbimento e
scattering di luce possono essere spiegati dalle equazioni di Maxwell,
con opportune condizioni al contorno.

Supponiamo di avere una sorgente armonica che incide su una particella
di un certo raggio. La particella irraggia e l'interferenza fra campo
creato dalla particella e campo sorgente crea un campo esterno. C'è una
discontinuità dei valori di permeabilità magnetica e elettrica che detta
le condizioni al contorno.

Mie ha risolto il sistema di equazioni di Maxwell per le particelle
sferiche. Il punto cruciale della teoria di Mie è che la sezione d'urto
di estinzione (legato al coefficiente di estinzione molare quindi allo
spettro sperimentale) dell'elettrone dipende da un contributo di
assorbimento e uno di scattering. Ciò permette di ottenere informazioni
sulla sezione d'urto di assorbimento tramite differenza fra quella di
estinzione e quella di scattering.

Le sezioni d'urto hanno espressioni complicate dipendenti da funzioni di
Riccati-Bessel, che a loro volta dipendono da un parametro dimensionale
\(x\), dall'ordine di multipolo (per la sommatoria) e dal rapporto fra
la costante dielettrica delle particelle e del mezzo.

Se il raggio è \(R \ll \lambda\) si ha \(x \ll 1\) quindi regime
quasi-statico e ci si ferma all'ordine di dipolo. Questo regime ci
permette di approssimare il campo a costante all'interno della
particella.

Se invece \(R \sim \lambda\) si ha \(x \sim 1\) quindi regime dinamico e
multipolo.

\textbf{Modello di Drude-Sommerfeld}

Mentre la costante dielettrica del metallo è nota, abbiamo bisogno di un
modello che ci indichi la costante dielettrica della nanoparticella, che
è diversa da quella del metallo, perché nel passaggio da dimensioni
macroscopiche a microscopiche le proprietà della particella cambiano
notevolmente (questione del rapporto superficie-volume).

Per descrivere la nanoparticella si usa il modello di Drude-Sommerfeld:
gli elettroni sono oscillatori armonici che, oscillando per azione del
campo elettrico, inducono collisioni degli elettroni liberi con dei
centri di collisione (ioni, fononi, difetti..) con un certo rate
\(\gamma_{0} = 1/\tau\).

La costante dielettrica della particella dipende da un termine dovuto
agli elettroni legati e da diversi termini dovuti agli elettroni di
conduzione (alla loro densità e massa efficace), oltre che alla
frequenza di plasma.

Fononi: vibrazioni della parte atomica dei metalli. Plasmoni: vibrazioni
degli elettroni dei metalli.

Il modello fallisce ad alte energie dei fotoni. Se diamo energie molto
alte (\textgreater{}2eV) ci potrebbero essere eccitazioni fra bande più
profonde e banda di conduzione (modello interbanda). E poi, il modello
suggerisce una banda di conduzione di forma parabolica rispetto alla
coordinata di Brillouin, ma non è così.

Osservando nanoparticelle d'oro e di argento, vediamo che lo spettro è
molto diverso nei due casi, a parità di solvente in cui sono immerse.
Questo il modello lo spiega. E spiega anche il redshift di
nanoparticelle in soluzione acquosa e immerse in quarzo (environment più
rigido): infatti il quarzo scherma le cariche, diminuisce la forza di
richiamo e quindi la frequenza di risonanza.

\textbf{Approssimazione di Mie: effetto delle dimensioni}

Mie è poi riuscito a osservare un effetto estrinseco sulle dimensioni.
Per nanoparticelle piccole, c'è regime quasi-statico
\(\left( R \ll \lambda \right)\): il contributo significativo è solo
quello dell'oscillazione del dipolo, i cambiamenti del campo avvengono
su tempi più brevi delle oscillazioni quindi possiamo considerare il
campo costante.

La sezione d'urto di estinzione si può scrivere con un'espressione
complicata, interrotta al termine di dipolo. Abbiamo risonanza quando il
denominatore tende a zero.

In regime dinamico, quindi se le dimensioni sono confrontabili con λ
allora non posso fare approssimazioni di campo costante e la risonanza
dipende esplicitamente dalla dimensione. Il campo non è costante, la
polarizzazione non è omogenea quindi diventano importanti i modi di
ordine superiore.

Spettroscopicamente, al crescere delle dimensioni lo spettro subisce
redshift e la banda si allarga.

Per particelle piccole \(\left( R < 10nm \right)\) c'è anche un effetto
intrinseco della dimensione. Si osserva redshift della banda di
assorbimento all'aumentare del diametro, in misura molto maggiore
rispetto all'effetto estrinseco. Inoltre, l'ampiezza della banda non
cambia in modo omogeneo all'aumentare della dimensione della particella,
ma c'è un minimo (caso dell'oro) a un raggio di 15nm. E questo non era
previsto dall'approssimazione.

Mentre l'effetto estrinseco veniva spiegato dalla teoria
dell'interazione col campo elettromagnetico, l'effetto intrinseco può
essere spiegato solo da una dipendenza esplicita delle dimensioni dalla
costante dielettrica della particella. La dipendenza ``classica'' solo
con \(\omega\) non basta a spiegare questa cosa.

Metalli nobili. C'è un termine di smorzamento dell'oscillazione degli
elettroni che modifica la risposta al campo causato dalle interazioni
fra elettroni e fononi/difetti/altri elettroni.

Con nanoparticelle piccole, di dimensione del libero cammino medio
dell'elettrone (distanza percorsa prima di un urto), quindi 40nm, gli
effetti di superficie contano perchè gli elettroni fanno scattering
sulla superficie. Tale effetto può essere modellizzato con la formula:

\[\Gamma\left( r \right) = \Gamma_{\infty} + A\frac{v_{F}}{r}\]

dove \(\Gamma_{\infty}\) è il valore di smorzamento a dimensioni
infinite della particella, modificato dall'altro fattore, che dipende
dalla velocità degli elettroni ad energia di Fermi.

Posso ricostruire la costante dielettrica con il modello che mi tenga
conto anche delle dimensioni:

\[\epsilon\left( \omega,r \right) = \epsilon_{\infty}\left( \omega \right) + costante\ di\ bulk + effetto\ di\ dimensione\]

Se inserisco nell'espressione della sezione d'urto questa espressione
della costante dielettrica, risolvo l'anomalia di Drude-Sommerfeld.

Altri fattori che influenzano l'assorbimento plasmonico di superficie:

\begin{itemize}
\item
  In caso di materiale policristallino, che da scattering sul confine
  dei cristalli, c'è un aumento della frequenza di smorzamento e un
  allargamento della banda.
\item
  Temperatura alta: aumenta la probabilità di urti quindi c'è un
  allargamento della banda.
\item
  Concentrazioni alte: cambia la costante dielettrica del mezzo e quindi
  la forma della banda (comparsa di un secondo picco con aggregati).
\item
  CID (chemical interfacing damping): in presenza di sostanze assorbite
  (chemioassorbite o fisioassorbite sulla superficie c'è allargamento e
  redshift della banda.
\end{itemize}

\textbf{Teoria di Gans: effetto della forma}

Le nanoparticelle possono avere anche forme diverse, oltre che
dimensioni. Avendo ellissoidi, si possono individuare una banda
trasversale (oscillano sull'asse minore) e una banda longitudinale
(oscillano sull'asse maggiore).

La banda di assorbimento si separa in due componenti all'aumentare
dell'aspect ratio \(R = h/w\) (lunghezza/larghezza) cioè la separazione
aumenta all'aumentare dell'ar (quanto sono allungate le particelle).

Abbiamo quindi una prima banda fissa, che coincide con il plasmone delle
corrispondenti nanoparticelle sferiche (trasversale) e una seconda banda
(longitudinale) che si sposta a λ maggiori a seconda di quanto è lungo
l'asse maggiore.

Per sintetizzare i rod si parte da un seed di nanoparticelle sferiche,
ecco perché la frequenza corrispondente è fissa. Si può estendere la
teoria di Mie per avere un'espressione esatta della sezione d'urto. La
posizione del massimo è lineare sia in R che nella costante dielettrica
del mezzo:

\[\lambda_{\max} = \left( 33.34\ R - 46.31 \right)\epsilon_{\text{mezzo}} + 472\]

Nano-rods d'oro, vantaggi:

\begin{itemize}
\item
  Facilmente sintetizzabili con diverso ar
\item
  Hanno dimensioni ideali per il trasporto dei farmaci (abbastanza
  grandi per legare il farmaco ma abbastanza piccole per entrare nelle
  cellule)
\item
  Bassa tossicità per l'organismo
\item
  Ottimi fotoassorbitori di luce IR quindi ottimi per:

  \begin{itemize}
  \item
    Terapia fototermica
  \item
    Rilasciano energia come calore terapia cancro
  \end{itemize}
\item
  Nel NearIR siamo infatti nell'optical window, ideale per immagini in
  vivo (basso assorbimento del tessuto, maggiore penetrazione della
  radiazione) quindi sono ideali per imaging in vivo su tessuti
  biologici
\item
  Hanno emissione separata spettralmente dalla fluorescenza del tessuto
\item
  Sono luminescenti nel visibile.
\end{itemize}

La fotoluminescenza è osservata soprattutto in nanoparticelle anisotrope
perché danno un buon ``effetto punta'' che invece è nullo nelle
sferiche. Il rendimento quantico del processo è quindi non trascurabile
nelle particelle anisotrope. Il processo viene intensificato se fatto a
λ di risonanza plasmonica. Gli elettroni della banda di valenza vengono
eccitati in banda di conduzione, che lascia un posto vacante nella banda
di conduzione. I processi di scattering intrabanda muovono la lacuna
nella zona di Brillouin e c'è la ricombinazione delle coppie
elettrone-lacuna tramite processi non radiativi. Spettroscopicamente
abbiamo bande ampie tutto il visibile con picchi nelle zone di
Brillouin. Quindi c'è luminescenza in un ampio range di energie.

\textbf{Luminescenza a due fotoni}

La luminescenza può essere indotta sfruttando la banda nel visibile, ma
più spesso si utilizza la banda a λ maggiori, facendo quindi
luminescenza a due fotoni. Non è chiaro se sia l'assorbimento successivo
di due fotoni (assorbimento non coerente) o l'assorbimento di due fotoni
contemporanei (assorbimento coerente).

Per distinguere i due processi si può studiare come reagiscono alla
polarizzazione del campo incidente. Nel primo caso (assorbimento non
coerente) un fotone eccita un elettrone all'interno della banda di
conduzione (da sotto l'energia di Fermi a sopra) lasciando una lacuna
qui. Il secondo fotone eccita un elettrone dalla banda d alla banda di
conduzione, riempiendo la lacuna. La restante lacuna viene riempita per
ricombinazione radiativa o per produzione di plasmoni. In ogni caso la
ricombinazione è un processo interbanda, mentre l'assorbimento no: in
questo caso abbiamo un assorbimento interbanda e uno intrabanda.

Nel processo a due fotoni (assorbimento coerente) ci sono due elettroni
della banda d che passano in banda di conduzione dopo l'assorbimento
simultaneo dei due fotoni. È essenzialmente come il singolo fotone
moltiplicato per due. L'energia dei fotoni emessi dipende dalla
separazione fra le bande, che non è costante perché dipende dalla
posizione sulla zona di Brillouin quindi ciò che vediamo
spettroscopicamente è una banda ampia.

Quindi la radiazione può subire:

\begin{itemize}
\item
  Scattering
\item
  Assorbimento a doppio fotone o a singolo e la ricombinazione dà luogo
  a fotoluminescenza
\item
  Accoppiamento elettroni-fononi
\item
  Accoppiamento fononi del reticolo con fotoni del mezzo esterno. Questo
  provoca emissione di calore dalle particelle e conseguente
  riscaldamento dell'ambiente circostante.
\end{itemize}

Le scale temporali dei processi coinvolti nel riscaldamento di
nanoparticelle metalliche sono rappresentate qui:

\includegraphics[width=3.80718in,height=2.40740in]{media/image38.emf}

Il calore che irraggiano è quindi una propagazione di eccitazione,
ovvero propagazione dei modi di eccitazione.

La velocità di emissione del calore è legata alla differenza fra
temperatura raggiunta dal reticolo e temperatura della soluzione
all'interfaccia, alla superficie della nanoparticella (quante sono le
superfici di contatto) e alla conduttanza dell'interfaccia.

Aumentando troppo l'energia del laser possiamo sciogliere le particelle.

Uno degli effetti interessanti è che non solo le particelle stesse si
riscaldano, ma possono anche influire sulle proprietà radiative di
fluorofori posizionati in prossimità. Il campo elettrico sentito dal
fluoroforo viene modificato dalle interazioni con la superficie
metallica della particella, quindi col dipolo indotto. Queste
interazioni possono aumentare o diminuire il campo incidente quindi la
sua resa quantica o tempo di vita. Il quenching dipende dalla distanza
del fluoroforo e dalla geometria del sistema. È l'unico modo per
aumentare la resa quantica. Il metallo può subire:

\begin{itemize}
\item
  Quenching che provoca smorzamento delle oscillazioni quindi riduzione
  dell'intensità del campo
\item
  Amplificazione del campo incidente (effetto punta) a distanze minori
\item
  Aumento del rate di decadimento radiativo del fluoroforo quindi
  conseguenze sulla resa quantica.
\end{itemize}

In assenza del metallo lo schema è semplice, in presenza ci sono
processi che accrescono la probabilità di eccitazione ed altri processi
di decadimento per energy transfer (negativi per l'emissione radiativa).

Il processo dipende fortemente dalla distanza. A contatto (distanza
minima) il fluoroforo fa solamente quenching, e non emette più. Per
ottenere enhancement bisogna metterlo a una certa distanza. Si riesce a
modulare il rate di decadimento radiativo variando la distanza.

L'effetto di aumento della resa quantica e della diminuzione del tempo
di vita si chiama MEF (metal enhanced fluorescence): aumenta l'intensità
di fluorescenza e anche la fotostabilità (perché si riduce la
probabilità di bleaching). Con un fluoroforo di resa quantica pari a 1
questo effetto non è osservabile, la resa quantica non si altera
sensibilmente e domina il quenching.

Perché sia significativo c'è bisogno di rese quantiche più piccole,
quindi di costanti non radiative alte. In questo caso aumenta
sensibilmente la resa quantica se il fluoroforo è posizionato a distanza
opportuna dalla superficie metallica. Si ha incremento dell'intensità.

Quindi le nanoparticelle metalliche fanno da antenne per il
riscaldamento di fluorofori.

\textbf{Sintesi di nanoparticelle}

\textbf{Nanorods}

Si parte da sali di oro e si aggiungono sulfatanti, formando il seme. Il
CTAB (sulfatante) è molto tossico, quindi va poi rimosso. Al seme si
aggiunge acido ascorbico e ioni di argento (catalizzatori della
reazione) e abbiamo soluzione di crescita. Poi si aggiunge ancora CTAB.

Il CTAB agisce selettivamente sulle pareti lunghe dei rods quindi lo
accresce in una sola direzione. Accrescendo si sviluppano delle facce e
le cariche del CTAB si attaccheranno solo su di esse. Poi si aggiunge
una molecola mimetica che impedisce ai macrofagi di riconoscere le
nanoparticelle come oggetti estranei.

\textbf{Nanostars}

Si usa un sulfatante diverso, il LSB che si lega in modo diverso,
allungando la particella in direzioni diverse. A seconda della
concentrazione di sulfatante si regola la lunghezza.

\textbf{Caratteristiche nanorods}

Si fanno misure di assorbimento, misure di backscattering dinamico e di
zeta-potential per la carica. I rods hanno due picchi nello spettro.

Le nanostar hanno spettri più complicati.

Gli spettri di emissione invece sono simili nei due casi.

Le nanoparticelle si possono usare per fare sensori di concentrazioni.
Le nanoparticelle vengono funzionalizzate con streptavidine, che sono
molto sensibili alla biotina, con la quale si funzionalizza la
fluorescina o un anticorpo. Mescolandoli in soluzione si ottiene un
composto in cui streptavidina a biotina sono legate. La forma
complessiva si altera e cambiano i plasmoni. Legandovi poi altre
proteine si nota una differenza: studiamo le funzioni di correlazione
delle fluttuazioni di fluorescenza e il segnale è diverso da non legata
a legata.

L'altro utilizzo è per la terapia fototermica in cellula. Le particelle
funzionalizzate con anticorpo specifico per la cellula vengono
internalizzate e poi irraggiate con luce IR.

\end{document}
